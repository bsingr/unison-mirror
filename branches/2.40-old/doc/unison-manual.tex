\documentclass{article} 
\usepackage{alltt}
\usepackage{fullpage}
\usepackage{moreverb}
% \usepackage{hyperref}
\usepackage{hevea}

\newif\iftextversion \textversionfalse
\newif\iffull \fullfalse
\newif\ifdraft  \drafttrue

\input{texdirectives}
\input{unisonversion}

\newcommand{\finish}[1]{\ifdraft{\ifhevea\red\else \large\bf\fi [#1]\ifhevea\fi}\fi}
\newcommand{\finishlater}[1]{}
\newcommand{\fortrevor}[1]{}

\newcommand{\CLIENT}{\iftextversion CLIENT \else {\em client}\fi}
\newcommand{\SERVER}{\iftextversion SERVER \else {\em server}\fi}

\newcommand{\showtt}[1]{%
  \ifhevea
    \iftextversion
      "#1"%
    \else
      {\large\tt #1}%
    \fi
  \else
    {\tt #1}%
  \fi
}

\makeatletter
\def\@opentoc#1{\begingroup
  \makeatletter
  \if@filesw \expandafter\newwrite\csname tf@#1\endcsname
             \immediate\openout \csname tf@#1\endcsname \jobname.#1\relax
  \fi \global\@nobreakfalse \endgroup}
\newcommand{\TABLEOFCONTENTS}{%
  \ifhevea
    \iftextversion\else
      \section*{Contents}
      \begin{quote}
      \documentclass{article} 
\usepackage{alltt}
\usepackage{fullpage}
\usepackage{moreverb}
% \usepackage{hyperref}
\usepackage{hevea}

\newif\iftextversion \textversionfalse
\newif\iffull \fullfalse
\newif\ifdraft  \drafttrue

\input{texdirectives}
\input{unisonversion}

\newcommand{\finish}[1]{\ifdraft{\ifhevea\red\else \large\bf\fi [#1]\ifhevea\fi}\fi}
\newcommand{\finishlater}[1]{}
\newcommand{\fortrevor}[1]{}

\newcommand{\CLIENT}{\iftextversion CLIENT \else {\em client}\fi}
\newcommand{\SERVER}{\iftextversion SERVER \else {\em server}\fi}

\newcommand{\showtt}[1]{%
  \ifhevea
    \iftextversion
      "#1"%
    \else
      {\large\tt #1}%
    \fi
  \else
    {\tt #1}%
  \fi
}

\makeatletter
\def\@opentoc#1{\begingroup
  \makeatletter
  \if@filesw \expandafter\newwrite\csname tf@#1\endcsname
             \immediate\openout \csname tf@#1\endcsname \jobname.#1\relax
  \fi \global\@nobreakfalse \endgroup}
\newcommand{\TABLEOFCONTENTS}{%
  \ifhevea
    \iftextversion\else
      \section*{Contents}
      \begin{quote}
      \input{unison-manual.htoc}
      \end{quote}
    \fi
  \else
    \@opentoc{htoc}
    \tableofcontents
  \fi
}
\makeatother

\newcommand{\SNIP}[2]{%
\ifhevea\iftextversion
\begin{rawhtml}<pre>----SNIP----
\end{rawhtml}
#1
#2 %
\begin{rawhtml}</pre>\end{rawhtml}%
\fi\fi
}

\newcommand{\sectionref}[2]{%
\ifhevea
  \iftextversion
    the section ``#2''
  \else
    the \url{##1}{#2} section%
  \fi
\else
  Section~\ref{#1} {[#2]}%
\fi
}

\newcommand{\bcpurl}[1]{\url{#1}}

\newcommand{\urlref}[2]{\bcpurl{##1}{#2}}
\newcommand{\ONEURL}[1]{%
  \iftextversion#1\else{\def~{\symbol{"7E}}\oneurl{#1}}\fi}
\newcommand{\URL}[2]{%
  \iftextversion#2 (#1)\else\bcpurl{#1}{#2}\fi}
\newcommand{\SHOWURL}[2]{%
  \ifhevea\URL{#1}{#2}\else#2\footnote{{\def~{\symbol{"7E}}\tt #1}}\fi}

% Usage: \SECTION{Title and menu item name}{tex label}{man section id}
\newcommand{\SECTION}[3]{%
  \ifhevea
    \SNIP{#1}{#3}%
    \iftextversion\else \@print{<hr>}\fi%
    \section*{\label{#2}#1}%
  \else
    \newpage
    \section{\label{#2}#1}%
    \addtocontents{htoc}{{\string\large\string\bf\string\urlref{#2}{#1}}\\}%
  \fi
}

\newcommand{\SUBSECTION}[2]{%
  \ifhevea
    \subsection*{\label{#2}#1}%
  \else
    \subsection{\label{#2}#1}%
    \addtocontents{htoc}{\hspace{10em}\bullet\string\urlref{#2}{#1}\\}
  \fi
}

\newcommand{\SUBSUBSECTION}[2]{%
  \ifhevea
    \subsubsection*{\label{#2}#1}%
  \else
    \subsubsection{\label{#2}#1}%
    \addtocontents{htoc}{\hspace{18em}\string\urlref{#2}{#1}\\}
  \fi
}

\newcommand{\TOPSUBSECTION}[2]{%
  \ifhevea\SNIP{#1}{#2}\fi
  \SUBSECTION{#1}{#2}%
}

% The quote-based macros looks a imperfect, perhaps due to the lack of 
% alignment
% \newenvironment{textui}{{\em Textual Interface:}\begin{quote}}{\end{quote}}
% \newenvironment{tkui}{{\em Graphical Interface:}\begin{quote}}{\end{quote}}
\newenvironment{textui}{\medskip{\em Textual Interface:}\begin{itemize}\item[]
  }{\end{itemize}}
\newenvironment{tkui}{\medskip{\em Graphical Interface:}\begin{itemize}\item[]
  }{\end{itemize}}
\newenvironment{changesfromversion}[1]{%
  \noindent Changes since #1:
  \begin{itemize}
}{
  \end{itemize}
}

\newcommand{\incompatible}{%
  \iftextversion
    INCOMPATIBLE CHANGE:
  \else
    {\bf Incompatible change:} 
  \fi}

\newcommand{\UNISONUSERS}{\URL{mailto:unison-users@yahoogroups.com}{{\tt
      unison-users@yahoogroups.com}}}
\newcommand{\UNISONHACKERS}{\URL{mailto:unison-hackers@yahoogroups.com}{{\tt
      unison-hackers@yahoogroups.com}}}

\ifhevea
 \makeatletter
 \let\oldmeta=\@meta
 \renewcommand{\@meta}{%
 \oldmeta
\ifdraft
 \begin{rawhtml}
 <META name="Author" content="Benjamin C. Pierce">
 <link rel="stylesheet" href="/home/bcpierce/pub/unison/unison.css">
 \end{rawhtml}
\else
 \begin{rawhtml}
 <META name="Author" content="Benjamin C. Pierce">
 <link rel="stylesheet" href="http://www.cis.upenn.edu/~bcpierce/unison/unison.css">
 \end{rawhtml}
\fi
}
 \makeatother
\fi


\fulltrue

%\newcommand{\NT}[1]{\(\langle\)\textit{#1}\(\rangle\)}
\newcommand{\NT}[1]{\textit{#1}}
\newcommand{\ARG}[1]{\texttt{\textit{#1}}}

%%%%%%%%%%%%%%%%%%%%%%%%%%%%%%%%%%%%%%%%%%%%%%%%%%%%%%%%%%%%%%%%%%%%%%
%%%%%%%%%%%%%%%%%%%%%%%%%%%%%%%%%%%%%%%%%%%%%%%%%%%%%%%%%%%%%%%%%%%%%%
\begin{document}

\ifhevea\begin{rawhtml}<div id="manualbody">\end{rawhtml}\fi

\ifhevea\else\bigskip\fi%
\ifdraft%
\begin{center}%
{\Huge \ifhevea\red\fi DraftDraftDraftDraft}%
\end{center}%
\ifhevea\else \bigskip \fi
\fi

\ifhevea\begin{rawhtml}<div id="manualheader">\end{rawhtml}%
\else \thispagestyle{empty}
\fi%
\SNIP{About Unison}{about}%
\iftextversion
  \section*{Unison File Synchronizer 
%%   \\ 
%%   \ONEURL{http://www.cis.upenn.edu/\home{bcpierce}/unison}
  \\
  Version
  \unisonversion 
  }
\else%
  \ifhevea\else \vspace*{2in} \fi%
  \begin{center}%
  \Huge{\ifhevea\black\else\bf \fi Unison File  Synchronizer}%
%%  \ifhevea \\ \else \\[2ex] \fi
%%   \large
%%   \ONEURL{http://www.cis.upenn.edu/\home{bcpierce}/unison}
  \ifhevea \\ \else \\[2ex] \fi%
  \huge {\ifhevea\black\else\bf \fi User Manual and Reference Guide}%
  \ifhevea \\ \else \\[6ex] \fi%
  \LARGE%
  Version \unisonversion \\ %
  % \today %
  \end{center}%
\fi%
%
%
\ifhevea\begin{rawhtml}</div>\end{rawhtml}\fi

\ifhevea\else\newpage\fi 
\TABLEOFCONTENTS
\ifhevea\else\newpage\fi

\ifhevea\else\iftextversion\else \section*{Overview}\fi\fi

Unison is a file-synchronization tool for Unix and Windows.  It allows
two replicas of a collection of files and directories to be stored on
different hosts (or different disks on the same host), modified
separately, and then brought up to date by propagating the changes in
each replica to the other.

Unison 
shares a number of features with tools such as configuration
management packages %
(\URL{http://www.cyclic.com/}{CVS},
\URL{http://www.XCF.Berkeley.EDU/\home{jmacd}/prcs.html}{PRCS},
etc.),
%
distributed filesystems 
(\URL{http://www.coda.cs.cmu.edu/}{Coda}, 
etc.),
%
uni-directional mirroring utilities 
(\URL{http://samba.anu.edu.au/rsync/}{rsync}, 
etc.),
%
and other synchronizers 
(\URL{http://www.pumatech.com}{Intellisync}, 
\URL{http://www.merl.com/reports/TR99-14/}{Reconcile},
etc).  \finishlater{Midnight commander??}
%
However, there are several points where it differs:
\begin{itemize}
\item Unison runs on both Windows (95, 98, NT, 2k, and XP) and Unix (OSX, Solaris,
  Linux, etc.) systems.  Moreover, Unison works {\em across}
  platforms, allowing you to synchronize a Windows laptop with a
  Unix server, for example.
\item Unlike a distributed filesystem, Unison is a user-level program:
  there is no need to modify the kernel or to have
  superuser privileges on either host.
\item Unlike simple mirroring or backup utilities, Unison can deal
  with updates to both replicas of a distributed directory structure.
  Updates that do not conflict are propagated automatically.
  Conflicting updates are detected and displayed.
\item Unison works between any pair of machines connected to the
  internet, communicating over either a direct socket link or
  tunneling over an encrypted {\tt ssh} connection.
  It is careful with network bandwidth, and runs well over slow links
  such as PPP connections.  Transfers of small updates to large files are
  optimized using a compression protocol similar to rsync.
\item Unison has a clear and precise specification\iffull, described
below. \else. \fi
  \item Unison is resilient to failure.  It is careful to leave the
  replicas and its own private structures in a sensible state at all
  times, even in case of abnormal termination or communication
  failures.
% \item Unison is easy to install.  Just one executable file (for each
%   host architecture) is all you need.
\item Unison is free; full source code is available under the GNU
Public License.
\end{itemize}



\ifhevea\else\bigskip\fi

% \begin{quote}
% {\bf\ifhevea\red\fi Warning:} The current implementation of Unison is
% considered beta-test software.  It is in daily use by quite a few
% people, but there are still undoubtedly some bugs.  If you choose to 
% use it to synchronize important data, please pay careful attention
% to what it is doing!  Also, the installation/setup procedure is not
% yet as smooth as we want it to be.
% \end{quote}


\SECTION{Preface}{intro}{ }

\TOPSUBSECTION{People}{people}

\URL{http://www.cis.upenn.edu/\home{bcpierce}/}{Benjamin Pierce} leads the
Unison project.   
%
The current version of Unison was designed and implemented by
    \URL{http://www.research.att.com/\home{trevor}/}{Trevor Jim},
    \URL{http://www.cis.upenn.edu/\home{bcpierce}/}{Benjamin Pierce},
and
    J\'{e}r\^{o}me Vouillon,
with
    \URL{http://www.inrialpes.fr/\home{aschmitt}/}{Alan Schmitt},
    {Malo Denielou},
    \URL{http://www.brics.dk/\home{zheyang}/}{Zhe Yang},
    Sylvain Gommier, and
    Matthieu Goulay.
%
Our implementation of the
  \URL{http://samba.org/rsync/}{rsync}
  protocol was built by
  \URL{http://www.eecs.harvard.edu/\home{nr}/}{Norman Ramsey}
  and Sylvain Gommier.  It is is based on
  \URL{http://samba.anu.edu.au/\home{tridge}/}{Andrew Tridgell}'s
  \URL{http://samba.anu.edu.au/\home{tridge}/phd\_thesis.pdf}{thesis work}
  and inspired by his
  \URL{http://samba.org/rsync/}{rsync}
  utility.
% \finish{Our low-level fingerprinting implementation uses an algorithm
% by Michael Rabin and incorporates some coding tricks from Andrei
% Broder and Mike Burrows.}
%
The mirroring and merging functionality was implemented by
  Sylvain Roy and improved by Malo Denielou.
%
 \URL{http://wwwfun.kurims.kyoto-u.ac.jp/\home{garrigue}/}{Jacques Garrigue}
 contributed the original Gtk version of the user
  interface; the Gtk2 version was built by Stephen Tse. 
%
 Sundar Balasubramaniam helped build a prototype implementation of
an earlier synchronizer in Java.
\URL{http://www.cis.upenn.edu/\home{ishin}/}{Insik Shin}
and
\URL{http://www.cis.upenn.edu/\home{lee}/}{Insup Lee} contributed design
ideas to this implementation.
\URL{http://research.microsoft.com/\home{fournet}/}{Cedric Fournet}
contributed to an even earlier prototype.

\TOPSUBSECTION{Mailing Lists and Bug Reporting}{lists}

\paragraph{Mailing Lists:}

Moderated mailing lists are available for announcements of new
versions, discussions among users, and discussions among developers.
See \ONEURL{http://www.cis.upenn.edu/~bcpierce/unison/lists.html} for
more information.

\endinput
%%%%%%%%%%%%%%%%%%%%%%%%%%%%%%%%%%%%%%%%%%%%%%%%%%%%%%%%%%%%%%%%%%%%%%%%%%%%%%
%%%%%%%%%%%%%%%%%%%%%%%%%%%%%%%%%%%%%%%%%%%%%%%%%%%%%%%%%%%%%%%%%%%%%%%%%%%%%%
%%%%%%%%%%%%%%%%%%%%%%%%%%%%%%%%%%%%%%%%%%%%%%%%%%%%%%%%%%%%%%%%%%%%%%%%%%%%%%
%%%%%%%%%%%%%%%%%%%%%%%%%%%%%%%%%%%%%%%%%%%%%%%%%%%%%%%%%%%%%%%%%%%%%%%%%%%%%%
%%%%%%%%%%%%%%%%%%%%%%%%%%%%%%%%%%%%%%%%%%%%%%%%%%%%%%%%%%%%%%%%%%%%%%%%%%%%%%

It is strongly recommended that all Unison users subscribe to one of the
first two:  
\begin{itemize}
\item {\bf
  \URL{http://groups.yahoo.com/group/unison-announce}{unison-announce}}
is a moderated list where new Unison releases are announced.  It is very
low volume, averaging about one message per month. 

To subscribe, you can either visit 
  \ONEURL{http://groups.yahoo.com/group/unison-announce} (you will be
  asked to create a Yahoo groups account if you do not already have one),
  or else send an email to
  {\tt
  \URL{mailto:unison-announce-subscribe@groups.yahoo.com}{unison-announce-subscribe@groups.yahoo.com}}
  (which will 
  simply add you to the list, whether you have a Yahoo account or not).

  \item {\bf
    \URL{http://groups.yahoo.com/group/unison-users}{unison-users}} is a
  somewhat-higher-volume list for users of unison.  It is used for
  discussions of many sorts --- proposals and designs for new features,
  installation and configuration questions, usage tips, etc.  It is also
  moderated, but just to filter spam.

To subscribe, you can either visit 
  \ONEURL{http://groups.yahoo.com/group/unison-users}
  or else send an email to
  {\tt \URL{mailto:unison-users-subscribe@groups.yahoo.com}{unison-users-subscribe@groups.yahoo.com}}.

Release announcements are made on both of these lists, so there is
no need to subscribe to both.

\item {\bf
  \URL{}{unison-hackers}} is
for informal discussion among Unison developers.  Anyone who considers
themselves a Unison expert and wishes to lend a hand with maintaining and
improving Unison is welcome to join.  Only members can post to this list. 

To subscribe, you can either visit 
  \ONEURL{http://lists.seas.upenn.edu/mailman/listinfo/unison-hackers}
  or else send an email to
  {\tt unison-hackers-subscribe at lists dot seas dot upenn dot edu}.
\end{itemize}
Archives of all the lists are available via the
above links. Postings are limited to members, to reduce the spam load on moderators.

\paragraph{Reporting bugs:}

If Unison is not working the way you expect, here are some steps to 
follow.
\begin{itemize}
\item First, take a look at the Unison documentation, especially the
FAQ section.  Lots of questions are answered there.

\item Next, try running Unison with the {\tt -debug all} command line
option.  This will cause Unison to generate a detailed trace of what it's
doing, which may help pinpoint where the problem is occurring.

\item If this doesn't clarify matters, try sending an email describing
your problem to the users list at
\URL{mailto:unison-users@groups.yahoo.com}{{\tt 
    unison-users@groups.yahoo.com}}.  
Please include the version of Unison you are using (type {\tt unison
  -version}), the kind of machine(s) you are running it on, a record of
what gets printed when the {\tt -debug all} option is included, and as
much information as you can about what went wrong.
\end{itemize}

\paragraph{Feature Requests:}

Please post your feature requests, suggestions, etc. to the {\tt
  unison-users} list.   



\TOPSUBSECTION{Development Status}{status}

Unison is no longer under active development as a research
project.  (Our research efforts  are now focused on a follow-on
project called Harmony, described at
\ONEURL{http://www.cis.upenn.edu/\home{bcpierce}/harmony}.) 
At this point, there is no one whose job it is to maintain Unison,
fix bugs, or answer questions.

However, the original developers are all still using Unison daily.  It
will continue to be maintained and supported for the foreseeable future,
and we will occasionally release new versions with bug fixes, small
improvements, and contributed patches.

Reports of bugs affecting correctness or safety are of interest to many
people and will generally get high priority.  Other bug reports will be
looked at as time permits.  Bugs should be reported to the users list at
\UNISONUSERS. 

Feature requests are welcome, but will probably just be added to the
ever-growing todo list.  They should also be sent to \UNISONUSERS.

Patches are even more welcome.  They should be sent to
\UNISONHACKERS.
(Caveat: since safety and robustness are Unison's most important properties,
patches will be held to high standards of clear design and clean coding.)
If you want to contribute to Unison, start by downloading the developer
tarball from the download page.  For some details on how the code is
organized, etc., see the file {\tt CONTRIB}.

\TOPSUBSECTION{Copying}{copying}

Unison is free software.  You are free to change and redistribute it
under the terms of the GNU General Public License.  Please see the
file COPYING in the Unison distribution for more information.

\TOPSUBSECTION{Acknowledgements}{ack}

Work on Unison has been supported by the National Science Foundation
under grants CCR-9701826 and ITR-0113226, {\em Principles and Practice of
  Synchronization}, and by University of Pennsylvania's Institute for
Research in Cognitive Science (IRCS).

\SECTION{Installation}{install}{install}

Unison is designed to be easy to install.  The following sequence of
steps should get you a fully working installation in a few minutes.  If
you run into trouble, you may find the suggestions on the 
\SHOWURL{http://www.cis.upenn.edu/\home{bcpierce}/unison/faq.html}{Frequently Asked
Questions page} helpful.  Pre-built binaries are available for a
variety of platforms.

Unison can be used with either of two user interfaces: 
\begin{enumerate}
\item a simple textual interface, suitable for dumb terminals (and
running from scripts), and 
\item a more sophisticated grapical interface, based on Gtk2.  
\end{enumerate}

You will need to install a copy of Unison on every machine that you
want to synchronize.  However, you only need the version with a
graphical user interface (if you want a GUI at all) on the machine
where you're actually going to display the interface (the \CLIENT{}
machine).  Other machines that you synchronize with can get along just
fine with the textual version.


\SUBSECTION{Downloading Unison}{download}

The Unison download site lives under 
\ONEURL{http://www.cis.upenn.edu/\home{bcpierce}/unison}.

If a pre-built binary of Unison is available for the client machine's
architecture, just download it and put it somewhere in your search
path (if you're going to invoke it from the command line) or on your
desktop (if you'll be click-starting it).

The executable file for the graphical version (with a name including
\verb|gtkui|) actually provides {\em both} interfaces: the graphical one
appears by default, while the textual interface can be selected by including
\verb|-ui text| on the command line.  The \verb|textui| executable
provides just the textual interface.

If you don't see a pre-built executable for your architecture, you'll
need to build it yourself.  See \sectionref{building}{Building Unison}.
There are also a small number of contributed ports to other
architectures that are not maintained by us.  See the
\SHOWURL{http://www.cis.upenn.edu/\home{bcpierce}/unison/download.html}{Contributed 
Ports page} to check what's available.

Check to make sure that what you have downloaded is really executable.
Either click-start it, or type \showtt{unison -version} at the command
line.  

Unison can be used in three different modes: with different directories on a
single machine, with a remote machine over a direct socket connection, or
with a remote machine using {\tt ssh} for authentication and secure
transfer.  If you intend to use the last option, you may need to install
{\tt ssh}; see \sectionref{ssh}{Installing Ssh}.

\SUBSECTION{Running Unison}{afterinstall} 

Once you've got Unison installed on at least one system, read 
\sectionref{tutorial}{Tutorial} of the user manual (or type \showtt{unison -doc
  tutorial}) for instructions on how to get started.


\SUBSECTION{Upgrading}{upgrading}

Upgrading to a new version of Unison is as simple as throwing away the old
binary and installing the new one.

Before upgrading, it is a good idea to run the {\em old} version one last
time, to make sure all your replicas are completely synchronized.  A new
version of Unison will sometimes introduce a different format for the
archive files used to remember information about the previous state of the
replicas.  In this case, the old archive will be ignored (not deleted --- if
you roll back to the previous version of Unison, you will find the old
archives intact), which means that any differences between the replicas will
show up as conflicts that need to be resolved manually.


\SUBSECTION{Building Unison from Scratch}{building}

If a pre-built image is not available, you will need to compile it from
scratch; the sources are available from the same place as the binaries.

In principle, Unison should work on any platform to which OCaml has been
ported and on which the \verb|Unix| module is fully implemented.  It has
been tested on many flavors of Windows (98, NT, 2000, XP) and Unix (OS X,
Solaris, Linux, FreeBSD), and on both 32- and 64-bit architectures.


\SUBSUBSECTION{Unix}{build-unix}

You'll need the Objective Caml compiler (version 3.07 or later), which is
available from \ONEURL{http://caml.inria.fr}.  Building and installing OCaml
on Unix systems is very straightforward; just follow the instructions in the
distribution.  You'll probably want to build the native-code compiler in
addition to the bytecode compiler, as Unison runs much faster when compiled
to native code, but this is not absolutely necessary.
%
(Quick start: on many systems, the following sequence of commands will
get you a working and installed compiler: first do {\tt make world opt},
then {\tt su} to root and do {\tt make install}.)

You'll also need the GNU {\tt make} utility, standard on many Unix
systems. (Type \showtt{make --version} to check that you've got the
GNU version.)

Once you've got OCaml installed, grab a copy of the Unison sources,
unzip and untar them, change to the new \showtt{unison} directory, and
type ``{\tt make UISTYLE=text}.''
The result should be an executable file called \showtt{unison}.
Type \showtt{./unison} to make sure the program is executable.  You
should get back a usage message.

If you want to build the graphical user interface, you will need to install
two additional things:
\begin{itemize}
\item The Gtk2 libraries.  These areavailable from
  \ONEURL{http://www.gtk.org} and are standard on many Unix installations.   
  
\item The {\tt lablgtk2} OCaml library.  Grab the
  developers' tarball from
  \begin{quote}
  \ONEURL{http://wwwfun.kurims.kyoto-u.ac.jp/soft/olabl/lablgtk.html},
  \end{quote}
  untar it, and follow the instructions to build and install it.

  (Quick start: {\tt make configure}, then {\tt make}, then {\tt make
  opt}, then {\tt su} and {\tt make install}.)
\end{itemize}

Now build unison.  If your search paths are set up correctly, simply typing
{\tt make}
again should build a \verb|unison| executable with a Gtk2 graphical
interface.  (In previous releases of Unison, it was necessary to add {\tt
  UISTYLE=gtk2} to the 'make' command above.  This requirement has been
removed: the makefile should detect automatically when lablgtk2 is
present and set this flag automatically.)  

Put the \verb|unison| executable somewhere in your search path, either
by adding the Unison directory to your PATH variable or by copying the
executable to some standard directory where executables are stored.

\SUBSUBSECTION{Windows}{build-win}

Although the binary distribution should work on any version of Windows,
some people may want to build Unison from scratch on those systems too.

\paragraph{Bytecode version:} The simpler but slower compilation option
to build a Unison executable is to build a bytecode version.  You need
first install Windows version of the OCaml compiler (version 3.07 or
later, available from \ONEURL{http://caml.inria.fr}).  Then grab a copy
of Unison sources and type  
\begin{verbatim}
       make NATIVE=false
\end{verbatim}
to compile the bytecode.  The result should be an executable file called
\verb|unison.exe|. 

\paragraph{Native version:} Building a more efficient, native version of
Unison on Windows requires a little more work.  See the file {\tt
  INSTALL.win32} in the source code distribution.


\SUBSUBSECTION{Installation Options}{build-opts}

The \verb|Makefile| in the distribution includes several switches that
can be used to control how Unison is built.  Here are the most useful
ones:
\begin{itemize}
\item Building with \verb|NATIVE=true| uses the native-code OCaml
compiler, yielding an executable that will run quite a bit faster. We use
this for building distribution versions.
\item Building with \verb|make DEBUGGING=true| generates debugging
symbols. 
\item Building with \verb|make STATIC=true| generates a (mostly)
statically linked executable.  We use this for building distribution
versions, for portability.
\end{itemize}
%\finish{Any other important ones?}


\SECTION{Tutorial}{tutorial}{tutorial}

%\finish{Put a pointer somewhere in here to the typical profile in the
%  reference section.}

\SUBSECTION{Preliminaries}{prelim}

Unison can be used with either of two user interfaces: 
\begin{enumerate}
\item a straightforward textual interface and 
\item a more sophisticated graphical interface
\end{enumerate}
The textual interface is more convenient for running from scripts and
works on dumb terminals; the graphical interface is better for most
interactive use.  For this tutorial, you can use either.  If you are running
Unison from the command line, just typing {\tt unison} 
will select either the text or the graphical interface, depending on which
has been selected as default when the executable you are running was
built.  You can force the text interface even if graphical is the default by
adding {\tt -ui text}.  
The other command-line arguments to both versions are identical.  

The graphical version can also be run directly by clicking on its icon, but
this may require a little set-up (see \sectionref{click}{Click-starting
  Unison}).  For this tutorial, we assume that you're starting it from the
command line.

Unison can synchronize files and directories on a single machine, or
between two machines on a network.  (The same program runs on both
machines; the only difference is which one is responsible for
displaying the user interface.)  If you're only interested in a
single-machine setup, then let's call that machine the \CLIENT{}.  If
you're synchronizing two machines, let's call them \CLIENT{} and
\SERVER.

\SUBSECTION{Local Usage}{local}

Let's get the client machine set up first and see how to synchronize
two directories on a single machine.

Follow the instructions in \sectionref{install}{Installation} to either
download or build an executable version of Unison, and install it
somewhere on your search path.  (If you just want to use the textual user
interface, download the appropriate textui binary.  If you just want to
the graphical interface---or if you will use both interfaces [the gtkui
binary actually has both compiled in]---then download the gtkui binary.)

Create a small test directory {\tt a.tmp} containing a couple of files
and/or subdirectories, e.g.,
\begin{verbatim}
       mkdir a.tmp
       touch a.tmp/a a.tmp/b
       mkdir a.tmp/d
       touch a.tmp/d/f
\end{verbatim}
Copy this directory to b.tmp:
\begin{verbatim}
       cp -r a.tmp b.tmp
\end{verbatim}

Now try synchronizing {\tt a.tmp} and {\tt b.tmp}.  (Since they are
identical, synchronizing them won't propagate any changes, but Unison
will remember the current state of both directories so that it will be
able to tell next time what has changed.)  Type:
\begin{verbatim}
       unison a.tmp b.tmp
\end{verbatim}

\begin{textui}
You should see a message notifying you that all the files are actually
equal and then get returned to the command line.
\end{textui}

\begin{tkui}
You should get a big empty window with a message at the bottom
notifying you that all files are identical.  Choose the Exit item from
the File menu to get back to the command line.
\end{tkui}

Next, make some changes in a.tmp and/or b.tmp.  For example:
\begin{verbatim}
        rm a.tmp/a
        echo "Hello" > a.tmp/b
        echo "Hello" > b.tmp/b
        date > b.tmp/c
        echo "Hi there" > a.tmp/d/h
        echo "Hello there" > b.tmp/d/h
\end{verbatim}
Run Unison again:
\begin{verbatim}
       unison a.tmp b.tmp
\end{verbatim}

This time, the user interface will display only the files that have
changed.  If a file has been modified in just one
replica, then it will be displayed with an arrow indicating the
direction that the change needs to be propagated.  For example, 
\begin{verbatim}
                 <---  new file   c  [f]
\end{verbatim}
\noindent
indicates that the file {\tt c} has been modified only in the second
replica, and that the default action is therefore to propagate the new
version to the first replica.  To {\bf f}ollw Unison's recommendation,
press the ``f'' at the prompt.

If both replicas are modified and their contents are different, then
the changes are in conflict: \texttt{<-?->} is displayed to indicate
that Unison needs guidance on which replica should override the
other.  
\begin{verbatim}
     new file  <-?->  new file   d/h  []
\end{verbatim}
By default, neither version will be propagated and both
replicas will remain as they are.  

If both replicas have been modified but their new contents are the same
(as with the file {\tt b}), then no propagation is necessary and
nothing is shown.  Unison simply notes that the file is up to date.

These display conventions are used by both versions of the user
interface.  The only difference lies in the way in which Unison's
default actions are either accepted or overriden by the user.

\begin{textui}
The status of each modified file is displayed, in turn.  
When the copies of a file in the two replicas are not identical, the
user interface will ask for instructions as to how to propagate the
change.  If some default action is indicated (by an arrow), you can
simply press Return to go on to the next changed file.  If you want to
do something different with this file, press ``\verb|<|'' or ``\verb|>|'' to force
the change to be propagated from right to left or from left to right,
or else press ``\verb|/|'' to skip this file and leave both replicas alone.
When it reaches the end of the list of modified files, Unison will ask
you one more time whether it should proceed with the updates that have
been selected.

When Unison stops to wait for input from the user, pressing ``\verb|?|''
will always give a list of possible responses and their meanings.
\end{textui}

\begin{tkui}  
The main window shows all the files that have been modified in either
{\tt a.tmp} or {\tt b.tmp}.  To override a default action (or to select
an action in the case when there is no default), first select the file, either
by clicking on its name or by using the up- and down-arrow keys.  Then
press either the left-arrow or ``\verb|<|'' key (to cause the version in b.tmp to
propagate to a.tmp) or the right-arrow or ``\verb|>|'' key (which makes the a.tmp
version override b.tmp). 

Every keyboard command can also be invoked from the menus at the top
of the user interface.  (Conversely, each menu item is annotated with
its keyboard equivalent, if it has one.)

When you are satisfied with the directions for the propagation of changes
as shown in the main window, click the ``Go'' button to set them in
motion.  A check sign will be displayed next to each filename
when the file has been dealt with.
\end{tkui}


\SUBSECTION{Remote Usage}{remote}

Next, we'll get Unison set up to synchronize replicas on two different
machines.

Follow the instructions in the Installation section to download or
build an executable version of Unison on the server machine, and
install it somewhere on your search path.  (It doesn't matter whether
you install the textual or graphical version, since the copy of Unison on
the server doesn't need to display any user interface at all.)  

It is important that the version of Unison installed on the server
machine is the same as the version of Unison on the client machine.
But some flexibility on the version of Unison at the client side can
be achieved by using the \verb|-addversionno| option; see 
\sectionref{prefs}{Preferences}.

Now there is a decision to be made.  Unison provides two methods for
communicating between the client and the server:
\begin{itemize}
\item {\em Remote shell method}: To use this method, you must have
  some way of invoking remote commands on the server from the client's
  command line, using a facility such as \verb|ssh|.
  This method is more convenient (since there is no need to manually
  start a ``unison server'' process on the server) and also more
  secure (especially if you use \verb|ssh|).

\item {\em Socket method}: This method requires only that you can get
  TCP packets from the client to the server and back.  A draconian 
  firewall can prevent this, but otherwise it should work anywhere.
\end{itemize}

Decide which of these you want to try, and continue with
\sectionref{rshmeth}{Remote Shell Method} or
\sectionref{socketmeth}{Socket Method}, as appropriate.


\SUBSECTION{Remote Shell Method}{rshmeth}

The standard remote shell facility on Unix systems is \verb|ssh|, which provides the
same functionality as the older \verb|rsh| but much better security.  Ssh is available from
\ONEURL{ftp://ftp.cs.hut.fi/pub/ssh/}; up-to-date binaries for some
architectures can also be found at
\ONEURL{ftp://ftp.faqs.org/ssh/contrib}.  See section~\ref{ssh-win}
for installation instructions for the Windows version.

Running
\verb|ssh| requires some coordination between the client and server
machines to establish that the client is allowed to invoke commands on
the server; please refer to the or \verb|ssh| documentation
for information on how to set this up.  The examples in this section
use \verb|ssh|, but you can substitute \verb|rsh| for \verb|ssh| if
you wish.

First, test that we can invoke Unison on the server from the client.
Typing
\begin{alltt}
        ssh \NT{remotehostname} unison -version
\end{alltt}
should print the same version information as running
\begin{verbatim}
        unison -version
\end{verbatim}
locally on the client.  If remote execution fails, then either
something is wrong with your ssh setup (e.g., ``permission denied'')
or else the search path that's being used when executing commands on
the server doesn't contain the \verb|unison| executable (e.g.,
``command not found'').

Create a test directory {\tt a.tmp} in your home directory on the client
machine.  

Test that the local unison client can start and connect to the
remote server.  Type
\begin{alltt}
          unison -testServer a.tmp ssh://\NT{remotehostname}/a.tmp
\end{alltt}

Now cd to your home directory and type:
\begin{verbatim}
          unison a.tmp ssh://remotehostname/a.tmp
\end{verbatim}
The result should be that the entire directory {\tt a.tmp} is propagated
from the client to your home directory on the server.

After finishing the first synchronization, change a few files and try
synchronizing again.  You should see similar results as in the local
case.

If your user name on the server is not the same as on the client, you
need to specify it on the command line:
\begin{verbatim}
          unison a.tmp ssh://username@remotehostname/a.tmp
\end{verbatim}

\noindent {\it Notes:}
\begin{itemize}
\item If you want to put \verb|a.tmp| some place other than your home
directory on the remote host, you can give an absolute path for it by
adding an extra slash between \verb|remotehostname| and the beginning
of the path:
\begin{verbatim}
          unison a.tmp ssh://remotehostname//absolute/path/to/a.tmp
\end{verbatim}

\item You can give an explicit path for the \verb|unison| executable
  on the server by using the command-line option \showtt{-servercmd
    /full/path/name/of/unison} or adding
  \showtt{servercmd=/full/path/name/of/unison} to your profile (see
  \sectionref{profile}{Profile}).  Similarly, you can specify a
  explicit path for the \verb|ssh| program using the \showtt{-sshcmd}
  option.
  Extra arguments can be passed to \verb|ssh| by setting the
  \verb|-sshargs| preference.
\end{itemize}


\SUBSECTION{Socket Method}{socketmeth}

\begin{quote}
  {\bf\ifhevea\red\fi Warning:} The socket method is 
  insecure: not only are the texts of your changes transmitted over
  the network in unprotected form, it is also possible for anyone in
  the world to connect to the server process and read out the contents
  of your filesystem!  (Of course, to do this they must understand the
  protocol that Unison uses to communicate between client and server,
  but all they need for this is a copy of the Unison sources.)  The socket
  method is provided only for expert users with specific needs; everyone
  else should use the \verb|ssh| method.
\end{quote}

To run Unison over a socket connection, you must start a Unison
daemon process on the server.  This process runs continuously,
waiting for connections over a given socket from client machines
running Unison and processing their requests in turn.

To start the daemon, type
\begin{verbatim}
       unison -socket NNNN
\end{verbatim}
on the server machine, where {\tt NNNN} is the socket number that the
daemon should listen on for connections from clients.  ({\tt NNNN} can
be any large number that is not being used by some other program; if
\texttt{NNNN} is already in use, Unison will exit with an error
message.)  Note that paths specified by the client will be interpreted
relative to the directory in which you start the server process; this
behavior is different from the ssh case, where the path is relative to
your home directory on the server.

Create a test directory {\tt a.tmp} in your home directory on the
client machine.  Now type:
\begin{alltt}
       unison a.tmp socket://\NT{remotehostname}:NNNN/a.tmp
\end{alltt}
The result should be that the entire directory {\tt a.tmp} is
propagated from the client to the server (\texttt{a.tmp} will be
created on the server in the directory that the server was started
from).
%
After finishing the first synchronization, change a few files and try
synchronizing again.  You should see similar results as in the local
case.

Since the socket method is not used by many people, its functionality is
rather limited.  For example, the server can only deal with one client at a
time. 


\SUBSECTION{Using Unison for All Your Files}{usingit}

Once you are comfortable with the basic operation of Unison, you may
find yourself wanting to use it regularly to synchronize your commonly
used files.  There are several possible ways of going about this:

\begin{enumerate}
\item Synchronize your whole home directory, using the Ignore facility
(see \sectionref{ignore}{Ignore})
to avoid synchronizing temporary files and things that only belong on
one host.
\item Create a subdirectory called {\tt shared} (or {\tt current}, or
whatever) in your home directory on each host, and put all the files
you want to synchronize into this directory.  
\item Create a subdirectory called {\tt shared} (or {\tt current}, or
whatever) in your home directory on each host, and put {\em links to}
all the files you want to synchronize into this directory.  Use the
{\tt follow} preference (see \sectionref{symlinks}{Symbolic Links}) to make
Unison treat these links as transparent.
\item Make your home directory the root of the synchronization, but
tell Unison to synchronize only some of the files and subdirectories
within it on any given run.  This can be accomplished by using the {\tt -path} switch
on the command line:
\begin{alltt}
       unison /home/\NT{username} ssh://\NT{remotehost}//home/\NT{username} -path shared
\end{alltt}
The {\tt -path} option can be used as many times as needed, to 
synchronize several files or subdirectories:
\begin{alltt}
       unison /home/\NT{username} ssh://\NT{remotehost}//home/\NT{username} \verb|\|
          -path shared \verb|\|
          -path pub \verb|\|
          -path .netscape/bookmarks.html
\end{alltt}
These \verb|-path| arguments can also be put in your preference file.
See \sectionref{prefs}{Preferences} for an example.
\end{enumerate}

Most people find that they only need to maintain a profile (or
profiles) on one of the hosts that they synchronize, since Unison is
always initiated from this host.  (For example, if you're
synchronizing a laptop with a fileserver, you'll probably always run
Unison on the laptop.)  This is a bit different from the usual
situation with asymmetric mirroring programs like \verb|rdist|, where
the mirroring operation typically needs to be initiated from the
machine with the most recent changes.  \sectionref{profile}{Profile}
covers the syntax of Unison profiles, together with some sample profiles.

Some tips on improving Unison's performance can be found on the
\SHOWURL{http://www.cis.upenn.edu/\home{bcpierce}/unison/faq.html}{Frequently
  Asked Questions page}.

\SUBSECTION{Using Unison to Synchronize More Than Two Machines}{usingmultiple}

Unison is designed for synchronizing pairs of replicas.  However, it is
possible to use it to keep larger groups of machines in sync by performing
multiple pairwise synchronizations.  

If you need to do this, the most reliable way to set things up is to
organize the machines into a ``star topology,'' with one machine designated
as the ``hub'' and the rest as ``spokes,'' and with each spoke machine
synchronizing only with the hub.  The big advantage of the star topology is
that it eliminates the possibility of confusing ``spurious conflicts''
arising from the fact that a separate archive is maintained by Unison for
every pair of hosts that it synchronizes.


\SUBSECTION{Going Further}{further}

On-line documentation for the various features of Unison
can be obtained either by typing
\begin{verbatim}
        unison -doc topics
\end{verbatim}
\noindent
at the command line, or by selecting the Help menu in the graphical
user interface.  
\iftextversion
The same information is also available in a typeset User's
Manual (HTML or PostScript format) through
\ONEURL{http://www.cis.upenn.edu/\home{bcpierce}/unison}. 
\else
The on-line information and the printed manual are essentially identical.
\fi

If you use Unison regularly, you should subscribe to one of the mailing
lists, to receive announcements of new versions.  See
\sectionref{lists}{Mailing Lists}. 

\SECTION{Basic Concepts}{basics}{basics}

To understand how Unison works, it is necessary to discuss a few
straightforward concepts.
%
These concepts are developed more rigorously and at more length in a number
of papers, available at \ONEURL{http://www.cis.upenn.edu/\home{bcpierce}/papers}.
But the informal presentation here should be enough for most users.


\SUBSECTION{Roots}{roots}

A replica's {\em root} tells Unison where to find a set of files to be
synchronized, either on the local machine or on a remote host.
For example,
\begin{alltt}
      \NT{relative/path/of/root}
\end{alltt}
\noindent
specifies a local root relative to the directory where Unison is
started, while
\begin{alltt}
      /\NT{absolute/path/of/root}
\end{alltt}
\noindent
specifies a root relative to the top of the local filesystem,
independent of where Unison is running.  Remote roots can begin with
\verb|ssh://|,
\verb|rsh://|
to indicate that the remote server should be started with rsh or ssh:
\begin{alltt}
      ssh://\NT{remotehost}//\NT{absolute/path/of/root}
      rsh://\NT{user}@\NT{remotehost}/\NT{relative/path/of/root}
\end{alltt}
If the remote server is already running (in the socket mode), then the syntax
\begin{alltt}
      socket://\NT{remotehost}:\NT{portnum}//\NT{absolute/path/of/root}
      socket://\NT{remotehost}:\NT{portnum}/\NT{relative/path/of/root}
\end{alltt}
\noindent
is used to specify the hostname and the port that the client Unison should
use to contact it.

The syntax for roots is based on that of URIs (described in RFC 2396).
The full grammar is: 
\begin{alltt}
  \NT{replica} ::= [\NT{protocol}:]//[\NT{user}@][\NT{host}][:\NT{port}][/\NT{path}]
           |  \NT{path}

  \NT{protocol} ::= file
            |  socket
            |  ssh
            |  rsh

  \NT{user} ::= [-_a-zA-Z0-9]+

  \NT{host} ::= [-_a-zA-Z0-9.]+

  \NT{port} ::= [0-9]+
\end{alltt}
When \verb|path| is given without any protocol prefix, the protocol is
assumed to be \verb|file:|.  Under Windows, it is possible to
synchronize with a remote directory using the \verb|file:| protocol over
the Windows Network Neighborhood.  For example,
\begin{verbatim}
       unison foo //host/drive/bar
\end{verbatim}
\noindent
synchronizes the local directory \verb|foo| with the directory
\verb|drive:\bar| on the machine \verb|host|, provided that \verb|host|
is accessible via Network Neighborhood.  When the \verb|file:| protocol
is used in this way, there is no need for a Unison server to be running
on the remote host.  However, running Unison this way is only a good
idea if the remote host is reached by a very fast network connection,
since the full contents of every file in the remote replica will have to
be transferred to the local machine to detect updates.

The names of roots are {\em canonized} by Unison before it uses them
to compute the names of the corresponding archive files, so {\tt
  //saul//home/bcpierce/common} and {\tt //saul.cis.upenn.edu/common}
will be recognized as the same replica under different names.

\SUBSECTION{Paths}{paths}

A {\em path} refers to a point {\em within} a set of files being
synchronized; it is specified relative to the root of the replica.

Formally, a path is just a sequence of names, separated by \verb|/|.
Note that the path separator character is always a forward slash, no
matter what operating system Unison is running on.  Forward slashes
are converted to backslashes as necessary when paths are converted to
filenames in the local filesystem on a particular host.
%
(For example, suppose that we run Unison on a Windows system, synchronizing
the local root \verb|c:\pierce| with the root
\verb|ssh://saul.cis.upenn.edu/home/bcpierce| on a Unix server.  Then
the path \verb|current/todo.txt| refers to the file
\verb|c:\pierce\current\todo.txt| on the client and
\verb|/home/bcpierce/current/todo.txt| on the server.)

The empty path (i.e., the empty sequence of names) denotes the whole
replica.  Unison displays the empty path as ``\verb|[root]|.''

If \verb|p| is a path and \verb|q| is a path beginning with \verb|p|, then
\verb|q| is said to be a {\em descendant} of \verb|p|.  (Each path is also a
descendant of itself.)


\SUBSECTION{What is an Update?}{updates}

The {\em contents} of a path \verb|p| in a particular replica could be a
file, a directory, a symbolic link, or absent (if \verb|p| does not
refer to anything at all in that replica).  More specifically:
\begin{itemize}
\item If \verb|p| refers to an ordinary file, then the
contents of \verb|p| are the actual contents of this file (a string of bytes)
plus the current permission bits of the file.  
\item If \verb|p| refers to a symbolic link, then the contents of \verb|p|
are just the string specifying where the link points.
\item If \verb|p| refers to a directory, then the
contents of \verb|p| are just the token ``DIRECTORY'' plus the current
permission bits of the directory.  
\item If \verb|p| does not refer to anything in this replica, then the
contents of \verb|p| are the token ``ABSENT.''
\end{itemize}
Unison keeps a record of the contents of each path after each
successful synchronization of that path (i.e., it remembers the
contents at the last moment when they were the same in the two
replicas).  

We say that a path is {\em updated} (in some replica) if its current
contents are different from its contents the last time it was successfully
synchronized.  Note that whether a path is updated has nothing to do with
its last modification time---Unison considers only the contents when
determining whether an update has occurred.  This means that touching a file
without changing its contents will {\em not} be recognized as an update.  A
file can even be changed several times and then changed back to its original
contents; as long as Unison is only run at the end of this process, no
update will be recognized.

What Unison actually calculates is a close approximation to this
definition; see \sectionref{caveats}{Caveats and Shortcomings}.

\SUBSECTION{What is a Conflict?}{conflicts}

A path is said to be {\em conflicting} if the following conditions all hold:
\begin{enumerate}
\item it has been updated in one replica, 
\item it or any of its descendants has been updated in the other
  replica, 
and
\item its contents in the two replicas are not identical.
\end{enumerate}

\finishlater{Note that this isn't precisely what we implement, in the
  case of directory permission changes!}


\SUBSECTION{Reconciliation}{recon}

Unison operates in several distinct stages:
\begin{enumerate}
\item On each host, it compares its archive file (which records
the state of each path in the replica when it was last synchronized)
with the current contents of the replica, to determine which paths
have been updated.
\item It checks for ``false conflicts'' --- paths that have been
updated on both replicas, but whose current values are identical.
These paths are silently marked as synchronized in the archive files
in both replicas.
\item It displays all the updated paths to the user.  For updates that
do not conflict, it suggests a default action (propagating the new
contents from the updated replica to the other).  Conflicting updates
are just displayed.  The user is given an opportunity to examine the
current state of affairs, change the default actions for
nonconflicting updates, and choose actions for conflicting updates.
\item It performs the selected actions, one at a time.  Each action is
performed by first transferring the new contents to a temporary file
on the receiving host, then atomically moving them into place.
\item It updates its archive files to reflect the new state of the
replicas. 
\end{enumerate}

\TOPSUBSECTION{Invariants}{failures}

Given the importance and delicacy of the job that it performs, it is
important to understand both what a synchronizer does under normal
conditions and what can happen under unusual conditions such as system
crashes and communication failures.  

% Unison deals with two sorts of information: the two replicas
% themselves and its own memory of the ``last synchronized state'' of
% each path in the replicas.  The latter is what allows it to detect
% correctly which replica is new when a file been updated.  Roughly,
% the sequence of actions that occur when Unison runs is:
% \begin{enumerate}
% \item It reads a private archive file stored with each replica
% and checks which paths on each replica have been updated.
% Technically, a path has been updated if its contents in a replica are
% different from the contents of that replica at the end of the last
% synchronization in which that path was successfully synchronized ---
% i.e., the last time the two replicas were equal at that path at the
% end of a run of Unison.  The ``contents'' of a path can be either a
% file, a directory, or nothing at all, so deleting a file or changing a
% directory to a file count as updates to the contents at that path.

% For efficiency, Unison does not try to calculate the set of updated
% paths exactly: it will sometimes falsely detect a change in a path
% whose contents have actually not changed (this can happen, for
% example, when the file's modification time has been changed, for some
% reason).  As long as this path has not been modified in the other
% replica, this ``conservativity'' in update detection is invisible to
% the user.  If the other replica {\em has} been modified, however, a
% ``false conflict'' may be reported.

% \item It combines the lists of paths that (may) have been updated in
% the two replicas, assigns default actions to those where the change
% was in one replica only, and records a conflict for those that were
% changed in both replicas.

% \item The current contents of the paths on this list are then
% compared, to see if they actually differ.  (This is done by comparing
% fingerprints, not transferring the whole files.)  Paths whose contents
% are actually identical are marked as synchronized and deleted from the
% list. 

% \item The remaining paths are displayed to the user, who then has an
% opportunity to change the default actions and choose actions for
% conflicting paths.

% \item When this process is finished, the selected changes are actually
% propagated between the replicas.

% \item Finally, Unison updates its internal state, marking as
% synchronized all the files for which changes were successfully
% propagated. 
% \end{enumerate}

Unison is careful to protect both its internal state and the state of
the replicas at every point in this process.  Specifically, the
following guarantees are enforced:
\begin{itemize}
\item At every moment, each path in each replica has either (1) its {\em
  original} contents (i.e., no change at all has been made to this
path), or (2) its {\em correct} final contents (i.e., the value that the
user expected to be propagated from the other replica).
\item At every moment, the information stored on disk about Unison's
private state can be either (1) unchanged, or (2) updated to reflect
those paths that have been successfully synchronized.
\end{itemize}
The upshot is that it is safe to interrupt Unison at any time, either
manually or accidentally.  [Caveat: the above is {\em almost} true there
are occasionally brief periods where it is not (and, because of
shortcoming of the Posix filesystem API, cannot be); in particular, when
it is copying a file onto a directory or vice versa, it must first move
the original contents out of the way.  If Unison gets
interrupted during one of these periods, some manual cleanup may be
required.  In this case, a file called {\tt DANGER.README} will be left
in your home directory, containing information about the operation that
was interrupted. The next time you try to run Unison, it will notice this
file and warn you about it.]

If an interruption happens while it is propagating updates, then there
may be some paths for which an update has been propagated but which
have not been marked as synchronized in Unison's archives.  This is no
problem: the next time Unison runs, it will detect changes to these
paths in both replicas, notice that the contents are now equal, and
mark the paths as successfully updated when it writes back its private
state at the end of this run.

If Unison is interrupted, it may sometimes leave temporary working files
(with suffix \verb|.tmp|) in the replicas.  It is safe to delete these
files.  Also, if the \verb|backups| flag is set, Unison will
leave around old versions of files that it overwrites, with names like
\verb|file.0.unison.bak|.  These can be deleted safely when they are no
longer wanted.

Unison is not bothered by clock skew between the different hosts on
which it is running.  It only performs comparisons between timestamps
obtained from the same host, and the only assumption it makes about
them is that the clock on each system always runs forward.

If Unison finds that its archive files have been deleted (or that the
archive format has changed and they cannot be read, or that they don't
exist because this is the first run of Unison on these particular
roots), it takes a conservative approach: it behaves as though the
replicas had both been completely empty at the point of the last
synchronization.  The effect of this is that, on the first run, files
that exist in only one replica will be propagated to the other, while
files that exist in both replicas but are unequal will be marked as
conflicting. 

Touching a file without changing its contents should never affect whether or
not Unison does an update. (When running with the fastcheck preference set
to true---the default on Unix systems---Unison uses file modtimes for a
quick first pass to tell which files have definitely not changed; then, for
each file that might have changed, it computes a fingerprint of the file's
contents and compares it against the last-synchronized contents. Also, the
\verb|-times| option allows you to synchronize file times, but it does not
cause identical files to be changed; Unison will only modify the file
times.)

It is safe to ``brainwash'' Unison by deleting its archive files
{\em on both replicas}.  The next time it runs, it will assume that
all the files it sees in the replicas are new.  

It is safe to modify files while Unison is working.  If Unison
discovers that it has propagated an out-of-date change, or that the
file it is updating has changed on the target replica, it will signal
a failure for that file.  Run Unison again to propagate the latest
change.
\finishlater{There are some race conditions. We should probably talk about them.}

Changes to the ignore patterns from the user interface (e.g., using
the `i' key) are immediately reflected in the current profile.


\SUBSECTION{Caveats and Shortcomings}{caveats}

Here are some things to be careful of when using Unison.  

\begin{itemize}
\item In the interests of speed, the update detection algorithm may
  (depending on which OS architecture that you run Unison on)
  actually use an approximation to the definition given in
  \sectionref{updates}{What is an Update?}.  

  In particular, the Unix
  implementation does not compare the actual contents of files to their
  previous contents, but simply looks at each file's inode number and
  modtime; if neither of these have changed, then it concludes that the
  file has not been changed.

  Under normal circumstances, this approximation is safe, in the sense
  that it may sometimes detect ``false updates'' will never miss a real
  one.  However, it is possible to fool it, for example by using
  \verb|retouch| to change a file's modtime back to a time in the past.
  \finishlater{One user---Marcus Mottl---claimed that it could also
  happen if we use 
  memory mapped I/O, but this is not clear}

\item If you synchronize between a single-user filesystem and a shared
Unix server, you should pay attention to your permission bits: by
default, Unison will synchronize permissions verbatim, which may leave
group-writable files on the server that could be written over by a lot of
people.  

You can control this by setting your \verb|umask| on both computers to
something like 022, masking out the ``world write'' and ``group write''
permission bits.  

Unison does not synchronize the \verb|setuid| and \verb|setgid| bits, for
security. 

\item The graphical user interface is single-threaded.  This
means that if Unison is performing some long-running operation, the
display will not be repainted until it finishes.  We recommend not
trying to do anything with the user interface while Unison is in the
middle of detecting changes or propagating files.

\item Unison does not understand hard links.

\item It is important to be a little careful when renaming directories
containing ``ignore''d files. 

For example, suppose Unison is synchronizing directory A between the two
machines called the ``local'' and the ``remote'' machine; suppose directory
A contains a subdirectory D; and suppose D on the local machine contains a
file or subdirectory P that matches an ignore directive in the profile used
to synchronize. Thus path A/D/P exists on the local machine but not on the
remote machine.
                                                                                
 If D is renamed to D' on the remote machine, and this change is                
 propagated to the local machine, all such files or subdirectories P            
 will be deleted.  This is because Unison sees the rename as a delete and a
 separate create: it deletes the old directory (including the ignored files)
 and creates a new one ({\em not} including the ignored files, since they
 are completely invisible to it).
\end{itemize}



\SECTION{Reference Guide}{reference}{ }

This section covers the features of Unison in detail.  

\TOPSUBSECTION{Running Unison}{running}

There are several ways to start Unison.
\begin{itemize}
\item Typing ``{\tt unison \NT{profile}}'' on the command line.  Unison
will look for a file \texttt{\NT{profile}.prf} in the \verb|.unison|
directory.  If this file does not specify a pair of roots, Unison will
prompt for them and add them to the information specified by the profile.
\item Typing ``{\tt unison \NT{profile} \NT{root1} \NT{root2}}'' on the command
line.
In this case, Unison will use {\tt \NT{profile}}, which should not contain
any {\tt root} directives.
\item Typing ``{\tt unison \NT{root1} \NT{root2}}'' on the command line.  This
has the same effect as typing ``{\tt unison default \NT{root1} \NT{root2}}.''
\item Typing just ``{\tt unison}'' (or invoking Unison by clicking on
a desktop icon).  In this case, Unison will ask for the profile to use
for synchronization (or create a new one, if necessary).   
\end{itemize}

% \finish{Need to check that the text UI actually works this way.  (It 
%   doesn't prompt, for sure, but it should.)}

\SUBSECTION{The {\tt .unison} Directory}{unisondir}

Unison stores a variety of information in a private directory on each
host.  If the environment variable {\tt UNISON} is defined, then its
value will be used as the name of this directory.  If {\tt UNISON} is
not defined, then the name of the directory depends on which
operating system you are using.  In Unix, the default is to use
{\tt \$HOME/.unison}.
In Windows, if the environment variable
{\tt USERPROFILE} is defined, then the directory will be
{\tt \$USERPROFILE$\backslash$.unison};
otherwise if {\tt HOME} is defined, it will be
{\tt \$HOME$\backslash$.unison};
otherwise, it will be
{\tt c:$\backslash$.unison}.

The archive file for each replica is found in the {\tt .unison}
directory on that replica's host.  Profiles (described below) are
always taken from the {\tt .unison} directory on the client host.

Note that Unison maintains a completely different set of archive files
for each pair of roots.

We do not recommend synchronizing the whole {\tt .unison} directory, as this
will involve frequent propagation of large archive files.  It should be safe
to do it, though, if you really want to.  Synchronizing just the profile
files in the {\tt .unison} directory is definitely OK.


\SUBSECTION{Archive Files}{archives}

The name of the archive file on each replica is calculated from 
\begin{itemize}
\item the {\em canonical names} of all the hosts (short names like
  \verb|saul| are converted into full addresses like \verb|saul.cis.upenn.edu|), 
\item the paths to the replicas on all the hosts (again, relative
  pathnames, symbolic links, etc.\ are converted into full, absolute paths), and 
\item an internal version number that is changed whenever a new Unison
  release changes the format of the information stored in the archive.
\end{itemize}
This method should work well for most users.  However, it is occasionally
useful to change the way archive names are generated.  Unison provides
two ways of doing this.

The function that finds the canonical hostname of the local host (which
is used, for example, in calculating the name of the archive file used to
remember which files have been synchronized) normally uses the
\verb|gethostname| operating system call.  However, if the environment
variable \verb|UNISONLOCALHOSTNAME| is set, its value will be used
instead.  This makes it easier to use Unison in situations where a
machine's name changes frequently (e.g., because it is a laptop and gets
moved around a lot).

A more powerful way of changing archive names is provided by the
\verb|rootalias| preference.  The preference file may contain any number of
lines of the form: 
\begin{alltt}
    rootalias = //\NT{hostnameA}//\NT{path-to-replicaA} -> //\NT{hostnameB}//\NT{path-to-replicaB}
\end{alltt}
When calculating the name of the archive files for a given pair of roots,
Unison replaces any root that matches the left-hand side of any rootalias
rule by the corresponding right-hand side.

So, if you need to relocate a root on one of the hosts, you can add a
rule of the form:
\begin{alltt}
    rootalias = //\NT{new-hostname}//\NT{new-path} -> //\NT{old-hostname}//\NT{old-path}
\end{alltt}

{\em Warning}: The \verb|rootalias| option is dangerous and should only
be used if you are sure you know what you're doing.  In particular, it
should only be used if you are positive that either (1) both the original
root and the new alias refer to the same set of files, or (2) the files
have been relocated so that the original name is now invalid and will
never be used again.  (If the original root and the alias refer to
different sets of files, Unison's update detector could get confused.)
%
After introducing a new \verb|rootalias|, it is a good idea to run Unison
a few times interactively (with the \verb|batch| flag off, etc.) and
carefully check that things look reasonable---in particular, that update
detection is working as expected.


\SUBSECTION{Preferences}{prefs}

Many details of Unison's behavior are configurable by user-settable
``preferences.''  

Some preferences are boolean-valued; these are often called {\em flags}.
Others take numeric or string arguments, indicated in the preferences
list by {\tt n} or {\tt xxx}.  Most of the string preferences can be
given several times; the arguments are accumulated into a list
internally.

There are two ways to set the values of preferences: temporarily, by
providing command-line arguments to a particular run of Unison, or
permanently, by adding commands to a {\em profile} in the {\tt .unison}
directory on the client host.  The order of preferences (either on the
command line or in preference files) is not significant.  On the command
line, preferences and other arguments (the profile name and roots) can be
intermixed in any order.

To set the value of a preference {\tt p} from the command line, add an
argument {\tt -p} (for a boolean flag) or {\tt -p n} or {\tt -p xxx} (for
a numeric or string preference) anywhere on the command line.  To set a
boolean flag to \verb|false| on the command line, use {\tt -p=false}.

Here are all the preferences supported by Unison.  This list can be
  obtained by typing {\tt unison -help}.
\begin{quote}
\verbatiminput{prefs.tmp} 
\end{quote}
Here, in more detail, are what they do.  Many are discussed in even greater
detail in other sections of the manual. 
%
\input{prefsdocs.tmp} 


\SUBSECTION{Profiles}{profile}

A {\em profile} is a text file that specifies permanent settings for
roots, paths, ignore patterns, and other preferences, so that they do
not need to be typed at the command line every time Unison is run.
Profiles should reside in the \verb|.unison| directory on the client
machine.  If Unison is started with just one argument \ARG{name} on
the command line, it looks for a profile called \texttt{\ARG{name}.prf} in
the \verb|.unison| directory.  If it is started with no arguments, it
scans the \verb|.unison| directory for files whose names end in
\verb|.prf| and offers a menu (provided that the Unison executable is compiled with the graphical user interface).  If a file named \verb|default.prf| is
found, its settings will be offered as the default choices.

To set the value of a preference {\tt p} permanently, add to the
appropriate profile a line of the form
\begin{verbatim}
        p = true
\end{verbatim}
for a boolean flag or
\begin{verbatim}
        p = <value>
\end{verbatim}
for a preference of any other type.  

Whitespaces around {\tt p} and {\tt xxx} are ignored.
A profile may also include blank lines and lines beginning
with {\tt \#}; both are ignored.

When Unison starts, it first reads the profile and then the command
line, so command-line options will override settings from the
profile.  

Profiles may also include lines of the form \texttt{include
  \ARG{name}}, which will cause the file \ARG{name} (or
\texttt{\ARG{name}.prf}, if \ARG{name} does not exist in the
\verb+.unison+ directory) to be read at the point, and included as if
its contents, instead of the \texttt{include} line, was part of the
profile.  Include lines allows settings common to several profiles to
be stored in one place.

A profile may include a preference `\texttt{label = \ARG{desc}}' to
provide a description of the options selected in this profile.  The
string \ARG{desc} is listed along with the profile name in the profile
selection dialog, and displayed in the top-right corner of the main
Unison window in the graphical user interface.

The graphical user-interface also supports one-key shortcuts for commonly
used profiles.  If a profile contains a preference of the form 
%
`\texttt{key = \ARG{n}}', where \ARG{n} is a single digit, then
pressing this digit key will cause Unison to immediately switch to
this profile and begin synchronization again from scratch.  In this
case, all actions that have been selected for a set of changes
currently being displayed will be discarded.


\SUBSECTION{Sample Profiles}{profileegs}

\SUBSUBSECTION{A Minimal Profile}{minimalprofile}

Here is a very minimal profile file, such as might be found in {\tt
  .unison/default.prf}:
\begin{verbatim}
    # Roots of the synchronization
    root = /home/bcpierce
    root = ssh://saul//home/bcpierce

    # Paths to synchronize 
    path = current
    path = common
    path = .netscape/bookmarks.html
\end{verbatim}

\SUBSUBSECTION{A Basic Profile}{basicprofile}

Here is a more sophisticated profile, illustrating some other useful
features. 
\begin{verbatim}
    # Roots of the synchronization
    root = /home/bcpierce
    root = ssh://saul//home/bcpierce

    # Paths to synchronize 
    path = current
    path = common
    path = .netscape/bookmarks.html

    # Some regexps specifying names and paths to ignore
    ignore = Name temp.*
    ignore = Name *~
    ignore = Name .*~
    ignore = Path */pilot/backup/Archive_*
    ignore = Name *.o
    ignore = Name *.tmp

    # Window height
    height = 37

    # Keep a backup copy of every file in a central location
    backuplocation = central
    backupdir = /home/bcpierce/backups
    backup = Name *
    backupprefix = $VERSION.
    backupsuffix = 

    # Use this command for displaying diffs
    diff = diff -y -W 79 --suppress-common-lines

    # Log actions to the terminal
    log = true
\end{verbatim}

\SUBSUBSECTION{A Power-User Profile}{powerprofile}

When Unison is used with large replicas, it is often convenient to be
able to synchronize just a part of the replicas on a given run (this
saves the time of detecting updates in the other parts).  This can be
accomplished by splitting up the profile into several parts --- a common
part containing most of the preference settings, plus one ``top-level''
file for each set of paths that need to be synchronized.  (The {\tt
  include} mechanism can also be used to allow the same set of preference
settings to be used with different roots.)

The collection
of profiles implementing this scheme might look as follows.
%
The file {\tt default.prf} is empty except for an {\tt include}
directive:
\begin{verbatim}
    # Include the contents of the file common
    include common
\end{verbatim}
Note that the name of the common file is {\tt common}, not {\tt
  common.prf}; this prevents Unison from offering {\tt common} as one of
the list of profiles in the opening dialog (in the graphical UI).

The file {\tt common} contains the real preferences:
\begin{verbatim}
    # Roots of the synchronization
    root = /home/bcpierce
    root = ssh://saul//home/bcpierce

    # (... other preferences ...)

    # If any new preferences are added by Unison (e.g. 'ignore'
    # preferences added via the graphical UI), then store them in the
    # file 'common' rathen than in the top-level preference file
    addprefsto = common

    # regexps specifying names and paths to ignore
    ignore = Name temp.*
    ignore = Name *~
    ignore = Name .*~
    ignore = Path */pilot/backup/Archive_*
    ignore = Name *.o
    ignore = Name *.tmp
\end{verbatim}
Note that there are no {\tt path} preferences in {\tt common}.  This
means that, when we invoke Unison with the default profile (e.g., by
typing '{\tt unison default}' or just '{\tt unison}' on the command
line), the whole replicas will be synchronized.  (If we {\em never} want
to synchronize the whole replicas, then {\tt default.prf} would instead
include settings for all the paths that are usually synchronized.)

To synchronize just part of the replicas, Unison is invoked with an
alternate preference file---e.g., doing '{\tt unison workingset}', where the
preference file {\tt workingset.prf} contains
\begin{verbatim}
    path = current/papers
    path = Mail/inbox
    path = Mail/drafts
    include common
\end{verbatim}
causes Unison to synchronize just the subdirectories {\tt current/papers}
and {\tt older/papers}.

The {\tt key} preference can be used in combination with the graphical UI
to quickly switch between different sets of paths.  For example, if the
file {\tt mail.prf} contains
\begin{verbatim}
    path = Mail
    batch = true
    key = 2
    include common
\end{verbatim}
then pressing 2 will cause Unison to look for updates in the {\tt Mail}
subdirectory and (because the {\tt batch} flag is set) immediately
propagate any that it finds.


\SUBSECTION{Keeping Backups}{backups}

When Unison overwrites a file or directory by propagating a new version from
the other replica, it can keep the old version around as a backup.  There
are several preferences that control precisely where these backups are
stored and how they are named.

To enable backups, you must give one or more \verb|backup| preferences.
Each of these has the form
\begin{verbatim}
    backup = <pathspec>
\end{verbatim}
where \verb|<pathspec>| has the same form as for the \verb|ignore|
preference.  For example, 
\begin{verbatim}
    backup = Name *
\end{verbatim}
causes Unison to keep backups of {\em all} files and directories.  The
\verb|backupnot| preference can be used to give a few exceptions: it
specifies which files and directories should {\em not} be backed up, even if
they match the \verb|backup| pathspec. 

It is important to note that the \verb|pathspec| is matched against the path
that is being updated by Unison, not its descendants.  For example, if you
set \verb|backup = Name *.txt| and then delete a whole directory named
\verb|foo| containing some text files, these files will not be backed up
because Unison will just check that \verb|foo| does not match \verb|*.txt|.
Similarly, if the directory itself happened to be called \verb|foo.txt|,
then the whole directory and all the files in it will be backed up,
regardless of their names. 

Backup files can be stored either {\em centrally} or {\em locally}.  This
behavior is controlled by the preference \verb|backuplocation|, whose value
must be either \verb|central| or \verb|local|.  (The default is
\verb|central|.)  

When backups are stored locally, they are kept in the same
directory as the original.

When backups are stored centrally, the directory used to hold them is
controlled by the preference \verb|backupdir| and the
environment variable \verb|UNISONBACKUPDIR|.  (The environment variable is
checked first.)  If neither of these are set, then the directory
\verb|.unison/backup| in the user's home directory is used.

The preference \verb|backupversions| controls how many previous versions of
each file are kept.  The default is 2.

By default, backup files are named \verb|.unison.FILENAME.VERSION.bak|,
where \verb|FILENAME| is the original filename and \verb|version| is the
backup number (001 for the most recent, 002 for the next most recent,
etc.).  This can be changed by setting the preferences \verb|backupprefix|
and/or \verb|backupsuffix|.  If desired, \verb|backupprefix| may include a
directory prefix; this can be used with \verb|backuplocation = local| to put all
backup files for each directory into a single subdirectory.  For example, setting
\begin{verbatim}
    backuplocation = local
    backupprefix = .unison/$VERSION.
    backupsuffix = 
\end{verbatim}
will put all backups in a local subdirectory named \verb|.unison|.  Also,
note that the string \verb|$VERSION| in either \verb|backupprefix| or
\verb|backupsuffix| (it must appear in one or the other) is replaced by
the version number.  This can be used, for example, to ensure that backup
files retain the same extension as the originals.

For backward compatibility, the \verb|backups| preference is also supported.
%
It simply means \verb|backup = Name *| and \verb|backuplocation = local|.


\SUBSECTION{Merging Conflicting Versions}{merge}

Unison can invoke external programs to merge conflicting versions of a file.
The preference \verb|merge| controls this process.  

The \verb|merge| preference may be given once or several times in a
preference file (it can also be given on the command line, of course, but
this tends to be awkward because of the spaces and special characters
involved).  Each instance of the preference looks like this:
\begin{verbatim}
    merge = <PATHSPEC> -> <MERGECMD>
\end{verbatim}
The \verb|<PATHSPEC>| here has exactly the same format as for the
\verb|ignore| preference (see \sectionref{pathspec}{Path specification}).  For example,
using ``\verb|Name *.txt|'' as the \verb|<PATHSPEC>| tells Unison that this
command should be used whenever a file with extension \verb|.txt| needs to
be merged.  

Many external merging programs require as inputs not just the two files that
need to be merged, but also a file containing the {\em last synchronized
  version}.  You can ask Unison to keep a copy of the last synchronized
version for some files using the \verb|backupcurrent| preference. This
preference is used in exactly the same way as \verb|backup| and its meaning
is similar, except that it causes backups to be kept of the {\em current}
contents of each file after it has been synchronized by Unison, rather than
the {\em previous} contents that Unison overwrote.  These backups are kept
on {\em both} replicas in the same place as ordinary backup files---i.e.
according to the \verb|backuplocation| and \verb|backupdir| preferences.
They are named like the original files if \verb|backupslocation| is set to
'central' and otherwise, Unison uses the \verb|backupprefix| and
\verb|backupsuffix| preferences and assumes a version number 000 for these
backups.

The \verb|<MERGECMD>| part of the preference specifies what external command
should be invoked to merge files at paths matching the \verb|<PATHSPEC>|.
Within this string, several special substrings are recognized; these will be
substituted with appropriate values before invoking a sub-shell to execute
the command.  
\begin{itemize}
\item \relax\verb|CURRENT1| is replaced by the name of (a temporary copy of)
  the local variant of the file.
\item \relax\verb|CURRENT2| is replaced by the name of a temporary
  file, into which the contents of the remote variant of the file have
  been transferred by Unison prior to performing the merge.
\item \relax\verb|CURRENTARCH| is replaced by the name of the backed up copy
  of the original version of the file (i.e., the file saved by Unison
  if the current filename matches the path specifications for the
  \verb|backupcurrent| preference, as explained above), if one exists.
  If no archive exists and \relax\verb|CURRENTARCH| appears in the
  merge command, then an error is signalled. 
\item \relax\verb|CURRENTARCHOPT| is replaced by the name of the backed up copy
  of the original version of the file (i.e., its state at the end of
  the last successful run of Unison), if one exists, or the empty
  string if no archive exists.
\item \relax\verb|NEW| is replaced by the name of a temporary file
  that Unison expects to be written by the merge program when it
  finishes, giving the desired new contents of the file.
\item \relax\verb|PATH| is replaced by the path (relative to the roots of
  the replicas) of the file being merged.
\item \relax\verb|NEW1| and \relax\verb|NEW2| are replaced by the names of temporary files
  that Unison expects to be written by the merge program when it
  is only able to partially merge the originals; in this case, \verb|NEW1|
  will be written back to the local replica and \verb|NEW2| to the remote
  replica; \verb|NEWARCH|, if present, will be used as the ``last common
  state'' of the replicas.  (These three options are provided for
  later compatibility with the Harmony data synchronizer.)
\end{itemize}
To accomodate the wide variety of programs that users might want to use for
merging, Unison checks for several possible situations when the merge
program exits:
\begin{itemize}
\item If the merge program exits with a non-zero status, then merge is
  considered to have failed and the replicas are not changed,
  \textit{unless} the command created two files that are actually
  equal (or modified the two input files in equal ways), in which case
  the merge is regarded as having succeeded.
\item If the file \verb|NEW| has been created, it is written back to both
  replicas (and stored in the backup directory).  Similarly, if just the
  file \verb|NEW1| has been created, it is written back to both 
  replicas.
\item If neither \verb|NEW| nor \verb|NEW1| have been created, then Unison
  examines the temporary files \verb|CURRENT1|  and \verb|CURRENT2| that
  were given as inputs to the merge program.  If either has been changed (or
  both have been changed in identical ways), then its new contents are written
  back to both replicas.  If either \verb|CURRENT1| or \verb|CURRENT2| has
  been {\em deleted}, then the contents of the other are written back to
  both replicas.
\item If the files \verb|NEW1|, \verb|NEW2|, and \verb|NEWARCH| have all
  been created, they are written back to the local replica, remote replica,
  and backup directory, respectively. If the files \verb|NEW1|, \verb|NEW2| have 
  been created, but \verb|NEWARCH| has not, then these files are written back to the
  local replica and remote replica, respectively.  Also, if \verb|NEW1| and
  \verb|NEW2| have identical contents, then the same contents are stored as
  a backup (if the \verb|backupcurrent| preference is set for this path) to
  reflect the fact that the path is currently in sync. 
  \item If \verb|NEW1| and \verb|NEW2| (resp. \verb|CURRENT1| and
  \verb|CURRENT2|) are created (resp. overwritten) with different contents
  but the merge command did not fail (i.e., it exited with status code 0),
  then we copy \verb|NEW1| (resp. \verb|CURRENT1|) to the other replica and
  to the archive.  
  
  This behavior is a design choice made to handle the case where a merge
  command only synchronizes some specific contents between two files,
  skipping some irrelevant information (order between entries, for
  instance).  We assume that, if the merge command exits normally, then the
  two resulting files are ``as good as equal.'' (The reason we copy one on
  top of the other is to avoid Unison detecting that the files are unequal
  the next time it is run and trying again to merge them when, in fact, the
  merge program has already made them as similar as it is able to.)
\end{itemize}

A large number of external merging programs are available.  
For example, on Unix systems setting the \verb|merge| preference to
\begin{verbatim}
    merge = Name *.txt -> diff3 CURRENT1 CURRENTARCH CURRENT2 -m > NEW
\end{verbatim}
\noindent
will tell Unison to use the external \verb|diff3| program for merging.  
%
Alternatively, users of \verb|emacs| may find the following settings convenient:
\begin{verbatim}
    merge = Name *.txt -> emacs -q --eval '(ediff-merge-files-with-ancestor 
                             "CURRENT1" "CURRENT2" "CURRENTARCH" nil "NEW")' 
\end{verbatim}
\noindent
(These commands are displayed here on two lines to avoid running off the
edge of the page.  In your preference file, each command should be written on a
single line.) 
Users running Mac OS X (you may need the Developer Tools installed to get
the {\tt opendiff} utility) may prefer
\begin{verbatim}
    merge = Name *.txt -> opendiff CURRENT1 CURRENT2 -ancestor CURRENTARCH -merge NEW
\end{verbatim}
Here is a slightly more involved hack.  The {\tt opendiff} program can
operate either with or without an archive file.  A merge command of this
form 
\begin{verbatim}
    merge = Name *.txt -> 
              if [ CURRENTARCHOPTx = x ]; 
              then opendiff CURRENT1 CURRENT2 -merge NEW; 
              else opendiff CURRENT1 CURRENT2 -ancestor CURRENTARCHOPT -merge NEW; 
              fi
\end{verbatim}
(still all on one line in the preference file!) will test whether an archive
file exists and use the appropriate variant of the arguments to {\tt
  opendiff}. 

Ordinarily, external merge programs are only invoked when Unison is {\em
  not} running in batch mode.  To specify an external merge program that
should be used no matter the setting of the {\tt batch} flag, use the {\tt
  mergebatch} preference instead of {\tt merge}.

\begin{quote}
\it
Please post suggestions for other useful values of the
\verb|merge| preference to the {\tt unison-users} mailing list---we'd like
to give several examples here.
\end{quote}

\finishlater{
\SUBSECTION{Communicating with a Remote Server}{server}

If you can mount both filesystems on the same host, then you can
run with no server (note, though, that this won't be fast enough over
a phone line)..........
}

\SUBSECTION{The User Interface}{ui}

Both the textual and the graphical user interfaces are intended to be
mostly self-explanatory.  Here are just a few tricks:
\begin{itemize}
\item By default, when running on Unix the textual user interface will
try to put the terminal into the ``raw mode'' so that it reads the input a
character at a time rather than a line at a time.  (This means you can
type just the single keystroke ``\verb|>|'' to tell Unison to
propagate a file from left to right, rather than ``\verb|>| Enter.'')

There are some situations, though, where this will not work --- for
example, when Unison is running in a shell window inside Emacs.
Setting the \verb|dumbtty| preference will force Unison to leave the
terminal alone and process input a line at a time.
\end{itemize}

\SUBSECTION{Exit code}{exit}

When running in the textual mode, Unison returns an exit status, which
describes whether, and at which level, the synchronization was successful.
The exit status could be useful when Unison is invoked from a script.
Currently, there are four possible values for the exit status:
\begin{itemize}
\item [0]: successful synchronization; everything is up-to-date now.
\item [1]: some files were skipped, but all file transfers were successful.
\item [2]: non-fatal failures occurred during file transfer.
\item [3]: a fatal error occurred, or the execution was interrupted.
\end{itemize}
The graphical interface does not return any useful information through the
exit status.

\SUBSECTION{Path specification}{pathspec}
Several Unison preferences (e.g., \verb|ignore|/\verb|ignorenot|,
\verb|follow|, \verb|sortfirst|/\verb|sortlast|, \verb|backup|,
\verb|merge|, etc.)
specify individual paths or sets of paths.  These preferences share a
common syntax based on regular-expressions.  Each preference
is associated with a list of path patterns; the paths specified are those
that match any one of the path pattern.

\begin{itemize}
\item Pattern preferences can be given on the command line,
  or, more often, stored in profiles, using the same syntax as other preferences.  
  For example, a profile line of the form
\begin{alltt}
             ignore = \ARG{pattern}
\end{alltt}
adds \ARG{pattern} to the list of patterns to be ignored.

\item Each \ARG{pattern} can have one of three forms.  The most
general form is a Posix extended regular expression introduced by the
keyword \verb|Regex|.  (The collating sequences and character classes of
full Posix regexps are not currently supported).
\begin{alltt}
                 Regex \ARG{regexp}
\end{alltt}
For convenience, two other styles of pattern are also recognized:
\begin{alltt}
                 Name \ARG{name}
\end{alltt}
matches any path in which the last component matches \ARG{name}, while
\begin{alltt}
                 Path \ARG{path}
\end{alltt}
matches exactly the path \ARG{path}.
%
The \ARG{name} and \ARG{path} arguments of the latter forms of
patterns are {\em not} regular expressions.  Instead, 
standard ``globbing'' conventions can be used in \ARG{name} and
\ARG{path}:  
\begin{itemize}
\item a \verb|?| matches any single character except \verb|/|
\item a \verb|*| matches any sequence of characters not including \verb|/|
(and not beginning with \verb|.|, when used at the beginning of a
\ARG{name})
\item \verb|[xyz]| matches any character from the set $\{{\tt x},
  {\tt y}, {\tt z} \}$
\item \verb|{a,bb,ccc}| matches any one of \verb|a|, \verb|bb|, or
  \verb|ccc|. 
\end{itemize}
\item 
The path separator in path patterns is always the
forward-slash character ``/'' --- even when the client or server is
running under Windows, where the normal separator character is a
backslash.  This makes it possible to use the same set of path
patterns for both Unix and Windows file systems.  
\end{itemize}
Some examples of path patterns appear in \sectionref{ignore}{Ignoring
  Paths}.

\SUBSECTION{Ignoring Paths}{ignore}

Most users of Unison will find that their replicas contain lots of
files that they don't ever want to synchronize --- temporary files,
very large files, old stuff, architecture-specific binaries, etc.
They can instruct Unison to ignore these paths using patterns
introduced in \sectionref{pathspec}{Path Patterns}.

For example, the following pattern will make Unison ignore any
path containing the name \verb|CVS| or a name ending in \verb|.cmo|:
\begin{verbatim}
             ignore = Name {CVS,*.cmo}
\end{verbatim}
The next pattern makes Unison ignore the path \verb|a/b|:
\begin{verbatim}
             ignore = Path a/b
\end{verbatim}
Path patterns do {\em not} skip filesnames beginning with \verb|.| (as Name
patterns do).  For example,
\begin{verbatim}
             ignore = Path */tmp
\end{verbatim}
will include \verb|.foo/tmp| in the set of ignore directories, as it is a
path, not a name, that is ignored.

The following pattern makes Unison ignore any path beginning with \verb|a/b|
and ending with a name ending by \verb|.ml|.
\begin{verbatim}
             ignore = Regex a/b/.*\.ml
\end{verbatim}
Note that regular expression patterns are ``anchored'': they must
match the whole path, not just a substring of the path.

Here are a few extra points regarding the \texttt{ignore} preference.
\begin{itemize}
\item If a directory is ignored, all its descendents will be too.
  
\item The user interface provides some convenient commands for adding
  new patterns to be ignored.  To ignore a particular file, select it
  and press ``{\tt i}''.  To ignore all files with the same extension,
  select it and press ``{\tt E}'' (with the shift key).  To ignore all
  files with the same name, no matter what directory they appear in,
  select it and press ``{\tt N}''.
%
These new patterns become permanent: they
are immediately added to the current profile on disk.

\item If you use the \verb|include| directive to include a common
collection of preferences in several top-level preference files, you will
probably also want to set the \verb|addprefsto| preference to the name of
this file.  This will cause any new ignore patterns that you add from
inside Unison to be appended to this file, instead of whichever top-level
preference file you started Unison with.  

\item Ignore patterns can also be specified on the command line, if
you like (this is probably not very useful), using an option like
\verb|-ignore 'Name temp.txt'|.

\item Be careful about renaming directories containing ignored files.
Because Unison understands the rename as a delete plus a create, any ignored
files in the directory will be lost (since they are invisible to Unison and
therefore they do not get recreated in the new version of the directory).
\end{itemize} 

\SUBSECTION{Symbolic Links}{symlinks}

Ordinarily, Unison treats symbolic links in Unix replicas as
``opaque'': it considers the contents of the link to be just the
string specifying where the link points, and it will propagate changes in
this string to the other replica.

It is sometimes useful to treat a symbolic link ``transparently,''
acting as though whatever it points to were physically {\em in} the
replica at the point where the symbolic link appears.  To tell Unison
to treat a link in this manner, add a line of the form
\begin{alltt}
             follow = \ARG{pathspec}
\end{alltt}
to the profile, where \ARG{pathspec} is a path pattern as described in
\sectionref{pathspec}{Path Patterns}.

Windows file systems do not support symbolic links; Unison will refuse
to propagate an opaque symbolic link from Unix to Windows and flag the
path as erroneous.  When a Unix replica is to be synchronized with a
Windows system, all symbolic links should match either an
\verb|ignore| pattern or a \verb|follow| pattern.


\SUBSECTION{Permissions}{perms}

Synchronizing the permission bits of files is slightly tricky when two
different filesytems are involved (e.g., when synchronizing a Windows
client and a Unix server).  In detail, here's how it works:
\begin{itemize}
\item When the permission bits of an existing file or directory are
changed, the values of those bits that make sense on {\em both}
operating systems will be propagated to the other replica.  The other
bits will not be changed.  
\item When a newly created file is propagated to a remote replica, the
permission bits that make sense in both operating systems are also
propagated.  The values of the other bits are set to default values
(they are taken from the current umask, if the receiving host is a
Unix system).
\item For security reasons, the Unix \verb|setuid| and \verb|setgid|
bits are not propagated.  
\item The Unix owner and group ids are not propagated.  (What would
this mean, in general?)  All files are created with the owner and
group of the server process.
\end{itemize}

\finishlater{
\SUBSECTION{Backup Files}{backups}
}
\finish{a lot to say about the backup system}


\SUBSECTION{Cross-Platform Synchronization}{crossplatform}

If you use Unison to synchronize files between Windows and Unix
systems, there are a few special issues to be aware of.

\textbf{Case conflicts.}  In Unix, filenames are case sensitive:
\texttt{foo} and \texttt{FOO} can refer to different files.  In
Windows, on the other hand, filenames are not case sensitive:
\texttt{foo} and \texttt{FOO} can only refer to the same file.  This
means that a Unix \texttt{foo} and \texttt{FOO} cannot be synchronized
onto a Windows system --- Windows won't allow two different files to
have the ``same'' name.  Unison detects this situation for you, and
reports that it cannot synchronize the files.  

You can deal with a case conflict in a couple of ways.  If you need to
have both files on the Windows system, your only choice is to rename
one of the Unix files to avoid the case conflict, and re-synchronize.
If you don't need the files on the Windows system, you can simply
disregard Unison's warning message, and go ahead with the
synchronization; Unison won't touch those files.  If you don't want to
see the warning on each synchronization, you can tell Unison to ignore
the files (see \sectionref{ignore}{Ignore}).

\textbf{Illegal filenames.}  Unix allows some filenames that are
illegal in Windows.  For example, colons (`:') are not allowed in
Windows filenames, but they are legal in Unix filenames.  This means
that a Unix file \texttt{foo:bar} can't be synchronized to a Windows
system.  As with case conflicts, Unison detects this situation for
you, and you have the same options: you can either rename the Unix
file and re-synchronize, or you can ignore it.


\SUBSECTION{Slow Links}{speed}

Unison is built to run well even over relatively slow links such as
modems and DSL connections.  

Unison uses the ``rsync protocol'' designed by Andrew Tridgell and Paul
Mackerras to greatly speed up transfers of large files in which only
small changes have been made.  More information about the rsync protocol
can be found at the rsync web site (\ONEURL{http://samba.anu.edu.au/rsync/}).

If you are using Unison with {\tt ssh}, you may get some speed
improvement by enabling {\tt ssh}'s compression feature.  Do this by
adding the option ``{\tt -rshargs -C}'' to the command line or ``{\tt
  rshargs = -C}'' to your profile.  


\SUBSECTION{Fast Update Detection}{fastcheck}

If your replicas are large and at least one of them is on a Windows
system, you may find that Unison's default method for detecting changes
(which involves scanning the full contents of every file on every
sync---the only completely safe way to do it under Windows) is too slow.
Unison provides a preference {\tt fastcheck} that, when set to
\verb|yes|, causes it to use file creation times as 'pseudo inode
numbers' when scanning replicas for updates, instead of reading the full
contents of every file.  

When \verb|fastcheck| is set to \verb|no|,
Unison will perform slow checking---re-scanning the contents of each file
on each synchronization---on all replicas.  When \verb|fastcheck| is set
to \verb|default| (which, naturally, is the default), Unison will use
fast checks on Unix replicas and slow checks on Windows replicas.

This strategy may cause Unison to miss propagating an update if the
create time, modification time, and length of the file are all unchanged
by the update (this is not easy to achieve, but it can be done).
However, Unison will never {\em overwrite} such an update with a change
from the other replica, since it always does a safe check for updates
just before propagating a change.  Thus, it is reasonable to use this
switch most of the time and occasionally run Unison once with {\tt
  fastcheck} set to \verb|no|, if you are worried that Unison may have
overlooked an update.



\SUBSECTION{Click-starting Unison}{click}

On Windows NT/2k/XP systems, the graphical version of Unison can be
invoked directly by clicking on its icon.  On Windows 95/98 systems,
click-starting also works, {\em as long as you are not using ssh}.
Due to an incompatibility with ocaml and Windows 95/98 that is not
under our control, you must start Unison from a DOS window in Windows
95/98 if you want to use ssh.

When you click on the Unison icon, two windows will be created:
Unison's regular window, plus a console window, which is used only for
giving your password to ssh (if you do not use ssh to connect, you can
ignore this window).  When your password is requested, you'll need to
activate the console window (e.g., by clicking in it) before typing.
If you start Unison from a DOS window, Unison's regular window will
appear and you will type your password in the DOS window you were
using.

To use Unison in this mode, you must first create a profile (see
\sectionref{profile}{Profile}).  Use your favorite editor for this.  


\appendix
\SECTION{Installing Ssh}{ssh}{ssh}

Your local host will need just an ssh client; the remote host needs an
ssh server (or daemon), which is available on Unix systems.  Unison is
known to work with ssh version 1.2.27 (Unix) and version 1.2.14
(Windows); other versions may or may not work.

\SUBSECTION{Unix}{ssh-unix}

Most modern Unix installations come with \verb|ssh| pre-installed.

\SUBSECTION{Windows}{ssh-win}
Many Windows implementations of ssh only provide graphical interfaces,
but Unison requires an ssh client that it can invoke with a
command-line interface.  A suitable version of ssh can be installed as
follows.  ({\em Warning: These instructions may be out of date.}) 

\begin{enumerate}
\item Download an \verb|ssh| executable.  
  
Warning: there are many implementations and ports of ssh for
Windows, and not all of them will work with Unison.  We have gotten
Unison to work with Cygwin's port of openssh, and we suggest you try
that one first.  Here's how to install it:
\begin{enumerate}
\item First, create a new folder on your desktop to hold temporary
  installation files.  It can have any name you like, but in these
  instructions we'll assume that you call it \verb|Foo|.
\item Direct your web browser to www.cygwin.com, and click on the
  ``Install now!'' link.  This will download a file, \verb|setup.exe|;
  save it in the directory \verb|Foo|.  The file \verb|setup.exe| is a
  small program that will download the actual install files from
  the Internet when you run it.
\item Start \verb|setup.exe| (by double-clicking).  This brings up a
  series of dialogs that you will have to go through.  Select
  ``Install from Internet.''  For ``Local Package Directory'' select
  the directory \verb|Foo|.  For ``Select install root directory'' we
  recommend that you use the default, \verb|C:\cygwin|.  The next
  dialog asks you to select the way that you want to connect to the
  network to download the installation files; we have used ``Use IE5
  Settings'' successfully, but you may need to make a different
  selection depending on your networking setup.  The next dialog gives
  a list of mirrors; select one close to you.
  
  Next you are asked to select which packages to install.  The default
  settings in this dialog download a lot of packages that are not
  strictly necessary to run Unison with ssh.  If you don't want to
  install a package, click on it until ``skip'' is shown.  For a
  minimum installation, select only the packages ``cygwin'' and
  ``openssh,'' which come to about 1900KB; the full installation is
  much larger.  

  \begin{quote} \em Note that you are plan to build unison using the free
    CygWin GNU C compiler, you need to install essential development
    packages such as ``gcc'', ``make'', ``fileutil'', etc; we refer to
    the file ``INSTALL.win32-cygwin-gnuc'' in the source distribution
    for further details.
  \end{quote}

  After the packages are downloaded and installed, the next dialog
  allows you to choose whether to ``Create Desktop Icon'' and ``Add to
  Start Menu.''  You make the call.
\item You can now delete the directory \verb|Foo| and its contents.
\end{enumerate}
Some people have reported problems using Cygwin's ssh with Unison.  If
you have trouble, you might try this one instead:
\begin{verbatim}
  http://opensores.thebunker.net/pub/mirrors/ssh/contrib/ssh-1.2.14-win32bin.zip
\end{verbatim}

\item You must set the environment variables HOME and PATH\@.
  Ssh will create a directory \verb|.ssh| in the directory given
  by HOME, so that it has a place to keep data like your public and
  private keys.  PATH must be set to include the Cygwin \verb|bin|
  directory, so that Unison can find the ssh executable.
  \begin{itemize}
  \item 
    On Windows 95/98, add the lines
\begin{verbatim}
    set PATH=%PATH%;<SSHDIR>
    set HOME=<HOMEDIR>
\end{verbatim}
    to the file \verb|C:\AUTOEXEC.BAT|, where \verb|<HOMEDIR>| is the
    directory where you want ssh to create its \verb|.ssh| directory,
    and \verb|<SSHDIR>| is the directory where the executable
    \verb|ssh.exe| is stored; if you've installed Cygwin in the
    default location, this is \verb|C:\cygwin\bin|.  You will have to
    reboot your computer to take the changes into account.
  \item On Windows NT/2k/XP, open the environment variables dialog box:
    \begin{itemize}
    \item Windows NT: My Computer/Properties/Environment
    \item Windows 2k: My Computer/Properties/Advanced/Environment
      variables
    \end{itemize}
    then select Path and edit its value by appending \verb|;<SSHDIR>|
    to it, where \verb|<SSHDIR>| is the full name of the directory 
    that includes the ssh executable; if you've installed Cygwin in
    the default location, this is \verb|C:\cygwin\bin|.
  \end{itemize}
  \item Test ssh from a DOS shell by typing
\begin{verbatim}
      ssh <remote host> -l <login name>
\end{verbatim}
    You should get a prompt for your password on \verb|<remote host>|,
    followed by a working connection.
  \item Note that \verb|ssh-keygen| may not work (fails with
  ``gethostname: no such file or directory'') on some systems.  This is
  OK: you can use ssh with your regular password for the remote
  system. 
\item You should now be able to use Unison with an ssh connection. If
  you are logged in with a different user name on the local and remote
  hosts, provide your remote user name when providing the remote root
  (i.e., \verb|//username@host/path...|).
\end{enumerate}

\SECTION{Changes in Version \unisonversion}{news}{news}

\begin{changesfromversion}{2.13.0}
\item The features for performing backups and for invoking external merge
programs have been completely rewritten by Stephane Lescuyer (thanks,
Stephane!).  The user-visible functionality should not change, but the
internals have been rationalized and there are a number of new features.
See the manual (in particular, the description of the \verb|backupXXX|
preferences) for details.

\item Incorporated patches for ipv6 support, contributed by Samuel Thibault.
(Note that, due to a bug in the released OCaml 3.08.3 compiler, this code
will not actually work with ipv6 unless compiled with the CVS version of the
OCaml compiler, where the bug has been fixed; however, ipv4 should continue
to work normally.)

\item OSX interface:
\begin{itemize}
\item Incorporated Ben Willmore's cool new icon for the Mac UI.
\end{itemize}

\item Small fixes:
\begin{itemize}
\item Fixed off by one error in month numbers (in printed dates) reported 
  by Bob Burger
\end{itemize}

\end{changesfromversion}

\begin{changesfromversion}{2.12.0}
\item New convention for release numbering: Releases will continue to be
given numbers of the form \verb|X.Y.Z|, but, 
from now on, just the major version number (\verb|X.Y|) will be considered
significant when checking compatibility between client and server versions.
The third component of the version number will be used only to identify
``patch levels'' of releases.

This change goes hand in hand with a change to the procedure for making new
releases.  Candidate releases will initially be given ``beta release''
status when they are announced for public consumption.  Any bugs that are
discovered will be fixed in a separate branch of the source repository
(without changing the major version number) and new tarballs re-released as
needed.  When this process converges, the patched beta version will be
dubbed stable.

\item Warning (failure in batch mode) when one path is completely emptied.
  This prevents Unison from deleting everything on one replica when
  the other disappear.

\item Fix diff bug (where no difference is shown the first time the diff
  command is given).

\item User interface changes:
\begin{itemize}
\item Improved workaround for button focus problem (GTK2 UI)
\item Put leading zeroes in date fields
\item More robust handling of character encodings in GTK2 UI
\item Changed format of modification time displays, from \verb|modified at hh:mm:ss on dd MMM, yyyy|
to \verb|modified on yyyy-mm-dd hh:mm:ss|
\item Changed time display to include seconds (so that people on FAT
  filesystems will not be confused when Unison tries to update a file
  time to an odd number of seconds and the filesystem truncates it to
  an even number!)
\item Use the diff "-u" option by default when showing differences between files
  (the output is more readable)
\item In text mode, pipe the diff output to a pager if the environment
  variable PAGER is set
\item Bug fixes and cleanups in ssh password prompting.  Now works with
  the GTK2 UI under Linux.  (Hopefully the Mac OS X one is not broken!)
\item Include profile name in the GTK2 window name
\item Added bindings ',' (same as '<') and '.' (same as '>') in the GTK2 UI
\end{itemize}

\item Mac GUI:
\begin{itemize}
\item actions like < and > scroll to the next item as necessary.
\item Restart has a menu item and keyboard shortcut (command-R).
\item 
    Added a command-line tool for Mac OS X.  It can be installed from
    the Unison menu.
\item New icon.
\item   Handle the "help" command-line argument properly.
\item   Handle profiles given on the command line properly.
\item  When a profile has been selected, the profile dialog is replaced by a
    "connecting" message while the connection is being made.  This
    gives better feedback.
\item   Size of left and right columns is now large enough so that
    "PropsChanged" is not cut off.
\end{itemize}


\item Minor changes:
\begin{itemize}
\item Disable multi-threading when both roots are local
\item Improved error handling code.  In particular, make sure all files
  are closed in case of a transient failure
\item Under Windows, use \verb|$UNISON| for home directory as a last resort
  (it was wrongly moved before \verb|$HOME| and \verb|$USERPROFILE| in
  Unison 2.12.0)
\item Reopen the logfile if its name changes (profile change)
\item Double-check that permissions and modification times have been
  properly set: there are some combination of OS and filesystem on
  which setting them can fail in a silent way.
\item Check for bad Windows filenames for pure Windows synchronization
  also (not just cross architecture synchronization).
  This way, filenames containing backslashes, which are not correctly
  handled by unison, are rejected right away.
\item Attempt to resolve issues with synchronizing modification times
  of read-only files under Windows
\item Ignore chmod failures when deleting files
\item Ignore trailing dots in filenames in case insensitive mode
\item Proper quoting of paths, files and extensions ignored using the UI
\item The strings CURRENT1 and CURRENT2 are now correctly substitued when
  they occur in the diff preference
\item Improvements to syncing resource forks between Macs via a non-Mac system.
\end{itemize}

\end{changesfromversion}

\begin{changesfromversion}{2.10.2}
\item \incompatible{} Archive format has changed.  

\item Source code availability: The Unison sources are now managed using
  Subversion.  One nice side-effect is that anonymous checkout is now
  possible, like this:
\begin{verbatim}
        svn co https://cvs.cis.upenn.edu:3690/svnroot/unison/
\end{verbatim}
We will also continue to export a ``developer tarball'' of the current
(modulo one day) sources in the web export directory.  To receive commit logs
for changes to the sources, subscribe to the \verb|unison-hackers| list
(\ONEURL{http://www.cis.upenn.edu/~bcpierce/unison/lists.html}). 

\item Text user interface:
\begin{itemize}
\item Substantial reworking of the internal logic of the text UI to make it
a bit easier to modify.
\item The {\tt dumbtty} flag in the text UI is automatically set to true if
the client is running on a Unix system and the {\tt EMACS} environment
variable is set to anything other than the empty string.
\end{itemize}

\item Native OS X gui:
\begin{itemize}
\item Added a synchronize menu item with keyboard shortcut
\item Added a merge menu item, still needs to be debugged
\item Fixes to compile for Panther
\item Miscellaneous improvements and bugfixes
\end{itemize}

\item Small changes:
\begin{itemize}
\item Changed the filename checking code to apply to Windows only, instead
  of OS X as well.
\item Finder flags now synchronized
\item Fallback in copy.ml for filesystem that do not support \verb|O_EXCL|
\item  Changed buffer size for local file copy (was highly inefficient with
  synchronous writes)
\item Ignore chmod failure when deleting a directory
\item  Fixed assertion failure when resolving a conflict content change /
  permission changes in favor of the content change.
\item Workaround for transferring large files using rsync.
\item Use buffered I/O for files (this is the only way to open files in binary
  mode under Cygwin).
\item On non-Cygwin Windows systems, the UNISON environment variable is now checked first to determine 
  where to look for Unison's archive and preference files, followed by \verb|HOME| and 
  \verb|USERPROFILE| in that order.  On Unix and Cygwin systems, \verb|HOME| is used.
\item Generalized \verb|diff| preference so that it can be given either as just 
  the command name to be used for calculating diffs or else a whole command
  line, containing the strings \verb|CURRENT1| and \verb|CURRENT2|, which will be replaced
  by the names of the files to be diff'ed before the command is called.
\item Recognize password prompts in some newer versions of ssh.
\end{itemize}
\end{changesfromversion}

\begin{changesfromversion}{2.9.20}
\item \incompatible{} Archive format has changed.  
\item Major functionality changes:
\begin{itemize}
\item Major tidying and enhancement of 'merge' functionality.  The main
  user-visible change is that the external merge program may either write
  the merged output to a single new file, as before, or it may modify one or
  both of its input files, or it may write {\em two} new files.  In the
  latter cases, its modifications will be copied back into place on both the
  local and the remote host, and (if the two files are now equal) the
  archive will be updated appropriately.  More information can be found in
  the user manual.  Thanks to Malo Denielou and Alan Schmitt for these
  improvements.

  Warning: the new merging functionality is not completely compatible with
  old versions!  Check the manual for details.
  
\item Files larger than 2Gb are now supported.

\item Added preliminary (and still somewhat experimental) support for the
  Apple OS X operating system.   
\begin{itemize}
\item Resource forks should be transferred correctly.  (See the manual for
details of how this works when synchronizing HFS with non-HFS volumes.)
Synchronization of file type and creator information is also supported.
\item On OSX systems, the name of the directory for storing Unison's
archives, preference files, etc., is now determined as follows:
\begin{itemize}
    \item if \verb+~/.unison+ exists, use it
     \item otherwise, use \verb|~/Library/Application Support/Unison|, 
         creating it if necessary.
\end{itemize}
\item A preliminary native-Cocoa user interface is under construction.  This
still needs some work, and some users experience unpredictable crashes, so
it is only for hackers for now.  Run make with {\tt UISTYLE=mac} to build
this interface.
\end{itemize}
\end{itemize}

\item Minor functionality changes:
\begin{itemize}

\item Added an {\tt ignorelocks} preference, which forces Unison to override left-over
  archive locks.  (Setting this preference is dangerous!  Use it only if you
  are positive you know what you are doing.) 
\item Running with the {\tt -timers} flag set to true will now show the total time taken
  to check for updates on each directory.  (This can be helpful for tidying directories to improve
  update detection times.)
\item Added a new preference {\tt assumeContentsAreImmutable}.  If a directory
  matches one of the patterns set in this preference, then update detection
  is skipped for files in this directory.  (The 
  purpose is to speed update detection for cases like Mail folders, which
  contain lots and lots of immutable files.)  Also a preference
  {\tt assumeContentsAreImmutableNot}, which overrides the first, similarly
  to {\tt ignorenot}.  (Later amendment: these preferences are now called
  {\tt immutable} and {\tt immutablenot}.)

\item The {\tt ignorecase} flag has been changed from a boolean to a three-valued
  preference.  The default setting, called {\tt default}, checks the operating systems
  running on the client and server and ignores filename case if either of them is
  OSX or Windows.  Setting ignorecase to {\tt true} or {\tt false} overrides
  this behavior.  If you have been setting {\tt ignorecase} on the command
  line using {\tt -ignorecase=true} or {\tt -ignorecase=false}, you will
  need to change to {\tt -ignorecase true} or {\tt -ignorecase false}.

\item a new preference, 'repeat', for the text user interface (only).  If 'repeat' is set to
  a number, then, after it finishes synchronizing, Unison will wait for that many seconds and
  then start over, continuing this way until it is killed from outside.  Setting repeat to true
  will automatically set the batch preference to true.  
  
\item Excel files are now handled specially, so that the {\tt fastcheck}
  optimization is skipped even if the {\tt fastcheck} flag is set.  (Excel
  does some naughty things with modtimes, making this optimization
  unreliable and leading to failures during change propagation.)

\item The ignorecase flag has been changed from a boolean to a three-valued
  preference.  The default setting, called 'default', checks the operating systems
  running on the client and server and ignores filename case if either of them is
  OSX or Windows.  Setting ignorecase to 'true' or 'false' overrides this behavior.
  
\item Added a new preference, 'repeat', for the text user interface (only,
  at the moment).  If 'repeat' is set to a number, then, after it finishes
  synchronizing, Unison will wait for that many seconds and then start over,
  continuing this way until it is killed from outside.  Setting repeat to
  true will automatically set the batch preference to true.

\item The 'rshargs' preference has been split into 'rshargs' and 'sshargs' 
  (mainly to make the documentation clearer).  In fact, 'rshargs' is no longer
  mentioned in the documentation at all, since pretty much everybody uses
  ssh now anyway.
\end{itemize}

\item Documentation
\begin{itemize}
\item The web pages have been completely redesigned and reorganized.
  (Thanks to Alan Schmitt for help with this.)
\end{itemize}

\item User interface improvements
\begin{itemize}
\item Added a GTK2 user interface, capable (among other things) of displaying filenames
  in any locale encoding.  Kudos to Stephen Tse for contributing this code!  
\item The text UI now prints a list of failed and skipped transfers at the end of
  synchronization. 
\item Restarting update detection from the graphical UI will reload the current
  profile (which in particular will reset the -path preference, in case
  it has been narrowed by using the ``Recheck unsynchronized items''
  command).
\item Several small improvements to the text user interface, including a
  progress display.
\end{itemize}

\item Bug fixes (too numerous to count, actually, but here are some):
\begin{itemize}
\item The {\tt maxthreads} preference works now.
\item Fixed bug where warning message about uname returning an unrecognized
  result was preventing connection to server.  (The warning is no longer
  printed, and all systems where 'uname' returns anything other than 'Darwin' 
  are assumed not to be running OS X.)
\item Fixed a problem on OS X that caused some valid file names (e.g.,
  those including colons) to be considered invalid.
\item Patched Path.followLink to follow links under cygwin in addition to Unix
  (suggested by Matt Swift).
\item Small change to the storeRootsName function, suggested by bliviero at 
  ichips.intel.com, to fix a problem in unison with the `rootalias'
  option, which allows you to tell unison that two roots contain the same 
  files.  Rootalias was being applied after the hosts were 
  sorted, so it wouldn't work properly in all cases.
\item Incorporated a fix by Dmitry Bely for setting utimes of read-only files
  on Win32 systems.   
\end{itemize}

\item Installation / portability:
\begin{itemize}
\item Unison now compiles with OCaml version 3.07 and later out of the box.
\item Makefile.OCaml fixed to compile out of the box under OpenBSD.
\item a few additional ports (e.g. OpenBSD, Zaurus/IPAQ) are now mentioned in 
  the documentation 
\item Unison can now be installed easily on OSX systems using the Fink
  package manager
\end{itemize}
\end{changesfromversion}

\begin{changesfromversion}{2.9.1}
\item Added a preference {\tt maxthreads} that can be used to limit the
number of simultaneous file transfers.
\item Added a {\tt backupdir} preference, which controls where backup
files are stored.
\item Basic support added for OSX.  In particular, Unison now recognizes
when one of the hosts being synchronized is running OSX and switches to
a case-insensitive treatment of filenames (i.e., 'foo' and 'FOO' are
considered to be the same file).
  (OSX is not yet fully working,
  however: in particular, files with resource forks will not be
  synchronized correctly.)
\item The same hash used to form the archive name is now also added to
the names of the temp files created during file transfer.  The reason for
this is that, during update detection, we are going to silently delete
any old temp files that we find along the way, and we want to prevent
ourselves from deleting temp files belonging to other instances of Unison
that may be running in parallel, e.g. synchronizing with a different
host.  Thanks to Ruslan Ermilov for this suggestion.
\item Several small user interface improvements
\item Documentation
\begin{itemize}
\item FAQ and bug reporting instructions have been split out as separate
      HTML pages, accessible directly from the unison web page.
\item Additions to FAQ, in particular suggestions about performance
tuning. 
\end{itemize}
\item Makefile
\begin{itemize}
\item Makefile.OCaml now sets UISTYLE=text or UISTYLE=gtk automatically,
  depending on whether it finds lablgtk installed
\item Unison should now compile ``out of the box'' under OSX
\end{itemize}
\end{changesfromversion}

\begin{changesfromversion}{2.8.1}
\item Changing profile works again under Windows
\item File movement optimization: Unison now tries to use local copy instead of
  transfer for moved or copied files.  It is controled by a boolean option
  ``xferbycopying''.
\item Network statistics window (transfer rate, amount of data transferred).
      [NB: not available in Windows-Cygwin version.]
\item symlinks work under the cygwin version (which is dynamically linked).
\item Fixed potential deadlock when synchronizing between Windows and
Unix 
\item Small improvements:
  \begin{itemize} 
  \item If neither the {\\tt USERPROFILE} nor the {\\tt HOME} environment
    variables are set, then Unison will put its temporary commit log
    (called {\\tt DANGER.README}) into the directory named by the 
    {\\tt UNISON} environment variable, if any; otherwise it will use
    {\\tt C:}.
  \item alternative set of values for fastcheck: yes = true; no = false;
  default = auto.
  \item -silent implies -contactquietly
  \end{itemize}
\item Source code:
  \begin{itemize}
  \item Code reorganization and tidying.  (Started breaking up some of the
    basic utility modules so that the non-unison-specific stuff can be
    made available for other projects.)
  \item several Makefile and docs changes (for release);
  \item further comments in ``update.ml'';
  \item connection information is not stored in global variables anymore.
  \end{itemize}
\end{changesfromversion}

\begin{changesfromversion}{2.7.78}
\item Small bugfix to textual user interface under Unix (to avoid leaving
  the terminal in a bad state where it would not echo inputs after Unison
  exited).
\end{changesfromversion}

\begin{changesfromversion}{2.7.39}
\item Improvements to the main web page (stable and beta version docs are
  now both accessible).
\item User manual revised.
\item Added some new preferences:
\begin{itemize}
\item ``sshcmd'' and ``rshcmd'' for specifying paths to ssh and rsh programs.
\item ``contactquietly'' for suppressing the ``contacting server'' message
during Unison startup (under the graphical UI).
\end{itemize}
\item Bug fixes:
\begin{itemize}
\item Fixed small bug in UI that neglected to change the displayed column 
  headers if loading a new profile caused the roots to change.
\item Fixed a bug that would put the text UI into an infinite loop if it
  encountered a conflict when run in batch mode.
\item Added some code to try to fix the display of non-Ascii characters in 
  filenames on Windows systems in the GTK UI.  (This code is currently 
  untested---if you're one of the people that had reported problems with
  display of non-ascii filenames, we'd appreciate knowing if this actually 
  fixes things.)
\item `\verb|-prefer/-force newer|' works properly now.  
        (The bug was reported by Sebastian Urbaniak and Sean Fulton.)
\end{itemize}
\item User interface and Unison behavior:
\begin{itemize}
\item Renamed `Proceed' to `Go' in the graphical UI.
\item Added exit status for the textual user interface.
\item Paths that are not synchronized because of conflicts or errors during 
  update detection are now noted in the log file.
\item \verb|[END]| messages in log now use a briefer format
\item Changed the text UI startup sequence so that
  {\\tt ./unison -ui text} will use the default profile instead of failing.
\item Made some improvements to the error messages.
\item Added some debugging messages to remote.ml.
\end{itemize}
\end{changesfromversion}

\begin{changesfromversion}{2.7.7}
\item Incorporated, once again, a multi-threaded transport sub-system.
  It transfers several files at the same time, thereby making much
  more effective use of available network bandwidth.  Unlike the
  earlier attempt, this time we do not rely on the native thread
  library of OCaml.  Instead, we implement a light-weight,
  non-preemptive multi-thread library in OCaml directly.  This version
  appears stable.  

  Some adjustments to unison are made to accommodate the multi-threaded
  version.  These include, in particular, changes to the
  user interface and logging, for example:
  \begin{itemize}
  \item Two log entries for each transferring task, one for the
    beginning, one for the end.
  \item Suppressed warning messages against removing temp files left
    by a previous unison run, because warning does not work nicely
    under multi-threading.  The temp file names are made less likely
    to coincide with the name of a file created by the user.  They
    take the form \\ \verb|.#<filename>.<serial>.unison.tmp|.
  \end{itemize}
\item Added a new command to the GTK user interface: pressing 'f' causes
  Unison to start a new update detection phase, using as paths {\em just}
  those paths that have been detected as changed and not yet marked as
  successfully completed.  Use this command to quickly restart Unison on
  just the set of paths still needing attention after a previous run.
\item Made the {\tt ignorecase} preference user-visible, and changed the
  initialization code so that it can be manually set to true, even if
  neither host is running Windows.  (This may be useful, e.g., when using 
  Unison running on a Unix system with a FAT volume mounted.)
\item Small improvements and bug fixes:
  \begin{itemize}
  \item Errors in preference files now generate fatal errors rather than
    warnings at startup time.  (I.e., you can't go on from them.)  Also,
    we fixed a bug that was preventing these warnings from appearing in the
    text UI, so some users who have been running (unsuspectingly) with 
    garbage in their prefs files may now get error reports.
  \item Error reporting for preference files now provides file name and
    line number.
  \item More intelligible message in the case of identical change to the same 
    files: ``Nothing to do: replicas have been changed only in identical 
    ways since last sync.''
  \item Files with prefix '.\#' excluded when scanning for preference
    files.
  \item Rsync instructions are send directly instead of first
    marshaled.
  \item Won't try forever to get the fingerprint of a continuously changing file:
    unison will give up after certain number of retries.
  \item Other bug fixes, including the one reported by Peter Selinger
    (\verb|force=older preference| not working).
  \end{itemize}
\item Compilation:
  \begin{itemize}
  \item Upgraded to the new OCaml 3.04 compiler, with the LablGtk
    1.2.3 library (patched version used for compiling under Windows).
  \item Added the option to compile unison on the Windows platform with
    Cygwin GNU C compiler.  This option only supports building
    dynamically linked unison executables.
  \end{itemize}
\end{changesfromversion}

\begin{changesfromversion}{2.7.4}
\item Fixed a silly (but debilitating) bug in the client startup sequence.
\end{changesfromversion}

\begin{changesfromversion}{2.7.1}
\item Added \verb|addprefsto| preference, which (when set) controls which
preference file new preferences (e.g. new ignore patterns) are added to.
\item Bug fix: read the initial connection header one byte at a time, so
that we don't block if the header is shorter than expected.  (This bug
did not affect normal operation --- it just made it hard to tell when you
were trying to use Unison incorrectly with an old version of the server,
since it would hang instead of giving an error message.)
\end{changesfromversion}

\begin{changesfromversion}{2.6.59}
\item Changed \verb|fastcheck| from a boolean to a string preference.  Its 
  legal values are \verb|yes| (for a fast check), \verb|no| (for a safe 
  check), or \verb|default| (for a fast check---which also happens to be 
  safe---when running on Unix and a safe check when on Windows).  The default 
  is \verb|default|.
  \item Several preferences have been renamed for consistency.  All
  preference names are now spelled out in lowercase.  For backward
  compatibility, the old names still work, but they are not mentioned in
  the manual any more.
\item The temp files created by the 'diff' and 'merge' commands are now
   named by {\em pre}pending a new prefix to the file name, rather than
   appending a suffix.  This should avoid confusing diff/merge programs
   that depend on the suffix  to guess the type of the file contents.
\item We now set the keepalive option on the server socket, to make sure
  that the server times out if the communication link is unexpectedly broken. 
\item Bug fixes:
\begin{itemize}
\item When updating small files, Unison now closes the destination file.
\item File permissions are properly updated when the file is behind a
  followed link.
\item Several other small fixes.
\end{itemize}
\end{changesfromversion}


\begin{changesfromversion}{2.6.38}
\item Major Windows performance improvement!  

We've added a preference \verb|fastcheck| that makes Unison look only at
a file's creation time and last-modified time to check whether it has
changed.  This should result in a huge speedup when checking for updates
in large replicas.

  When this switch is set, Unison will use file creation times as 
  'pseudo inode numbers' when scanning Windows replicas for updates, 
  instead of reading the full contents of every file.  This may cause 
  Unison to miss propagating an update if the create time, 
  modification time, and length of the file are all unchanged by 
  the update (this is not easy to achieve, but it can be done).  
  However, Unison will never {\em overwrite} such an update with
  a change from the other replica, since it 
  always does a safe check for updates just before propagating a 
  change.  Thus, it is reasonable to use this switch most of the time 
  and occasionally run Unison once with {\tt fastcheck} set to false, 
  if you are worried that Unison may have overlooked an update.

  Warning: This change is has not yet been thoroughly field-tested.  If you 
  set the \verb|fastcheck| preference, pay careful attention to what
  Unison is doing.

\item New functionality: centralized backups and merging 
\begin{itemize}
\item This version incorporates two pieces of major new functionality,
   implemented by Sylvain Roy during a summer internship at Penn: a
   {\em centralized backup} facility that keeps a full backup of
   (selected files 
   in) each replica, and a {\em merging} feature that allows Unison to
   invoke an external file-merging tool to resolve conflicting changes to
   individual files.
 
\item Centralized backups:
\begin{itemize}
  \item Unison now maintains full backups of the last-synchronized versions
      of (some of) the files in each replica; these function both as
      backups in the usual sense
      and as the ``common version'' when invoking external
      merge programs.
  \item The backed up files are stored in a directory ~/.unison/backup on each
      host.  (The name of this directory can be changed by setting
      the environment variable \verb|UNISONBACKUPDIR|.)
  \item The predicate \verb|backup| controls which files are actually
     backed up:
      giving the preference '\verb|backup = Path *|' causes backing up
      of all files.
  \item Files are added to the backup directory whenever unison updates
      its archive.  This means that
      \begin{itemize}
       \item When unison reconstructs its archive from scratch (e.g., 
           because of an upgrade, or because the archive files have
           been manually deleted), all files will be backed up.
       \item Otherwise, each file will be backed up the first time unison
           propagates an update for it.
      \end{itemize}
  \item The preference \verb|backupversions| controls how many previous
      versions of each file are kept.  The default is 2 (i.e., the last 
      synchronized version plus one backup).
  \item For backward compatibility, the \verb|backups| preference is also
      still supported, but \verb|backup| is now preferred.
  \item It is OK to manually delete files from the backup directory (or to throw
      away the directory itself).  Before unison uses any of these files for 
      anything important, it checks that its fingerprint matches the one 
      that it expects. 
\end{itemize}

\item Merging:
\begin{itemize}
  \item Both user interfaces offer a new 'merge' command, invoked by pressing
      'm' (with a changed file selected).  
  \item The actual merging is performed by an external program.  
      The preferences \verb|merge| and \verb|merge2| control how this
      program is invoked.  If a backup exists for this file (see the
      \verb|backup| preference), then the \verb|merge| preference is used for 
      this purpose; otherwise \verb|merge2| is used.  In both cases, the 
      value of the preference should be a string representing the command 
      that should be passed to a shell to invoke the 
      merge program.  Within this string, the special substrings
      \verb|CURRENT1|, \verb|CURRENT2|, \verb|NEW|,  and \verb|OLD| may appear
      at any point.  Unison will substitute these as follows before invoking
      the command:
        \begin{itemize}
        \item \relax\verb|CURRENT1| is replaced by the name of the local 
        copy of the file;
        \item \relax\verb|CURRENT2| is replaced by the name of a temporary
        file, into which the contents of the remote copy of the file have
        been transferred by Unison prior to performing the merge;
        \item \relax\verb|NEW| is replaced by the name of a temporary
        file that Unison expects to be written by the merge program when
        it finishes, giving the desired new contents of the file; and
        \item \relax\verb|OLD| is replaced by the name of the backed up
        copy of the original version of the file (i.e., its state at the 
        end of the last successful run of Unison), if one exists 
        (applies only to \verb|merge|, not \verb|merge2|).
        \end{itemize}
      For example, on Unix systems setting the \verb|merge| preference to
\begin{verbatim}
   merge = diff3 -m CURRENT1 OLD CURRENT2 > NEW
\end{verbatim}
      will tell Unison to use the external \verb|diff3| program for merging.  

      A large number of external merging programs are available.  For 
      example, \verb|emacs| users may find the following convenient:
\begin{verbatim}
    merge2 = emacs -q --eval '(ediff-merge-files "CURRENT1" "CURRENT2" 
               nil "NEW")' 
    merge = emacs -q --eval '(ediff-merge-files-with-ancestor 
               "CURRENT1" "CURRENT2" "OLD" nil "NEW")' 
\end{verbatim}
(These commands are displayed here on two lines to avoid running off the
edge of the page.  In your preference file, each should be written on a
single line.) 

  \item If the external program exits without leaving any file at the
  path \verb|NEW|, 
      Unison considers the merge to have failed.  If the merge program writes
      a file called \verb|NEW| but exits with a non-zero status code,
      then Unison 
      considers the merge to have succeeded but to have generated conflicts.
      In this case, it attempts to invoke an external editor so that the
      user can resolve the conflicts.  The value of the \verb|editor| 
      preference controls what editor is invoked by Unison.  The default
      is \verb|emacs|.

  \item Please send us suggestions for other useful values of the
       \verb|merge2| and \verb|merge| preferences -- we'd like to give several 
       examples in the manual.
\end{itemize}
\end{itemize}

\item Smaller changes:
\begin{itemize}
\item When one preference file includes another, unison no longer adds the
  suffix '\verb|.prf|' to the included file by default.  If a file with 
  precisely the given name exists in the .unison directory, it will be used; 
  otherwise Unison will 
  add \verb|.prf|, as it did before.  (This change means that included 
  preference files can be named \verb|blah.include| instead of 
  \verb|blah.prf|, so that unison will not offer them in its 'choose 
  a preference file' dialog.)
\item For Linux systems, we now offer both a statically linked and a dynamically
  linked executable.  The static one is larger, but will probably run on more
  systems, since it doesn't depend on the same versions of dynamically
  linked library modules being available.
\item Fixed the \verb|force| and \verb|prefer| preferences, which were
  getting the propagation direction exactly backwards.
\item Fixed a bug in the startup code that would cause unison to crash
  when the default profile (\verb|~/.unison/default.prf|) does not exist.
\item Fixed a bug where, on the run when a profile is first created, 
  Unison would confusingly display the roots in reverse order in the user
  interface.
\end{itemize}

\item For developers:
\begin{itemize}
\item We've added a module dependency diagram to the source distribution, in
   \verb|src/DEPENDENCIES.ps|, to help new prospective developers with
   navigating the code. 
\end{itemize}
\end{changesfromversion}

\begin{changesfromversion}{2.6.11}
\item \incompatible{} Archive format has changed.  

\item \incompatible{} The startup sequence has been completely rewritten
and greatly simplified.  The main user-visible change is that the
\verb|defaultpath| preference has been removed.  Its effect can be
approximated by using multiple profiles, with \verb|include| directives
to incorporate common settings.  All uses of \verb|defaultpath| in
existing profiles should be changed to \verb|path|.

Another change in startup behavior that will affect some users is that it
is no longer possible to specify roots {\em both} in the profile {\em
  and} on the command line.

You can achieve a similar effect, though, by breaking your profile into
two:
\begin{verbatim}
  
  default.prf = 
      root = blah
      root = foo
      include common

  common.prf = 
      <everything else>
\end{verbatim}
Now do
\begin{verbatim}
  unison common root1 root2
\end{verbatim}
when you want to specify roots explicitly.

\item The \verb|-prefer| and \verb|-force| options have been extended to
allow users to specify that files with more recent modtimes should be
propagated, writing either \verb|-prefer newer| or \verb|-force newer|.
(For symmetry, Unison will also accept \verb|-prefer older| or
\verb|-force older|.)  The \verb|-force older/newer| options can only be
used when \verb|-times| is also set.

The graphical user interface provides access to these facilities on a
one-off basis via the \verb|Actions| menu.

\item Names of roots can now be ``aliased'' to allow replicas to be
relocated without changing the name of the archive file where Unison
stores information between runs.  (This feature is for experts only.  See
the ``Archive Files'' section of the manual for more information.)

\item Graphical user-interface:
\begin{itemize}
\item A new command is provided in the Synchronization menu for
  switching to a new profile without restarting Unison from scratch.
\item The GUI also supports one-key shortcuts for commonly
used profiles.  If a profile contains a preference of the form 
%
'\verb|key = n|', where \verb|n| is a single digit, then pressing this
key will cause Unison to immediately switch to this profile and begin
synchronization again from scratch.  (Any actions that may have been
selected for a set of changes currently being displayed will be
discarded.) 

\item Each profile may include a preference '\verb|label = <string>|' giving a
  descriptive string that described the options selected in this profile.
  The string is listed along with the profile name in the profile selection
  dialog, and displayed in the top-right corner of the main Unison window.
\end{itemize}

\item Minor:
\begin{itemize}
\item Fixed a bug that would sometimes cause the 'diff' display to order
  the files backwards relative to the main user interface.  (Thanks
  to Pascal Brisset for this fix.)
\item On Unix systems, the graphical version of Unison will check the
  \verb|DISPLAY| variable and, if it is not set, automatically fall back
  to the textual user interface.
\item Synchronization paths (\verb|path| preferences) are now matched
  against the ignore preferences.  So if a path is both specified in a
  \verb|path| preference and ignored, it will be skipped.
\item Numerous other bugfixes and small improvements.
\end{itemize}
\end{changesfromversion}

\begin{changesfromversion}{2.6.1}
\item The synchronization of modification times has been disabled for
  directories.

\item Preference files may now include lines of the form
  \verb+include <name>+, which will cause \verb+name.prf+ to be read
  at that point.

\item The synchronization of permission between Windows and Unix now
  works properly.

\item A binding \verb|CYGWIN=binmode| in now added to the environment
  so that the Cygwin port of OpenSSH works properly in a non-Cygwin
  context.

\item The \verb|servercmd| and \verb|addversionno| preferences can now
  be used together: \verb|-addversionno| appends an appropriate
  \verb+-NNN+ to the server command, which is found by using the value
  of the \verb|-servercmd| preference if there is one, or else just
  \verb|unison|.

\item Both \verb|'-pref=val'| and \verb|'-pref val'| are now allowed for
  boolean values.  (The former can be used to set a preference to false.)

\item Lot of small bugs fixed.
\end{changesfromversion}

\begin{changesfromversion}{2.5.31}
\item The \verb|log| preference is now set to \verb|true| by default,
  since the log file seems useful for most users.  
\item Several miscellaneous bugfixes (most involving symlinks).
\end{changesfromversion}

\begin{changesfromversion}{2.5.25}
\item \incompatible{} Archive format has changed (again).  

\item Several significant bugs introduced in 2.5.25 have been fixed.  
\end{changesfromversion}

\begin{changesfromversion}{2.5.1}
\item \incompatible{} Archive format has changed.  Make sure you
synchronize your replicas before upgrading, to avoid spurious
conflicts.  The first sync after upgrading will be slow.

\item New functionality:
\begin{itemize}
\item Unison now synchronizes file modtimes, user-ids, and group-ids.  

These new features are controlled by a set of new preferences, all of
which are currently \verb|false| by default.  

\begin{itemize}
\item When the \verb|times| preference is set to \verb|true|, file
modification times are propaged.  (Because the representations of time
may not have the same granularity on both replicas, Unison may not always
be able to make the modtimes precisely equal, but it will get them as
close as the operating systems involved allow.)
\item When the \verb|owner| preference is set to \verb|true|, file
ownership information is synchronized.
\item When the \verb|group| preference is set to \verb|true|, group 
information is synchronized.
\item When the \verb|numericIds| preference is set to \verb|true|, owner
and group information is synchronized numerically.  By default, owner and
group numbers are converted to names on each replica and these names are
synchronized.  (The special user id 0 and the special group 0 are never
mapped via user/group names even if this preference is not set.)
\end{itemize}

\item Added an integer-valued preference \verb|perms| that can be used to
control the propagation of permission bits.  The value of this preference
is a mask indicating which permission bits should be synchronized.  It is
set by default to $0o1777$: all bits but the set-uid and set-gid bits are
synchronised (synchronizing theses latter bits can be a security hazard).
If you want to synchronize all bits, you can set the value of this
preference to $-1$.

\item Added a \verb|log| preference (default \verb|false|), which makes
Unison keep a complete record of the changes it makes to the replicas.
By default, this record is written to a file called \verb|unison.log| in
the user's home directory (the value of the \verb|HOME| environment
variable).  If you want it someplace else, set the \verb|logfile|
preference to the full pathname you want Unison to use.

\item Added an \verb|ignorenot| preference that maintains a set of patterns 
  for paths that should definitely {\em not} be ignored, whether or not
  they match an \verb|ignore| pattern.  (That is, a path will now be ignored
  iff it matches an ignore pattern and does not match any ignorenot patterns.)
\end{itemize}
  
\item User-interface improvements:
\begin{itemize}
\item Roots are now displayed in the user interface in the same order
as they were given on the command line or in the preferences file.
\item When the \verb|batch| preference is set, the graphical user interface no 
  longer waits for user confirmation when it displays a warning message: it
  simply pops up an advisory window with a Dismiss button at the bottom and
  keeps on going.
\item Added a new preference for controlling how many status messages are
  printed during update detection: \verb|statusdepth| controls the maximum
  depth for paths on the local machine (longer paths are not displayed, nor
  are non-directory paths).  The value should be an integer; default is 1.  
\item Removed the \verb|trace| and \verb|silent| preferences.  They did
not seem very useful, and there were too many preferences for controlling
output in various ways.
\item The text UI now displays just the default command (the one that
will be used if the user just types \verb|<return>|) instead of all
available commands.  Typing \verb|?| will print the full list of
possibilities.
\item The function that finds the canonical hostname of the local host
(which is used, for example, in calculating the name of the archive file
used to remember which files have been synchronized) normally uses the
\verb|gethostname| operating system call.  However, if the environment
variable \verb|UNISONLOCALHOSTNAME| is set, its value will now be used
instead.  This makes it easier to use Unison in situations where a
machine's name changes frequently (e.g., because it is a laptop and gets
moved around a lot).
\item File owner and group are now displayed in the ``detail window'' at
the bottom of the screen, when unison is configured to synchronize them.
\end{itemize}

\item For hackers:
\begin{itemize}
\item Updated to Jacques Garrigue's new version of \verb|lablgtk|, which
  means we can throw away our local patched version.  

  If you're compiling the GTK version of unison from sources, you'll need
  to update your copy of lablgtk to the developers release.
  (Warning: installing lablgtk under Windows is currently a bit
  challenging.) 

\item The TODO.txt file (in the source distribution) has been cleaned up
and reorganized.  The list of pending tasks should be much easier to
make sense of, for people that may want to contribute their programming
energies.  There is also a separate file BUGS.txt for open bugs.
\item The Tk user interface has been removed (it was not being maintained
and no longer compiles).
\item The \verb|debug| preference now prints quite a bit of additional
information that should be useful for identifying sources of problems.
\item The version number of the remote server is now checked right away 
  during the connection setup handshake, rather than later.  (Somebody
  sent a bug report of a server crash that turned out to come from using
  inconsistent versions: better to check this earlier and in a way that
  can't crash either client or server.)
\item Unison now runs correctly on 64-bit architectures (e.g. Alpha
linux).  We will not be distributing binaries for these architectures
ourselves (at least for a while) but if someone would like to make them
available, we'll be glad to provide a link to them.
\end{itemize}

\item Bug fixes:
\begin{itemize}
\item Pattern matching (e.g. for \verb|ignore|) is now case-insensitive
  when Unison is in case-insensitive mode (i.e., when one of the replicas
  is on a windows machine).
\item Some people had trouble with mysterious failures during
  propagation of updates, where files would be falsely reported as having
  changed during synchronization.  This should be fixed.
\item Numerous smaller fixes.
\end{itemize}
\end{changesfromversion}

\begin{changesfromversion}{2.4.1}
\item Added a number of 'sorting modes' for the user interface.  By
default, conflicting changes are displayed at the top, and the rest of
the entries are sorted in alphabetical order.  This behavior can be
changed in the following ways:
\begin{itemize}
\item Setting  the \verb|sortnewfirst| preference to \verb|true| causes
newly created files to be displayed before changed files.
\item Setting \verb|sortbysize| causes files to be displayed in
increasing order of size.
\item Giving the preference \verb|sortfirst=<pattern>| (where
\verb|<pattern>| is a path descriptor in the same format as 'ignore' and 'follow'
patterns, causes paths matching this pattern to be displayed first.
\item Similarly, giving the preference \verb|sortlast=<pattern>| 
causes paths matching this pattern to be displayed last.
\end{itemize}
The sorting preferences are described in more detail in the user manual.
The \verb|sortnewfirst| and \verb|sortbysize| flags can also be accessed
from the 'Sort' menu in the grpahical user interface.

\item Added two new preferences that can be used to change unison's
fundamental behavior to make it more like a mirroring tool instead of
a synchronizer.
\begin{itemize}
\item Giving the preference \verb|prefer| with argument \verb|<root>|
(by adding \verb|-prefer <root>| to the command line or \verb|prefer=<root>|)
to your profile) means that, if there is a conflict, the contents of
\verb|<root>| 
should be propagated to the other replica (with no questions asked).
Non-conflicting changes are treated as usual.
\item Giving the preference \verb|force| with argument \verb|<root>|
will make unison resolve {\em all} differences in favor of the given
root, even if it was the other replica that was changed.
\end{itemize}
These options should be used with care!  (More information is available in
the manual.)

\item Small changes:
\begin{itemize}
\item 
Changed default answer to 'Yes' in all two-button dialogs in the 
  graphical interface (this seems more intuitive).

\item The \verb|rsync| preference has been removed (it was used to
activate rsync compression for file transfers, but rsync compression is
now enabled by default). 
\item  In the text user interface, the arrows indicating which direction
changes are being 
  propagated are printed differently when the user has overridded Unison's
  default recommendation (\verb|====>| instead of \verb|---->|).  This
  matches the behavior of the graphical interface, which displays such
  arrows in a different color.
\item Carriage returns (Control-M's) are ignored at the ends of lines in
  profiles, for Windows compatibility.
\item All preferences are now fully documented in the user manual. 
\end{itemize}
\end{changesfromversion}

\begin{changesfromversion}{2.3.12}
\item \incompatible{} Archive format has changed.  Make sure you
synchronize your replicas before upgrading, to avoid spurious
conflicts.  The first sync after upgrading will be slow.

\item New/improved functionality:
\begin{itemize}
\item  A new preference -sortbysize controls the order in which changes
  are displayed to the user: when it is set to true, the smallest
  changed files are displayed first.  (The default setting is false.) 
\item A new preference -sortnewfirst causes newly created files to be 
  listed before other updates in the user interface.
\item We now allow the ssh protocol to specify a port.  
\item Incompatible change: The unison: protocol is deprecated, and we added
  file: and socket:.  You may have to modify your profiles in the
  .unison directory.
  If a replica is specified without an explicit protocol, we now
  assume it refers to a file.  (Previously "//saul/foo" meant to use
  SSH to connect to saul, then access the foo directory.  Now it means
  to access saul via a remote file mechanism such as samba; the old
  effect is now achieved by writing {\tt ssh://saul/foo}.)
\item Changed the startup sequence for the case where roots are given but
  no profile is given on the command line.  The new behavior is to
  use the default profile (creating it if it does not exist), and
  temporarily override its roots.  The manual claimed that this case
  would work by reading no profile at all, but AFAIK this was never
  true.
\item In all user interfaces, files with conflicts are always listed first
\item A new preference 'sshversion' can be used to control which version
  of ssh should be used to connect to the server.  Legal values are 1 and 2.
  (Default is empty, which will make unison use whatever version of ssh
  is installed as the default 'ssh' command.)
\item The situation when the permissions of a file was updated the same on
  both side is now handled correctly (we used to report a spurious conflict)

\end{itemize}

\item Improvements for the Windows version:
\begin{itemize}
\item The fact that filenames are treated case-insensitively under
Windows should now be handled correctly.  The exact behavior is described
in the cross-platform section of the manual.
\item It should be possible to synchronize with Windows shares, e.g.,
  //host/drive/path.
\item Workarounds to the bug in syncing root directories in Windows.
The most difficult thing to fix is an ocaml bug: Unix.opendir fails on
c: in some versions of Windows.
\end{itemize}

\item Improvements to the GTK user interface (the Tk interface is no
longer being maintained): 
\begin{itemize}
\item The UI now displays actions differently (in blue) when they have been
  explicitly changed by the user from Unison's default recommendation.
\item More colorful appearance.
\item The initial profile selection window works better.
\item If any transfers failed, a message to this effect is displayed along with
  'Synchronization complete' at the end of the transfer phase (in case they
  may have scrolled off the top).
\item Added a global progress meter, displaying the percentage of {\em total}
  bytes that have been transferred so far.
\end{itemize}

\item Improvements to the text user interface:
\begin{itemize}
\item The file details will be displayed automatically when a
  conflict is been detected.
\item when a warning is generated (e.g. for a temporary
  file left over from a previous run of unison) Unison will no longer
  wait for a response if it is running in -batch mode.
\item The UI now displays a short list of possible inputs each time it waits
  for user interaction.  
\item The UI now quits immediately (rather than looping back and starting
  the interaction again) if the user presses 'q' when asked whether to 
  propagate changes.
\item Pressing 'g' in the text user interface will proceed immediately
  with propagating updates, without asking any more questions.
\end{itemize}

\item Documentation and installation changes:
\begin{itemize}
\item The manual now includes a FAQ, plus sections on common problems and
on tricks contributed by users.
\item Both the download page and the download directory explicitly say
what are the current stable and beta-test version numbers.
\item The OCaml sources for the up-to-the-minute developers' version (not
guaranteed to be stable, or even to compile, at any given time!) are now
available from the download page.
\item Added a subsection to the manual describing cross-platform
  issues (case conflicts, illegal filenames)
\end{itemize}

\item Many small bug fixes and random improvements.

\end{changesfromversion}

\begin{changesfromversion}{2.3.1}
\item Several bug fixes.  The most important is a bug in the rsync
module that would occasionally cause change propagation to fail with a
'rename' error.
\end{changesfromversion}

\begin{changesfromversion}{2.2}
\item The multi-threaded transport system is now disabled by default.
(It is not stable enough yet.)
\item Various bug fixes.
\item A new experimental feature: 

  The final component of a -path argument may now be the wildcard 
  specifier \verb|*|.  When Unison sees such a path, it expands this path on 
  the client into into the corresponding list of paths by listing the
  contents of that directory.  

  Note that if you use wildcard paths from the command line, you will
  probably need to use quotes or a backslash to prevent the * from
  being interpreted by your shell.

  If both roots are local, the contents of the first one will be used
  for expanding wildcard paths.  (Nb: this is the first one {\em after} the
  canonization step -- i.e., the one that is listed first in the user 
  interface -- not the one listed first on the command line or in the
  preferences file.)
\end{changesfromversion}

\begin{changesfromversion}{2.1}
\item The transport subsystem now includes an implementation by
Sylvain Gommier and Norman Ramsey of Tridgell and Mackerras's
\verb|rsync| protocol.  This protocol achieves much faster 
transfers when only a small part of a large file has been changed by
sending just diffs.  This feature is mainly helpful for transfers over
slow links---on fast local area networks it can actually degrade
performance---so we have left it off by default.  Start unison with
the \verb|-rsync| option (or put \verb|rsync=true| in your preferences
file) to turn it on.

\item ``Progress bars'' are now diplayed during remote file transfers,
showing what percentage of each file has been transferred so far.

\item The version numbering scheme has changed.  New releases will now
      be have numbers like 2.2.30, where the second component is
      incremented on every significant public release and the third
      component is the ``patch level.''

\item Miscellaneous improvements to the GTK-based user interface.
\item The manual  is now available in PDF format.

\item We are experimenting with using a multi-threaded transport
subsystem to transfer several files at the same time, making
much more effective use of available network bandwidth.  This feature
is not completely stable yet, so by default it is disabled in the
release version of Unison.

If you want to play with the multi-threaded version, you'll need to
recompile Unison from sources (as described in the documentation),
setting the THREADS flag in Makefile.OCaml to true.  Make sure that
your OCaml compiler has been installed with the \verb|-with-pthreads|
configuration option.  (You can verify this by checking whether the
file \verb|threads/threads.cma| in the OCaml standard library
directory contains the string \verb|-lpthread| near the end.)
\end{changesfromversion}

\begin{changesfromversion}{1.292}
\item Reduced memory footprint (this is especially important during
the first run of unison, where it has to gather information about all
the files in both repositories). 
\item Fixed a bug that would cause the socket server under NT to fail
  after the client exits. 
\item Added a SHIFT modifier to the Ignore menu shortcut keys in GTK
  interface (to avoid hitting them accidentally).  
\end{changesfromversion}

\begin{changesfromversion}{1.231}
\item Tunneling over ssh is now supported in the Windows version.  See
the installation section of the manual for detailed instructions.

\item The transport subsystem now includes an implementation of the
\verb|rsync| protocol, built by Sylvain Gommier and Norman Ramsey.
This protocol achieves much faster transfers when only a small part of
a large file has been changed by sending just diffs.  The rsync
feature is off by default in the current version.  Use the
\verb|-rsync| switch to turn it on.  (Nb.  We still have a lot of
tuning to do: you may not notice much speedup yet.)

\item We're experimenting with a multi-threaded transport subsystem,
written by Jerome Vouillon.  The downloadable binaries are still
single-threaded: if you want to try the multi-threaded version, you'll
need to recompile from sources.  (Say \verb|make THREADS=true|.)
Native thread support from the compiler is required.  Use the option
\verb|-threads N| to select the maximal number of concurrent 
threads (default is 5).  Multi-threaded
and single-threaded clients/servers can interoperate.  

\item A new GTK-based user interface is now available, thanks to
Jacques Garrigue.  The Tk user interface still works, but we'll be
shifting development effort to the GTK interface from now on.
\item OCaml 3.00 is now required for compiling Unison from sources.
The modules \verb|uitk| and \verb|myfileselect| have been changed to
use labltk instead of camltk.  To compile the Tk interface in Windows,
you must have ocaml-3.00 and tk8.3.  When installing tk8.3, put it in
\verb|c:\Tcl| rather than the suggested \verb|c:\Program Files\Tcl|, 
and be sure to install the headers and libraries (which are not 
installed by default).

\item Added a new \verb|-addversionno| switch, which causes unison to
use \verb|unison-<currentversionnumber>| instead of just \verb|unison|
as the remote server command.  This allows multiple versions of unison
to coexist conveniently on the same server: whichever version is run
on the client, the same version will be selected on the server.
\end{changesfromversion}

\begin{changesfromversion}{1.219}
\item \incompatible{} Archive format has changed.  Make sure you
synchronize your replicas before upgrading, to avoid spurious
conflicts.  The first sync after upgrading will be slow.

\item This version fixes several annoying bugs, including:
\begin{itemize}
\item Some cases where propagation of file permissions was not
working.
\item umask is now ignored when creating directories
\item directories are create writable, so that a read-only directory and
    its contents can be propagated.
\item Handling of warnings generated by the server.
\item Synchronizing a path whose parent is not a directory on both sides is
now flagged as erroneous.  
\item Fixed some bugs related to symnbolic links and nonexistant roots.
\begin{itemize}
\item 
   When a change (deletion or new contents) is propagated onto a 
     'follow'ed symlink, the file pointed to by the link is now changed.
     (We used to change the link itself, which doesn't fit our assertion
     that 'follow' means the link is completely invisible)
   \item When one root did not exist, propagating the other root on top of it
     used to fail, becuase unison could not calculate the working directory
     into which to write changes.  This should be fixed.
\end{itemize}
\end{itemize}

\item A human-readable timestamp has been added to Unison's archive files.

\item The semantics of Path and Name regular expressions now
correspond better. 

\item Some minor improvements to the text UI (e.g. a command for going
back to previous items)

\item The organization of the export directory has changed --- should
be easier to find / download things now.
\end{changesfromversion}

\begin{changesfromversion}{1.200}
\item \incompatible{} Archive format has changed.  Make sure you
synchronize your replicas before upgrading, to avoid spurious
conflicts.  The first sync after upgrading will be slow.

\item This version has not been tested extensively on Windows.

\item Major internal changes designed to make unison safer to run
at the same time as the replicas are being changed by the user.

\item Internal performance improvements.  
\end{changesfromversion}

\begin{changesfromversion}{1.190}
\item \incompatible{} Archive format has changed.  Make sure you
synchronize your replicas before upgrading, to avoid spurious
conflicts.  The first sync after upgrading will be slow.

\item A number of internal functions have been changed to reduce the
amount of memory allocation, especially during the first
synchronization.  This should help power users with very big replicas.

\item Reimplementation of low-level remote procedure call stuff, in
preparation for adding rsync-like smart file transfer in a later
release.   

\item Miscellaneous bug fixes.
\end{changesfromversion}

\begin{changesfromversion}{1.180}
\item \incompatible{} Archive format has changed.  Make sure you
synchronize your replicas before upgrading, to avoid spurious
conflicts.  The first sync after upgrading will be slow.

\item Fixed some small bugs in the interpretation of ignore patterns. 

\item Fixed some problems that were preventing the Windows version
from working correctly when click-started.

\item Fixes to treatment of file permissions under Windows, which were
causing spurious reports of different permissions when synchronizing
between windows and unix systems.

\item Fixed one more non-tail-recursive list processing function,
which was causing stack overflows when synchronizing very large
replicas. 
\end{changesfromversion}

\begin{changesfromversion}{1.169}
\item The text user interface now provides commands for ignoring
  files. 
\item We found and fixed some {\em more} non-tail-recursive list
  processing functions.  Some power users have reported success with
  very large replicas.
\item \incompatible 
Files ending in \verb|.tmp| are no longer ignored automatically.  If you want
to ignore such files, put an appropriate ignore pattern in your profile.

\item \incompatible{} The syntax of {\tt ignore} and {\tt follow}
patterns has changed. Instead of putting a line of the form
\begin{verbatim}
                 ignore = <regexp>
\end{verbatim}
  in your profile ({\tt .unison/default.prf}), you should put:
\begin{verbatim}
                 ignore = Regexp <regexp>
\end{verbatim}
Moreover, two other styles of pattern are also recognized:
\begin{verbatim}
                 ignore = Name <name>
\end{verbatim}
matches any path in which one component matches \verb|<name>|, while
\begin{verbatim}
                 ignore = Path <path>
\end{verbatim}
matches exactly the path \verb|<path>|.

Standard ``globbing'' conventions can be used in \verb|<name>| and
\verb|<path>|:  
\begin{itemize}
\item a \verb|?| matches any single character except \verb|/|
\item a \verb|*| matches any sequence of characters not including \verb|/|
\item \verb|[xyz]| matches any character from the set $\{{\tt x},
  {\tt y}, {\tt z} \}$
\item \verb|{a,bb,ccc}| matches any one of \verb|a|, \verb|bb|, or
  \verb|ccc|. 
\end{itemize}

See the user manual for some examples.
\end{changesfromversion}

\begin{changesfromversion}{1.146}
\item Some users were reporting stack overflows when synchronizing
  huge directories.  We found and fixed some non-tail-recursive list
  processing functions, which we hope will solve the problem.  Please 
  give it a try and let us know.
\item Major additions to the documentation.  
\end{changesfromversion}

\begin{changesfromversion}{1.142}
\item Major internal tidying and many small bugfixes.
\item Major additions to the user manual.
\item Unison can now be started with no arguments -- it will prompt
automatically for the name of a profile file containing the roots to
be synchronized.  This makes it possible to start the graphical UI
from a desktop icon.
\item Fixed a small bug where the text UI on NT was raising a 'no such
  signal' exception.
\end{changesfromversion}

\begin{changesfromversion}{1.139}
\item The precompiled windows binary in the last release was compiled
with an old OCaml compiler, causing propagation of permissions not to
work (and perhaps leading to some other strange behaviors we've heard
reports about).  This has been corrected.  If you're using precompiled
binaries on Windows, please upgrade.
\item Added a \verb|-debug| command line flag, which controls debugging
of various modules.  Say \verb|-debug XXX| to enable debug tracing for
module \verb|XXX|, or \verb|-debug all| to turn on absolutely everything.
\item Fixed a small bug where the text UI on NT was raising a 'no such signal'
exception.
\end{changesfromversion}

\begin{changesfromversion}{1.111}
\item \incompatible{} The names and formats of the preference files in
the .unison directory have changed.  In particular:
\begin{itemize}
\item the file ``prefs'' should be renamed to default.prf
\item the contents of the file ``ignore'' should be merged into
  default.prf.  Each line of the form \verb|REGEXP| in ignore should
  become a line of the form \verb|ignore = REGEXP| in default.prf.
\end{itemize}
\item Unison now handles permission bits and  symbolic links.  See the
manual for details.

\item You can now have different preference files in your .unison
directory.  If you start unison like this
\begin{verbatim}
             unison profilename
\end{verbatim}
(i.e. with just one ``anonymous'' command-line argument), then the
file \verb|~/.unison/profilename.prf| will be loaded instead of
\verb|default.prf|. 

\item Some improvements to terminal handling in the text user interface

\item Added a switch -killServer that terminates the remote server process
when the unison client is shutting down, even when using sockets for 
communication.  (By default, a remote server created using ssh/rsh is 
terminated automatically, while a socket server is left running.)
\item When started in 'socket server' mode, unison prints 'server started' on
  stderr when it is ready to accept connections.  
  (This may be useful for scripts that want to tell when a socket-mode server 
  has finished initalization.)
\item We now make a nightly mirror of our current internal development
  tree, in case anyone wants an up-to-the-minute version to hack
  around with.
\item Added a file CONTRIB with some suggestions for how to help us
make Unison better.
\end{changesfromversion}



\finishlater{
\SECTION{Other Synchronizers}{other}{other}

Unison is just one of several file synchronizers that are currently
available. 

Check out:
  http://www.bell-labs.com/project/stage/
  I notice a bunch of people are also doing "data vaulting", e.g.,
    http://www.pc.ibm.com/us/thinkpad/datavault.html
  midnight commander??

Also:
  D. Duchamp
  A Toolkit Approach to Partially Disconnected Operation
  Proc. USENIX 1997 Ann. Technical Conf.
  USENIX, Anaheim CA, pp. 305-318, January 1997
}

\finishlater{
\SECTION{TODO}{todo}{ }

Things to write about:
\begin{itemize}
\item When started in 'socket server' mode, Unison prints 'server started' on
  stderr when it is ready to accept connections.  
  (This may be useful for scripts that want to tell when a socket-mode server 
  has finished initialization.)
\item {\tt DANGER.README}.
\end{itemize}
}

\finishlater{
Things to write about later:
\begin{itemize}
\item Document different reporting of file status when no archives
  were found.
\item Document buttons in graphical UI
\end{itemize}
}

\iftextversion
\SECTION{Junk}{ }{ } 
\fi

\ifhevea\begin{rawhtml}</div>\end{rawhtml}\fi

\end{document}

      \end{quote}
    \fi
  \else
    \@opentoc{htoc}
    \tableofcontents
  \fi
}
\makeatother

\newcommand{\SNIP}[2]{%
\ifhevea\iftextversion
\begin{rawhtml}<pre>----SNIP----
\end{rawhtml}
#1
#2 %
\begin{rawhtml}</pre>\end{rawhtml}%
\fi\fi
}

\newcommand{\sectionref}[2]{%
\ifhevea
  \iftextversion
    the section ``#2''
  \else
    the \url{##1}{#2} section%
  \fi
\else
  Section~\ref{#1} {[#2]}%
\fi
}

\newcommand{\bcpurl}[1]{\url{#1}}

\newcommand{\urlref}[2]{\bcpurl{##1}{#2}}
\newcommand{\ONEURL}[1]{%
  \iftextversion#1\else{\def~{\symbol{"7E}}\oneurl{#1}}\fi}
\newcommand{\URL}[2]{%
  \iftextversion#2 (#1)\else\bcpurl{#1}{#2}\fi}
\newcommand{\SHOWURL}[2]{%
  \ifhevea\URL{#1}{#2}\else#2\footnote{{\def~{\symbol{"7E}}\tt #1}}\fi}

% Usage: \SECTION{Title and menu item name}{tex label}{man section id}
\newcommand{\SECTION}[3]{%
  \ifhevea
    \SNIP{#1}{#3}%
    \iftextversion\else \@print{<hr>}\fi%
    \section*{\label{#2}#1}%
  \else
    \newpage
    \section{\label{#2}#1}%
    \addtocontents{htoc}{{\string\large\string\bf\string\urlref{#2}{#1}}\\}%
  \fi
}

\newcommand{\SUBSECTION}[2]{%
  \ifhevea
    \subsection*{\label{#2}#1}%
  \else
    \subsection{\label{#2}#1}%
    \addtocontents{htoc}{\hspace{10em}\bullet\string\urlref{#2}{#1}\\}
  \fi
}

\newcommand{\SUBSUBSECTION}[2]{%
  \ifhevea
    \subsubsection*{\label{#2}#1}%
  \else
    \subsubsection{\label{#2}#1}%
    \addtocontents{htoc}{\hspace{18em}\string\urlref{#2}{#1}\\}
  \fi
}

\newcommand{\TOPSUBSECTION}[2]{%
  \ifhevea\SNIP{#1}{#2}\fi
  \SUBSECTION{#1}{#2}%
}

% The quote-based macros looks a imperfect, perhaps due to the lack of 
% alignment
% \newenvironment{textui}{{\em Textual Interface:}\begin{quote}}{\end{quote}}
% \newenvironment{tkui}{{\em Graphical Interface:}\begin{quote}}{\end{quote}}
\newenvironment{textui}{\medskip{\em Textual Interface:}\begin{itemize}\item[]
  }{\end{itemize}}
\newenvironment{tkui}{\medskip{\em Graphical Interface:}\begin{itemize}\item[]
  }{\end{itemize}}
\newenvironment{changesfromversion}[1]{%
  \noindent Changes since #1:
  \begin{itemize}
}{
  \end{itemize}
}

\newcommand{\incompatible}{%
  \iftextversion
    INCOMPATIBLE CHANGE:
  \else
    {\bf Incompatible change:} 
  \fi}

\newcommand{\UNISONUSERS}{\URL{mailto:unison-users@yahoogroups.com}{{\tt
      unison-users@yahoogroups.com}}}
\newcommand{\UNISONHACKERS}{\URL{mailto:unison-hackers@yahoogroups.com}{{\tt
      unison-hackers@yahoogroups.com}}}

\ifhevea
 \makeatletter
 \let\oldmeta=\@meta
 \renewcommand{\@meta}{%
 \oldmeta
\ifdraft
 \begin{rawhtml}
 <META name="Author" content="Benjamin C. Pierce">
 <link rel="stylesheet" href="/home/bcpierce/pub/unison/unison.css">
 \end{rawhtml}
\else
 \begin{rawhtml}
 <META name="Author" content="Benjamin C. Pierce">
 <link rel="stylesheet" href="http://www.cis.upenn.edu/~bcpierce/unison/unison.css">
 \end{rawhtml}
\fi
}
 \makeatother
\fi


\fulltrue

%\newcommand{\NT}[1]{\(\langle\)\textit{#1}\(\rangle\)}
\newcommand{\NT}[1]{\textit{#1}}
\newcommand{\ARG}[1]{\texttt{\textit{#1}}}

%%%%%%%%%%%%%%%%%%%%%%%%%%%%%%%%%%%%%%%%%%%%%%%%%%%%%%%%%%%%%%%%%%%%%%
%%%%%%%%%%%%%%%%%%%%%%%%%%%%%%%%%%%%%%%%%%%%%%%%%%%%%%%%%%%%%%%%%%%%%%
\begin{document}

\ifhevea\begin{rawhtml}<div id="manualbody">\end{rawhtml}\fi

\ifhevea\else\bigskip\fi%
\ifdraft%
\begin{center}%
{\Huge \ifhevea\red\fi DraftDraftDraftDraft}%
\end{center}%
\ifhevea\else \bigskip \fi
\fi

\ifhevea\begin{rawhtml}<div id="manualheader">\end{rawhtml}%
\else \thispagestyle{empty}
\fi%
\SNIP{About Unison}{about}%
\iftextversion
  \section*{Unison File Synchronizer 
%%   \\ 
%%   \ONEURL{http://www.cis.upenn.edu/\home{bcpierce}/unison}
  \\
  Version
  \unisonversion 
  }
\else%
  \ifhevea\else \vspace*{2in} \fi%
  \begin{center}%
  \Huge{\ifhevea\black\else\bf \fi Unison File  Synchronizer}%
%%  \ifhevea \\ \else \\[2ex] \fi
%%   \large
%%   \ONEURL{http://www.cis.upenn.edu/\home{bcpierce}/unison}
  \ifhevea \\ \else \\[2ex] \fi%
  \huge {\ifhevea\black\else\bf \fi User Manual and Reference Guide}%
  \ifhevea \\ \else \\[6ex] \fi%
  \LARGE%
  Version \unisonversion \\[4ex] %
  % \today %
  \large Copyright 1998-2009, Benjamin C. Pierce
  \end{center}%
\fi%
%
%
\ifhevea\begin{rawhtml}</div>\end{rawhtml}\fi

\ifhevea\else\newpage\fi 
\TABLEOFCONTENTS
\ifhevea\else\newpage\fi

\SECTION{Overview}{overview}{ }

Unison is a file-synchronization tool for Unix and Windows.  It allows
two replicas of a collection of files and directories to be stored on
different hosts (or different disks on the same host), modified
separately, and then brought up to date by propagating the changes in
each replica to the other.

Unison 
shares a number of features with tools such as configuration
management packages %
(\URL{http://www.cyclic.com/}{CVS},
\URL{http://www.XCF.Berkeley.EDU/\home{jmacd}/prcs.html}{PRCS},
etc.),
%
distributed filesystems 
(\URL{http://www.coda.cs.cmu.edu/}{Coda}, 
etc.),
%
uni-directional mirroring utilities 
(\URL{http://samba.anu.edu.au/rsync/}{rsync}, 
etc.),
%
and other synchronizers 
(\URL{http://www.pumatech.com}{Intellisync}, 
\URL{http://www.merl.com/reports/TR99-14/}{Reconcile},
etc).  \finishlater{Midnight commander??}
%
However, there are several points where it differs:
\begin{itemize}
\item Unison runs on both Windows (95, 98, NT, 2k, and XP) and Unix (OSX, Solaris,
  Linux, etc.) systems.  Moreover, Unison works {\em across}
  platforms, allowing you to synchronize a Windows laptop with a
  Unix server, for example.
\item Unlike a distributed filesystem, Unison is a user-level program:
  there is no need to modify the kernel or to have
  superuser privileges on either host.
\item Unlike simple mirroring or backup utilities, Unison can deal
  with updates to both replicas of a distributed directory structure.
  Updates that do not conflict are propagated automatically.
  Conflicting updates are detected and displayed.
\item Unison works between any pair of machines connected to the
  internet, communicating over either a direct socket link or
  tunneling over an encrypted {\tt ssh} connection.
  It is careful with network bandwidth, and runs well over slow links
  such as PPP connections.  Transfers of small updates to large files are
  optimized using a compression protocol similar to rsync.
\item Unison has a clear and precise specification\iffull, described
below. \else. \fi
  \item Unison is resilient to failure.  It is careful to leave the
  replicas and its own private structures in a sensible state at all
  times, even in case of abnormal termination or communication
  failures.
% \item Unison is easy to install.  Just one executable file (for each
%   host architecture) is all you need.
\item Unison is free; full source code is available under the GNU
Public License.
\end{itemize}



\ifhevea\else\bigskip\fi

% \begin{quote}
% {\bf\ifhevea\red\fi Warning:} The current implementation of Unison is
% considered beta-test software.  It is in daily use by quite a few
% people, but there are still undoubtedly some bugs.  If you choose to 
% use it to synchronize important data, please pay careful attention
% to what it is doing!  Also, the installation/setup procedure is not
% yet as smooth as we want it to be.
% \end{quote}


\SECTION{Preface}{intro}{ }

\TOPSUBSECTION{People}{people}

\URL{http://www.cis.upenn.edu/\home{bcpierce}/}{Benjamin Pierce} leads the
Unison project.   
%
The current version of Unison was designed and implemented by
    \URL{http://www.research.att.com/\home{trevor}/}{Trevor Jim},
    \URL{http://www.cis.upenn.edu/\home{bcpierce}/}{Benjamin Pierce},
and
    \URL{http://www.pps.jussieu.fr/\home{vouillon}/}{J\'{e}r\^{o}me Vouillon},
with
    \URL{http://alan.petitepomme.net/}{Alan Schmitt},
    {Malo Denielou},
    \URL{http://www.brics.dk/\home{zheyang}/}{Zhe Yang},
    Sylvain Gommier, and
    Matthieu Goulay.
%
The Mac user interface was started by Trevor Jim and enormously improved by
Ben Willmore. 
%
Our implementation of the
  \URL{http://samba.org/rsync/}{rsync}
  protocol was built by
  \URL{http://www.eecs.harvard.edu/\home{nr}/}{Norman Ramsey}
  and Sylvain Gommier.  It is is based on
  \URL{http://samba.anu.edu.au/\home{tridge}/}{Andrew Tridgell}'s
  \URL{http://samba.anu.edu.au/\home{tridge}/phd\_thesis.pdf}{thesis work}
  and inspired by his
  \URL{http://samba.org/rsync/}{rsync}
  utility.
% \finish{Our low-level fingerprinting implementation uses an algorithm
% by Michael Rabin and incorporates some coding tricks from Andrei
% Broder and Mike Burrows.}
%
The mirroring and merging functionality was implemented by
  Sylvain Roy, improved by Malo Denielou, and improved yet further by
  St\'ephane Lescuyer.
%
 \URL{http://wwwfun.kurims.kyoto-u.ac.jp/\home{garrigue}/}{Jacques Garrigue}
 contributed the original Gtk version of the user
  interface; the Gtk2 version was built by Stephen Tse. 
%
Sundar Balasubramaniam helped build a prototype implementation of
an earlier synchronizer in Java.
\URL{http://www.cis.upenn.edu/\home{ishin}/}{Insik Shin}
and
\URL{http://www.cis.upenn.edu/\home{lee}/}{Insup Lee} contributed design
ideas to this implementation.
\URL{http://research.microsoft.com/\home{fournet}/}{Cedric Fournet}
contributed to an even earlier prototype.

\TOPSUBSECTION{Mailing Lists and Bug Reporting}{lists}

\paragraph{Mailing Lists:}

Moderated mailing lists are available for announcements of new
versions, discussions among users, and discussions among developers.
See \ONEURL{http://www.cis.upenn.edu/~bcpierce/unison/lists.html} for
more information.

\endinput
%%%%%%%%%%%%%%%%%%%%%%%%%%%%%%%%%%%%%%%%%%%%%%%%%%%%%%%%%%%%%%%%%%%%%%%%%%%%%%
%%%%%%%%%%%%%%%%%%%%%%%%%%%%%%%%%%%%%%%%%%%%%%%%%%%%%%%%%%%%%%%%%%%%%%%%%%%%%%
%%%%%%%%%%%%%%%%%%%%%%%%%%%%%%%%%%%%%%%%%%%%%%%%%%%%%%%%%%%%%%%%%%%%%%%%%%%%%%
%%%%%%%%%%%%%%%%%%%%%%%%%%%%%%%%%%%%%%%%%%%%%%%%%%%%%%%%%%%%%%%%%%%%%%%%%%%%%%
%%%%%%%%%%%%%%%%%%%%%%%%%%%%%%%%%%%%%%%%%%%%%%%%%%%%%%%%%%%%%%%%%%%%%%%%%%%%%%

It is strongly recommended that all Unison users subscribe to one of the
first two:  
\begin{itemize}
\item {\bf
  \URL{http://groups.yahoo.com/group/unison-announce}{unison-announce}}
is a moderated list where new Unison releases are announced.  It is very
low volume, averaging about one message per month. 

To subscribe, you can either visit 
  \ONEURL{http://groups.yahoo.com/group/unison-announce} (you will be
  asked to create a Yahoo groups account if you do not already have one),
  or else send an email to
  {\tt
  \URL{mailto:unison-announce-subscribe@groups.yahoo.com}{unison-announce-subscribe@groups.yahoo.com}}
  (which will 
  simply add you to the list, whether you have a Yahoo account or not).

  \item {\bf
    \URL{http://groups.yahoo.com/group/unison-users}{unison-users}} is a
  somewhat-higher-volume list for users of unison.  It is used for
  discussions of many sorts --- proposals and designs for new features,
  installation and configuration questions, usage tips, etc.  It is also
  moderated, but just to filter spam.

To subscribe, you can either visit 
  \ONEURL{http://groups.yahoo.com/group/unison-users}
  or else send an email to
  {\tt \URL{mailto:unison-users-subscribe@groups.yahoo.com}{unison-users-subscribe@groups.yahoo.com}}.

Release announcements are made on both of these lists, so there is
no need to subscribe to both.

\item {\bf
  \URL{}{unison-hackers}} is
for informal discussion among Unison developers.  Anyone who considers
themselves a Unison expert and wishes to lend a hand with maintaining and
improving Unison is welcome to join.  Only members can post to this list. 

To subscribe, you can either visit 
  \ONEURL{http://lists.seas.upenn.edu/mailman/listinfo/unison-hackers}
  or else send an email to
  {\tt unison-hackers-subscribe at lists dot seas dot upenn dot edu}.
\end{itemize}
Archives of all the lists are available via the
above links. Postings are limited to members, to reduce the spam load on moderators.

\paragraph{Reporting bugs:}

If Unison is not working the way you expect, here are some steps to 
follow.
\begin{itemize}
\item First, take a look at the Unison documentation, especially the
FAQ section.  Lots of questions are answered there.

\item Next, try running Unison with the {\tt -debug all} command line
option.  This will cause Unison to generate a detailed trace of what it's
doing, which may help pinpoint where the problem is occurring.

\item If this doesn't clarify matters, try sending an email describing
your problem to the users list at
\URL{mailto:unison-users@groups.yahoo.com}{{\tt 
    unison-users@groups.yahoo.com}}.  
Please include the version of Unison you are using (type {\tt unison
  -version}), the kind of machine(s) you are running it on, a record of
what gets printed when the {\tt -debug all} option is included, and as
much information as you can about what went wrong.
\end{itemize}

\paragraph{Feature Requests:}

Please post your feature requests, suggestions, etc. to the {\tt
  unison-users} list.   



\TOPSUBSECTION{Development Status}{status}

Unison is no longer under active development as a research
project.  (Our research efforts  are now focused on a follow-on
project called Harmony, described at
\ONEURL{http://www.cis.upenn.edu/\home{bcpierce}/harmony}.) 
At this point, there is no one whose job it is to maintain Unison,
fix bugs, or answer questions.

However, the original developers are all still using Unison daily.  It
will continue to be maintained and supported for the foreseeable future,
and we will occasionally release new versions with bug fixes, small
improvements, and contributed patches.

Reports of bugs affecting correctness or safety are of interest to many
people and will generally get high priority.  Other bug reports will be
looked at as time permits.  Bugs should be reported to the users list at
\UNISONUSERS. 

Feature requests are welcome, but will probably just be added to the
ever-growing todo list.  They should also be sent to \UNISONUSERS.

Patches are even more welcome.  They should be sent to
\UNISONHACKERS.
(Since safety and robustness are Unison's most important properties,
patches will be held to high standards of clear design and clean coding.)
If you want to contribute to Unison, start by downloading the developer
tarball from the download page.  For some details on how the code is
organized, etc., see the file {\tt CONTRIB}.

\TOPSUBSECTION{Copying}{copying}

This file is part of Unison.

    Unison is free software: you can redistribute it and/or modify
    it under the terms of the GNU General Public License as published by
    the Free Software Foundation, either version 3 of the License, or
    (at your option) any later version.

    Unison is distributed in the hope that it will be useful,
    but WITHOUT ANY WARRANTY; without even the implied warranty of
    MERCHANTABILITY or FITNESS FOR A PARTICULAR PURPOSE.  See the
    GNU General Public License for more details.

    The GNU Public License can be found at
    \ONEURL{http://www.gnu.org/licenses}.  A copy is also included in the
    Unison source distribution in the file {\tt COPYING}.

\TOPSUBSECTION{Acknowledgements}{ack}

Work on Unison has been supported by the National Science Foundation
under grants CCR-9701826 and ITR-0113226, {\em Principles and Practice of
  Synchronization}, and by University of Pennsylvania's Institute for
Research in Cognitive Science (IRCS).

\SECTION{Installation}{install}{install}

Unison is designed to be easy to install.  The following sequence of
steps should get you a fully working installation in a few minutes.  If
you run into trouble, you may find the suggestions on the 
\SHOWURL{http://www.cis.upenn.edu/\home{bcpierce}/unison/faq.html}{Frequently Asked
Questions page} helpful.  Pre-built binaries are available for a
variety of platforms.

Unison can be used with either of two user interfaces: 
\begin{enumerate}
\item a simple textual interface, suitable for dumb terminals (and
running from scripts), and 
\item a more sophisticated grapical interface, based on Gtk2.  
\end{enumerate}

You will need to install a copy of Unison on every machine that you
want to synchronize.  However, you only need the version with a
graphical user interface (if you want a GUI at all) on the machine
where you're actually going to display the interface (the \CLIENT{}
machine).  Other machines that you synchronize with can get along just
fine with the textual version.


\SUBSECTION{Downloading Unison}{download}

The Unison download site lives under 
\ONEURL{http://www.cis.upenn.edu/\home{bcpierce}/unison}.

If a pre-built binary of Unison is available for the client machine's
architecture, just download it and put it somewhere in your search
path (if you're going to invoke it from the command line) or on your
desktop (if you'll be click-starting it).

The executable file for the graphical version (with a name including
\verb|gtkui|) actually provides {\em both} interfaces: the graphical one
appears by default, while the textual interface can be selected by including
\verb|-ui text| on the command line.  The \verb|textui| executable
provides just the textual interface.

If you don't see a pre-built executable for your architecture, you'll
need to build it yourself.  See \sectionref{building}{Building Unison}.
There are also a small number of contributed ports to other
architectures that are not maintained by us.  See the
\SHOWURL{http://www.cis.upenn.edu/\home{bcpierce}/unison/download.html}{Contributed 
Ports page} to check what's available.

Check to make sure that what you have downloaded is really executable.
Either click-start it, or type \showtt{unison -version} at the command
line.  

Unison can be used in three different modes: with different directories on a
single machine, with a remote machine over a direct socket connection, or
with a remote machine using {\tt ssh} for authentication and secure
transfer.  If you intend to use the last option, you may need to install
{\tt ssh}; see \sectionref{ssh}{Installing Ssh}.

\SUBSECTION{Running Unison}{afterinstall} 

Once you've got Unison installed on at least one system, read 
\sectionref{tutorial}{Tutorial} of the user manual (or type \showtt{unison -doc
  tutorial}) for instructions on how to get started.


\SUBSECTION{Upgrading}{upgrading}

Upgrading to a new version of Unison is as simple as throwing away the old
binary and installing the new one.

Before upgrading, it is a good idea to run the {\em old} version one last
time, to make sure all your replicas are completely synchronized.  A new
version of Unison will sometimes introduce a different format for the
archive files used to remember information about the previous state of the
replicas.  In this case, the old archive will be ignored (not deleted --- if
you roll back to the previous version of Unison, you will find the old
archives intact), which means that any differences between the replicas will
show up as conflicts that need to be resolved manually.


\SUBSECTION{Building Unison from Scratch}{building}

If a pre-built image is not available, you will need to compile it from
scratch; the sources are available from the same place as the binaries.

In principle, Unison should work on any platform to which OCaml has been
ported and on which the \verb|Unix| module is fully implemented.  It has
been tested on many flavors of Windows (98, NT, 2000, XP) and Unix (OS X,
Solaris, Linux, FreeBSD), and on both 32- and 64-bit architectures.


\SUBSUBSECTION{Unix}{build-unix}

You'll need the Objective Caml compiler (version 3.07 or later), which is
available from \ONEURL{http://caml.inria.fr}.  Building and installing OCaml
on Unix systems is very straightforward; just follow the instructions in the
distribution.  You'll probably want to build the native-code compiler in
addition to the bytecode compiler, as Unison runs much faster when compiled
to native code, but this is not absolutely necessary.
%
(Quick start: on many systems, the following sequence of commands will
get you a working and installed compiler: first do {\tt make world opt},
then {\tt su} to root and do {\tt make install}.)

You'll also need the GNU {\tt make} utility, standard on many Unix
systems. (Type \showtt{make --version} to check that you've got the
GNU version.)

Once you've got OCaml installed, grab a copy of the Unison sources,
unzip and untar them, change to the new \showtt{unison} directory, and
type ``{\tt make UISTYLE=text}.''
The result should be an executable file called \showtt{unison}.
Type \showtt{./unison} to make sure the program is executable.  You
should get back a usage message.

If you want to build the graphical user interface, you will need to install
two additional things:
\begin{itemize}
\item The Gtk2 libraries.  These areavailable from
  \ONEURL{http://www.gtk.org} and are standard on many Unix installations.   
  
\item The {\tt lablgtk2} OCaml library.  Grab the
  developers' tarball from
  \begin{quote}
  \ONEURL{http://wwwfun.kurims.kyoto-u.ac.jp/soft/olabl/lablgtk.html},
  \end{quote}
  untar it, and follow the instructions to build and install it.

  (Quick start: {\tt make configure}, then {\tt make}, then {\tt make
  opt}, then {\tt su} and {\tt make install}.)
\end{itemize}

Now build unison.  If your search paths are set up correctly, simply typing
{\tt make}
again should build a \verb|unison| executable with a Gtk2 graphical
interface.  (In previous releases of Unison, it was necessary to add {\tt
  UISTYLE=gtk2} to the 'make' command above.  This requirement has been
removed: the makefile should detect automatically when lablgtk2 is
present and set this flag automatically.)  

Put the \verb|unison| executable somewhere in your search path, either
by adding the Unison directory to your PATH variable or by copying the
executable to some standard directory where executables are stored.

\SUBSUBSECTION{Windows}{build-win}

Although the binary distribution should work on any version of Windows,
some people may want to build Unison from scratch on those systems too.

\paragraph{Bytecode version:} The simpler but slower compilation option
to build a Unison executable is to build a bytecode version.  You need
first install Windows version of the OCaml compiler (version 3.07 or
later, available from \ONEURL{http://caml.inria.fr}).  Then grab a copy
of Unison sources and type  
\begin{verbatim}
       make NATIVE=false
\end{verbatim}
to compile the bytecode.  The result should be an executable file called
\verb|unison.exe|. 

\paragraph{Native version:} Building a more efficient, native version of
Unison on Windows requires a little more work.  See the file {\tt
  INSTALL.win32} in the source code distribution.


\SUBSUBSECTION{Installation Options}{build-opts}

The \verb|Makefile| in the distribution includes several switches that
can be used to control how Unison is built.  Here are the most useful
ones:
\begin{itemize}
\item Building with \verb|NATIVE=true| uses the native-code OCaml
compiler, yielding an executable that will run quite a bit faster. We use
this for building distribution versions.
\item Building with \verb|make DEBUGGING=true| generates debugging
symbols. 
\item Building with \verb|make STATIC=true| generates a (mostly)
statically linked executable.  We use this for building distribution
versions, for portability.
\end{itemize}
%\finish{Any other important ones?}


\SECTION{Tutorial}{tutorial}{tutorial}

%\finish{Put a pointer somewhere in here to the typical profile in the
%  reference section.}

\SUBSECTION{Preliminaries}{prelim}

Unison can be used with either of two user interfaces: 
\begin{enumerate}
\item a straightforward textual interface and 
\item a more sophisticated graphical interface
\end{enumerate}
The textual interface is more convenient for running from scripts and
works on dumb terminals; the graphical interface is better for most
interactive use.  For this tutorial, you can use either.  If you are running
Unison from the command line, just typing {\tt unison} 
will select either the text or the graphical interface, depending on which
has been selected as default when the executable you are running was
built.  You can force the text interface even if graphical is the default by
adding {\tt -ui text}.  
The other command-line arguments to both versions are identical.  

The graphical version can also be run directly by clicking on its icon, but
this may require a little set-up (see \sectionref{click}{Click-starting
  Unison}).  For this tutorial, we assume that you're starting it from the
command line.

Unison can synchronize files and directories on a single machine, or
between two machines on a network.  (The same program runs on both
machines; the only difference is which one is responsible for
displaying the user interface.)  If you're only interested in a
single-machine setup, then let's call that machine the \CLIENT{}.  If
you're synchronizing two machines, let's call them \CLIENT{} and
\SERVER.

\SUBSECTION{Local Usage}{local}

Let's get the client machine set up first and see how to synchronize
two directories on a single machine.

Follow the instructions in \sectionref{install}{Installation} to either
download or build an executable version of Unison, and install it
somewhere on your search path.  (If you just want to use the textual user
interface, download the appropriate textui binary.  If you just want to
the graphical interface---or if you will use both interfaces [the gtkui
binary actually has both compiled in]---then download the gtkui binary.)

Create a small test directory {\tt a.tmp} containing a couple of files
and/or subdirectories, e.g.,
\begin{verbatim}
       mkdir a.tmp
       touch a.tmp/a a.tmp/b
       mkdir a.tmp/d
       touch a.tmp/d/f
\end{verbatim}
Copy this directory to b.tmp:
\begin{verbatim}
       cp -r a.tmp b.tmp
\end{verbatim}

Now try synchronizing {\tt a.tmp} and {\tt b.tmp}.  (Since they are
identical, synchronizing them won't propagate any changes, but Unison
will remember the current state of both directories so that it will be
able to tell next time what has changed.)  Type:
\begin{verbatim}
       unison a.tmp b.tmp
\end{verbatim}
(You may need to add \verb|-ui text|, depending how your unison binary was built.)

\begin{textui}
You should see a message notifying you that all the files are actually
equal and then get returned to the command line.
\end{textui}

\begin{tkui}
You should get a big empty window with a message at the bottom
notifying you that all files are identical.  Choose the Exit item from
the File menu to get back to the command line.
\end{tkui}

Next, make some changes in a.tmp and/or b.tmp.  For example:
\begin{verbatim}
        rm a.tmp/a
        echo "Hello" > a.tmp/b
        echo "Hello" > b.tmp/b
        date > b.tmp/c
        echo "Hi there" > a.tmp/d/h
        echo "Hello there" > b.tmp/d/h
\end{verbatim}
Run Unison again:
\begin{verbatim}
       unison a.tmp b.tmp
\end{verbatim}

This time, the user interface will display only the files that have
changed.  If a file has been modified in just one
replica, then it will be displayed with an arrow indicating the
direction that the change needs to be propagated.  For example, 
\begin{verbatim}
                 <---  new file   c  [f]
\end{verbatim}
\noindent
indicates that the file {\tt c} has been modified only in the second
replica, and that the default action is therefore to propagate the new
version to the first replica.  To {\bf f}ollow Unison's recommendation,
press the ``f'' at the prompt.

If both replicas are modified and their contents are different, then
the changes are in conflict: \texttt{<-?->} is displayed to indicate
that Unison needs guidance on which replica should override the
other.  
\begin{verbatim}
     new file  <-?->  new file   d/h  []
\end{verbatim}
By default, neither version will be propagated and both
replicas will remain as they are.  

If both replicas have been modified but their new contents are the same
(as with the file {\tt b}), then no propagation is necessary and
nothing is shown.  Unison simply notes that the file is up to date.

These display conventions are used by both versions of the user
interface.  The only difference lies in the way in which Unison's
default actions are either accepted or overridden by the user.

\begin{textui}
The status of each modified file is displayed, in turn.  
When the copies of a file in the two replicas are not identical, the
user interface will ask for instructions as to how to propagate the
change.  If some default action is indicated (by an arrow), you can
simply press Return to go on to the next changed file.  If you want to
do something different with this file, press ``\verb|<|'' or ``\verb|>|'' to force
the change to be propagated from right to left or from left to right,
or else press ``\verb|/|'' to skip this file and leave both replicas alone.
When it reaches the end of the list of modified files, Unison will ask
you one more time whether it should proceed with the updates that have
been selected.

When Unison stops to wait for input from the user, pressing ``\verb|?|''
will always give a list of possible responses and their meanings.
\end{textui}

\begin{tkui}  
The main window shows all the files that have been modified in either
{\tt a.tmp} or {\tt b.tmp}.  To override a default action (or to select
an action in the case when there is no default), first select the file, either
by clicking on its name or by using the up- and down-arrow keys.  Then
press either the left-arrow or ``\verb|<|'' key (to cause the version in b.tmp to
propagate to a.tmp) or the right-arrow or ``\verb|>|'' key (which makes the a.tmp
version override b.tmp). 

Every keyboard command can also be invoked from the menus at the top
of the user interface.  (Conversely, each menu item is annotated with
its keyboard equivalent, if it has one.)

When you are satisfied with the directions for the propagation of changes
as shown in the main window, click the ``Go'' button to set them in
motion.  A check sign will be displayed next to each filename
when the file has been dealt with.
\end{tkui}


\SUBSECTION{Remote Usage}{remote}

Next, we'll get Unison set up to synchronize replicas on two different
machines.

Follow the instructions in the Installation section to download or
build an executable version of Unison on the server machine, and
install it somewhere on your search path.  (It doesn't matter whether
you install the textual or graphical version, since the copy of Unison on
the server doesn't need to display any user interface at all.)  

It is important that the version of Unison installed on the server
machine is the same as the version of Unison on the client machine.
But some flexibility on the version of Unison at the client side can
be achieved by using the \verb|-addversionno| option; see 
\sectionref{prefs}{Preferences}.

Now there is a decision to be made.  Unison provides two methods for
communicating between the client and the server:
\begin{itemize}
\item {\em Remote shell method}: To use this method, you must have
  some way of invoking remote commands on the server from the client's
  command line, using a facility such as \verb|ssh|.
  This method is more convenient (since there is no need to manually
  start a ``unison server'' process on the server) and also more
  secure (especially if you use \verb|ssh|).

\item {\em Socket method}: This method requires only that you can get
  TCP packets from the client to the server and back.  A draconian 
  firewall can prevent this, but otherwise it should work anywhere.
\end{itemize}

Decide which of these you want to try, and continue with
\sectionref{rshmeth}{Remote Shell Method} or
\sectionref{socketmeth}{Socket Method}, as appropriate.


\SUBSECTION{Remote Shell Method}{rshmeth}

The standard remote shell facility on Unix systems is \verb|ssh|, which provides the
same functionality as the older \verb|rsh| but much better security.  Ssh is available from
\ONEURL{ftp://ftp.cs.hut.fi/pub/ssh/}; up-to-date binaries for some
architectures can also be found at
\ONEURL{ftp://ftp.faqs.org/ssh/contrib}.  See section~\ref{ssh-win}
for installation instructions for the Windows version.

Running
\verb|ssh| requires some coordination between the client and server
machines to establish that the client is allowed to invoke commands on
the server; please refer to the or \verb|ssh| documentation
for information on how to set this up.  The examples in this section
use \verb|ssh|, but you can substitute \verb|rsh| for \verb|ssh| if
you wish.

First, test that we can invoke Unison on the server from the client.
Typing
\begin{alltt}
        ssh \NT{remotehostname} unison -version
\end{alltt}
should print the same version information as running
\begin{verbatim}
        unison -version
\end{verbatim}
locally on the client.  If remote execution fails, then either
something is wrong with your ssh setup (e.g., ``permission denied'')
or else the search path that's being used when executing commands on
the server doesn't contain the \verb|unison| executable (e.g.,
``command not found'').

Create a test directory {\tt a.tmp} in your home directory on the client
machine.  

Test that the local unison client can start and connect to the
remote server.  Type
\begin{alltt}
          unison -testServer a.tmp ssh://\NT{remotehostname}/a.tmp
\end{alltt}

Now cd to your home directory and type:
\begin{verbatim}
          unison a.tmp ssh://remotehostname/a.tmp
\end{verbatim}
The result should be that the entire directory {\tt a.tmp} is propagated
from the client to your home directory on the server.

After finishing the first synchronization, change a few files and try
synchronizing again.  You should see similar results as in the local
case.

If your user name on the server is not the same as on the client, you
need to specify it on the command line:
\begin{verbatim}
          unison a.tmp ssh://username@remotehostname/a.tmp
\end{verbatim}

\noindent {\it Notes:}
\begin{itemize}
\item If you want to put \verb|a.tmp| some place other than your home
directory on the remote host, you can give an absolute path for it by
adding an extra slash between \verb|remotehostname| and the beginning
of the path:
\begin{verbatim}
          unison a.tmp ssh://remotehostname//absolute/path/to/a.tmp
\end{verbatim}

\item You can give an explicit path for the \verb|unison| executable
  on the server by using the command-line option \showtt{-servercmd
    /full/path/name/of/unison} or adding
  \showtt{servercmd=/full/path/name/of/unison} to your profile (see
  \sectionref{profile}{Profile}).  Similarly, you can specify a
  explicit path for the \verb|ssh| program using the \showtt{-sshcmd}
  option.
  Extra arguments can be passed to \verb|ssh| by setting the
  \verb|-sshargs| preference.
\end{itemize}


\SUBSECTION{Socket Method}{socketmeth}

\begin{quote}
  {\bf\ifhevea\red\fi Warning:} The socket method is 
  insecure: not only are the texts of your changes transmitted over
  the network in unprotected form, it is also possible for anyone in
  the world to connect to the server process and read out the contents
  of your filesystem!  (Of course, to do this they must understand the
  protocol that Unison uses to communicate between client and server,
  but all they need for this is a copy of the Unison sources.)  The socket
  method is provided only for expert users with specific needs; everyone
  else should use the \verb|ssh| method.
\end{quote}

To run Unison over a socket connection, you must start a Unison
daemon process on the server.  This process runs continuously,
waiting for connections over a given socket from client machines
running Unison and processing their requests in turn.

To start the daemon, type
\begin{verbatim}
       unison -socket NNNN
\end{verbatim}
on the server machine, where {\tt NNNN} is the socket number that the
daemon should listen on for connections from clients.  ({\tt NNNN} can
be any large number that is not being used by some other program; if
\texttt{NNNN} is already in use, Unison will exit with an error
message.)  Note that paths specified by the client will be interpreted
relative to the directory in which you start the server process; this
behavior is different from the ssh case, where the path is relative to
your home directory on the server.

Create a test directory {\tt a.tmp} in your home directory on the
client machine.  Now type:
\begin{alltt}
       unison a.tmp socket://\NT{remotehostname}:NNNN/a.tmp
\end{alltt}
The result should be that the entire directory {\tt a.tmp} is
propagated from the client to the server (\texttt{a.tmp} will be
created on the server in the directory that the server was started
from).
%
After finishing the first synchronization, change a few files and try
synchronizing again.  You should see similar results as in the local
case.

Since the socket method is not used by many people, its functionality is
rather limited.  For example, the server can only deal with one client at a
time. 


\SUBSECTION{Using Unison for All Your Files}{usingit}

Once you are comfortable with the basic operation of Unison, you may
find yourself wanting to use it regularly to synchronize your commonly
used files.  There are several possible ways of going about this:

\begin{enumerate}
\item Synchronize your whole home directory, using the Ignore facility
(see \sectionref{ignore}{Ignore})
to avoid synchronizing temporary files and things that only belong on
one host.
\item Create a subdirectory called {\tt shared} (or {\tt current}, or
whatever) in your home directory on each host, and put all the files
you want to synchronize into this directory.  
\item Create a subdirectory called {\tt shared} (or {\tt current}, or
whatever) in your home directory on each host, and put {\em links to}
all the files you want to synchronize into this directory.  Use the
{\tt follow} preference (see \sectionref{symlinks}{Symbolic Links}) to make
Unison treat these links as transparent.
\item Make your home directory the root of the synchronization, but
tell Unison to synchronize only some of the files and subdirectories
within it on any given run.  This can be accomplished by using the {\tt -path} switch
on the command line:
\begin{alltt}
       unison /home/\NT{username} ssh://\NT{remotehost}//home/\NT{username} -path shared
\end{alltt}
The {\tt -path} option can be used as many times as needed, to 
synchronize several files or subdirectories:
\begin{alltt}
       unison /home/\NT{username} ssh://\NT{remotehost}//home/\NT{username} \verb|\|
          -path shared \verb|\|
          -path pub \verb|\|
          -path .netscape/bookmarks.html
\end{alltt}
These \verb|-path| arguments can also be put in your preference file.
See \sectionref{prefs}{Preferences} for an example.
\end{enumerate}

Most people find that they only need to maintain a profile (or
profiles) on one of the hosts that they synchronize, since Unison is
always initiated from this host.  (For example, if you're
synchronizing a laptop with a fileserver, you'll probably always run
Unison on the laptop.)  This is a bit different from the usual
situation with asymmetric mirroring programs like \verb|rdist|, where
the mirroring operation typically needs to be initiated from the
machine with the most recent changes.  \sectionref{profile}{Profile}
covers the syntax of Unison profiles, together with some sample profiles.

Some tips on improving Unison's performance can be found on the
\SHOWURL{http://www.cis.upenn.edu/\home{bcpierce}/unison/faq.html}{Frequently
  Asked Questions page}.

\SUBSECTION{Using Unison to Synchronize More Than Two Machines}{usingmultiple}

Unison is designed for synchronizing pairs of replicas.  However, it is
possible to use it to keep larger groups of machines in sync by performing
multiple pairwise synchronizations.  

If you need to do this, the most reliable way to set things up is to
organize the machines into a ``star topology,'' with one machine designated
as the ``hub'' and the rest as ``spokes,'' and with each spoke machine
synchronizing only with the hub.  The big advantage of the star topology is
that it eliminates the possibility of confusing ``spurious conflicts''
arising from the fact that a separate archive is maintained by Unison for
every pair of hosts that it synchronizes.


\SUBSECTION{Going Further}{further}

On-line documentation for the various features of Unison
can be obtained either by typing
\begin{verbatim}
        unison -doc topics
\end{verbatim}
\noindent
at the command line, or by selecting the Help menu in the graphical
user interface.  
\iftextversion
The same information is also available in a typeset User's
Manual (HTML or PostScript format) through
\ONEURL{http://www.cis.upenn.edu/\home{bcpierce}/unison}. 
\else
The on-line information and the printed manual are essentially identical.
\fi

If you use Unison regularly, you should subscribe to one of the mailing
lists, to receive announcements of new versions.  See
\sectionref{lists}{Mailing Lists}. 

\SECTION{Basic Concepts}{basics}{basics}

To understand how Unison works, it is necessary to discuss a few
straightforward concepts.
%
These concepts are developed more rigorously and at more length in a number
of papers, available at \ONEURL{http://www.cis.upenn.edu/\home{bcpierce}/papers}.
But the informal presentation here should be enough for most users.


\SUBSECTION{Roots}{roots}

A replica's {\em root} tells Unison where to find a set of files to be
synchronized, either on the local machine or on a remote host.
For example,
\begin{alltt}
      \NT{relative/path/of/root}
\end{alltt}
\noindent
specifies a local root relative to the directory where Unison is
started, while
\begin{alltt}
      /\NT{absolute/path/of/root}
\end{alltt}
\noindent
specifies a root relative to the top of the local filesystem,
independent of where Unison is running.  Remote roots can begin with
\verb|ssh://|,
\verb|rsh://|
to indicate that the remote server should be started with rsh or ssh:
\begin{alltt}
      ssh://\NT{remotehost}//\NT{absolute/path/of/root}
      rsh://\NT{user}@\NT{remotehost}/\NT{relative/path/of/root}
\end{alltt}
If the remote server is already running (in the socket mode), then the syntax
\begin{alltt}
      socket://\NT{remotehost}:\NT{portnum}//\NT{absolute/path/of/root}
      socket://\NT{remotehost}:\NT{portnum}/\NT{relative/path/of/root}
\end{alltt}
\noindent
is used to specify the hostname and the port that the client Unison should
use to contact it.

The syntax for roots is based on that of URIs (described in RFC 2396).
The full grammar is: 
\begin{alltt}
  \NT{replica} ::= [\NT{protocol}:]//[\NT{user}@][\NT{host}][:\NT{port}][/\NT{path}]
           |  \NT{path}

  \NT{protocol} ::= file
            |  socket
            |  ssh
            |  rsh

  \NT{user} ::= [-_a-zA-Z0-9]+

  \NT{host} ::= [-_a-zA-Z0-9.]+

  \NT{port} ::= [0-9]+
\end{alltt}
When \verb|path| is given without any protocol prefix, the protocol is
assumed to be \verb|file:|.  Under Windows, it is possible to
synchronize with a remote directory using the \verb|file:| protocol over
the Windows Network Neighborhood.  For example,
\begin{verbatim}
       unison foo //host/drive/bar
\end{verbatim}
\noindent
synchronizes the local directory \verb|foo| with the directory
\verb|drive:\bar| on the machine \verb|host|, provided that \verb|host|
is accessible via Network Neighborhood.  When the \verb|file:| protocol
is used in this way, there is no need for a Unison server to be running
on the remote host.  However, running Unison this way is only a good
idea if the remote host is reached by a very fast network connection,
since the full contents of every file in the remote replica will have to
be transferred to the local machine to detect updates.

The names of roots are {\em canonized} by Unison before it uses them
to compute the names of the corresponding archive files, so {\tt
  //saul//home/bcpierce/common} and {\tt //saul.cis.upenn.edu/common}
will be recognized as the same replica under different names.

\SUBSECTION{Paths}{paths}

A {\em path} refers to a point {\em within} a set of files being
synchronized; it is specified relative to the root of the replica.

Formally, a path is just a sequence of names, separated by \verb|/|.
Note that the path separator character is always a forward slash, no
matter what operating system Unison is running on.  Forward slashes
are converted to backslashes as necessary when paths are converted to
filenames in the local filesystem on a particular host.
%
(For example, suppose that we run Unison on a Windows system, synchronizing
the local root \verb|c:\pierce| with the root
\verb|ssh://saul.cis.upenn.edu/home/bcpierce| on a Unix server.  Then
the path \verb|current/todo.txt| refers to the file
\verb|c:\pierce\current\todo.txt| on the client and
\verb|/home/bcpierce/current/todo.txt| on the server.)

The empty path (i.e., the empty sequence of names) denotes the whole
replica.  Unison displays the empty path as ``\verb|[root]|.''

If \verb|p| is a path and \verb|q| is a path beginning with \verb|p|, then
\verb|q| is said to be a {\em descendant} of \verb|p|.  (Each path is also a
descendant of itself.)


\SUBSECTION{What is an Update?}{updates}

The {\em contents} of a path \verb|p| in a particular replica could be a
file, a directory, a symbolic link, or absent (if \verb|p| does not
refer to anything at all in that replica).  More specifically:
\begin{itemize}
\item If \verb|p| refers to an ordinary file, then the
contents of \verb|p| are the actual contents of this file (a string of bytes)
plus the current permission bits of the file.  
\item If \verb|p| refers to a symbolic link, then the contents of \verb|p|
are just the string specifying where the link points.
\item If \verb|p| refers to a directory, then the
contents of \verb|p| are just the token ``DIRECTORY'' plus the current
permission bits of the directory.  
\item If \verb|p| does not refer to anything in this replica, then the
contents of \verb|p| are the token ``ABSENT.''
\end{itemize}
Unison keeps a record of the contents of each path after each
successful synchronization of that path (i.e., it remembers the
contents at the last moment when they were the same in the two
replicas).  

We say that a path is {\em updated} (in some replica) if its current
contents are different from its contents the last time it was successfully
synchronized.  Note that whether a path is updated has nothing to do with
its last modification time---Unison considers only the contents when
determining whether an update has occurred.  This means that touching a file
without changing its contents will {\em not} be recognized as an update.  A
file can even be changed several times and then changed back to its original
contents; as long as Unison is only run at the end of this process, no
update will be recognized.

What Unison actually calculates is a close approximation to this
definition; see \sectionref{caveats}{Caveats and Shortcomings}.

\SUBSECTION{What is a Conflict?}{conflicts}

A path is said to be {\em conflicting} if the following conditions all hold:
\begin{enumerate}
\item it has been updated in one replica, 
\item it or any of its descendants has been updated in the other
  replica, 
and
\item its contents in the two replicas are not identical.
\end{enumerate}

\finishlater{Note that this isn't precisely what we implement, in the
  case of directory permission changes!}


\SUBSECTION{Reconciliation}{recon}

Unison operates in several distinct stages:
\begin{enumerate}
\item On each host, it compares its archive file (which records
the state of each path in the replica when it was last synchronized)
with the current contents of the replica, to determine which paths
have been updated.
\item It checks for ``false conflicts'' --- paths that have been
updated on both replicas, but whose current values are identical.
These paths are silently marked as synchronized in the archive files
in both replicas.
\item It displays all the updated paths to the user.  For updates that
do not conflict, it suggests a default action (propagating the new
contents from the updated replica to the other).  Conflicting updates
are just displayed.  The user is given an opportunity to examine the
current state of affairs, change the default actions for
nonconflicting updates, and choose actions for conflicting updates.
\item It performs the selected actions, one at a time.  Each action is
performed by first transferring the new contents to a temporary file
on the receiving host, then atomically moving them into place.
\item It updates its archive files to reflect the new state of the
replicas. 
\end{enumerate}

\TOPSUBSECTION{Invariants}{failures}

Given the importance and delicacy of the job that it performs, it is
important to understand both what a synchronizer does under normal
conditions and what can happen under unusual conditions such as system
crashes and communication failures.  

% Unison deals with two sorts of information: the two replicas
% themselves and its own memory of the ``last synchronized state'' of
% each path in the replicas.  The latter is what allows it to detect
% correctly which replica is new when a file been updated.  Roughly,
% the sequence of actions that occur when Unison runs is:
% \begin{enumerate}
% \item It reads a private archive file stored with each replica
% and checks which paths on each replica have been updated.
% Technically, a path has been updated if its contents in a replica are
% different from the contents of that replica at the end of the last
% synchronization in which that path was successfully synchronized ---
% i.e., the last time the two replicas were equal at that path at the
% end of a run of Unison.  The ``contents'' of a path can be either a
% file, a directory, or nothing at all, so deleting a file or changing a
% directory to a file count as updates to the contents at that path.

% For efficiency, Unison does not try to calculate the set of updated
% paths exactly: it will sometimes falsely detect a change in a path
% whose contents have actually not changed (this can happen, for
% example, when the file's modification time has been changed, for some
% reason).  As long as this path has not been modified in the other
% replica, this ``conservativity'' in update detection is invisible to
% the user.  If the other replica {\em has} been modified, however, a
% ``false conflict'' may be reported.

% \item It combines the lists of paths that (may) have been updated in
% the two replicas, assigns default actions to those where the change
% was in one replica only, and records a conflict for those that were
% changed in both replicas.

% \item The current contents of the paths on this list are then
% compared, to see if they actually differ.  (This is done by comparing
% fingerprints, not transferring the whole files.)  Paths whose contents
% are actually identical are marked as synchronized and deleted from the
% list. 

% \item The remaining paths are displayed to the user, who then has an
% opportunity to change the default actions and choose actions for
% conflicting paths.

% \item When this process is finished, the selected changes are actually
% propagated between the replicas.

% \item Finally, Unison updates its internal state, marking as
% synchronized all the files for which changes were successfully
% propagated. 
% \end{enumerate}

Unison is careful to protect both its internal state and the state of
the replicas at every point in this process.  Specifically, the
following guarantees are enforced:
\begin{itemize}
\item At every moment, each path in each replica has either (1) its {\em
  original} contents (i.e., no change at all has been made to this
path), or (2) its {\em correct} final contents (i.e., the value that the
user expected to be propagated from the other replica).
\item At every moment, the information stored on disk about Unison's
private state can be either (1) unchanged, or (2) updated to reflect
those paths that have been successfully synchronized.
\end{itemize}
The upshot is that it is safe to interrupt Unison at any time, either
manually or accidentally.  [Caveat: the above is {\em almost} true there
are occasionally brief periods where it is not (and, because of
shortcoming of the Posix filesystem API, cannot be); in particular, when
it is copying a file onto a directory or vice versa, it must first move
the original contents out of the way.  If Unison gets
interrupted during one of these periods, some manual cleanup may be
required.  In this case, a file called {\tt DANGER.README} will be left
in your home directory, containing information about the operation that
was interrupted. The next time you try to run Unison, it will notice this
file and warn you about it.]

If an interruption happens while it is propagating updates, then there
may be some paths for which an update has been propagated but which
have not been marked as synchronized in Unison's archives.  This is no
problem: the next time Unison runs, it will detect changes to these
paths in both replicas, notice that the contents are now equal, and
mark the paths as successfully updated when it writes back its private
state at the end of this run.

If Unison is interrupted, it may sometimes leave temporary working files
(with suffix \verb|.tmp|) in the replicas.  It is safe to delete these
files.  Also, if the \verb|backups| flag is set, Unison will
leave around old versions of files that it overwrites, with names like
\verb|file.0.unison.bak|.  These can be deleted safely when they are no
longer wanted.

Unison is not bothered by clock skew between the different hosts on
which it is running.  It only performs comparisons between timestamps
obtained from the same host, and the only assumption it makes about
them is that the clock on each system always runs forward.

If Unison finds that its archive files have been deleted (or that the
archive format has changed and they cannot be read, or that they don't
exist because this is the first run of Unison on these particular
roots), it takes a conservative approach: it behaves as though the
replicas had both been completely empty at the point of the last
synchronization.  The effect of this is that, on the first run, files
that exist in only one replica will be propagated to the other, while
files that exist in both replicas but are unequal will be marked as
conflicting. 

Touching a file without changing its contents should never affect whether or
not Unison does an update. (When running with the fastcheck preference set
to true---the default on Unix systems---Unison uses file modtimes for a
quick first pass to tell which files have definitely not changed; then, for
each file that might have changed, it computes a fingerprint of the file's
contents and compares it against the last-synchronized contents. Also, the
\verb|-times| option allows you to synchronize file times, but it does not
cause identical files to be changed; Unison will only modify the file
times.)

It is safe to ``brainwash'' Unison by deleting its archive files
{\em on both replicas}.  The next time it runs, it will assume that
all the files it sees in the replicas are new.  

It is safe to modify files while Unison is working.  If Unison
discovers that it has propagated an out-of-date change, or that the
file it is updating has changed on the target replica, it will signal
a failure for that file.  Run Unison again to propagate the latest
change.
\finishlater{There are some race conditions. We should probably talk about them.}

Changes to the ignore patterns from the user interface (e.g., using
the `i' key) are immediately reflected in the current profile.


\SUBSECTION{Caveats and Shortcomings}{caveats}

Here are some things to be careful of when using Unison.  

\begin{itemize}
\item In the interests of speed, the update detection algorithm may
  (depending on which OS architecture that you run Unison on)
  actually use an approximation to the definition given in
  \sectionref{updates}{What is an Update?}.  

  In particular, the Unix
  implementation does not compare the actual contents of files to their
  previous contents, but simply looks at each file's inode number and
  modtime; if neither of these have changed, then it concludes that the
  file has not been changed.

  Under normal circumstances, this approximation is safe, in the sense
  that it may sometimes detect ``false updates'' will never miss a real
  one.  However, it is possible to fool it, for example by using
  \verb|retouch| to change a file's modtime back to a time in the past.
  \finishlater{One user---Marcus Mottl---claimed that it could also
  happen if we use 
  memory mapped I/O, but this is not clear}

\item If you synchronize between a single-user filesystem and a shared
Unix server, you should pay attention to your permission bits: by
default, Unison will synchronize permissions verbatim, which may leave
group-writable files on the server that could be written over by a lot of
people.  

You can control this by setting your \verb|umask| on both computers to
something like 022, masking out the ``world write'' and ``group write''
permission bits.  

Unison does not synchronize the \verb|setuid| and \verb|setgid| bits, for
security. 

\item The graphical user interface is single-threaded.  This
means that if Unison is performing some long-running operation, the
display will not be repainted until it finishes.  We recommend not
trying to do anything with the user interface while Unison is in the
middle of detecting changes or propagating files.

\item Unison does not understand hard links.

\item It is important to be a little careful when renaming directories
containing ``ignore''d files. 

For example, suppose Unison is synchronizing directory A between the two
machines called the ``local'' and the ``remote'' machine; suppose directory
A contains a subdirectory D; and suppose D on the local machine contains a
file or subdirectory P that matches an ignore directive in the profile used
to synchronize. Thus path A/D/P exists on the local machine but not on the
remote machine.
                                                                                
 If D is renamed to D' on the remote machine, and this change is                
 propagated to the local machine, all such files or subdirectories P            
 will be deleted.  This is because Unison sees the rename as a delete and a
 separate create: it deletes the old directory (including the ignored files)
 and creates a new one ({\em not} including the ignored files, since they
 are completely invisible to it).
\end{itemize}



\SECTION{Reference Guide}{reference}{ }

This section covers the features of Unison in detail.  

\TOPSUBSECTION{Running Unison}{running}

There are several ways to start Unison.
\begin{itemize}
\item Typing ``{\tt unison \NT{profile}}'' on the command line.  Unison
will look for a file \texttt{\NT{profile}.prf} in the \verb|.unison|
directory.  If this file does not specify a pair of roots, Unison will
prompt for them and add them to the information specified by the profile.
\item Typing ``{\tt unison \NT{profile} \NT{root1} \NT{root2}}'' on the command
line.
In this case, Unison will use {\tt \NT{profile}}, which should not contain
any {\tt root} directives.
\item Typing ``{\tt unison \NT{root1} \NT{root2}}'' on the command line.  This
has the same effect as typing ``{\tt unison default \NT{root1} \NT{root2}}.''
\item Typing just ``{\tt unison}'' (or invoking Unison by clicking on
a desktop icon).  In this case, Unison will ask for the profile to use
for synchronization (or create a new one, if necessary).   
\end{itemize}

% \finish{Need to check that the text UI actually works this way.  (It 
%   doesn't prompt, for sure, but it should.)}

\SUBSECTION{The {\tt .unison} Directory}{unisondir}

Unison stores a variety of information in a private directory on each
host.  If the environment variable {\tt UNISON} is defined, then its
value will be used as the name of this directory.  If {\tt UNISON} is
not defined, then the name of the directory depends on which
operating system you are using.  In Unix, the default is to use
{\tt \$HOME/.unison}.
In Windows, if the environment variable
{\tt USERPROFILE} is defined, then the directory will be
{\tt \$USERPROFILE$\backslash$.unison};
otherwise if {\tt HOME} is defined, it will be
{\tt \$HOME$\backslash$.unison};
otherwise, it will be
{\tt c:$\backslash$.unison}.

The archive file for each replica is found in the {\tt .unison}
directory on that replica's host.  Profiles (described below) are
always taken from the {\tt .unison} directory on the client host.

Note that Unison maintains a completely different set of archive files
for each pair of roots.

We do not recommend synchronizing the whole {\tt .unison} directory, as this
will involve frequent propagation of large archive files.  It should be safe
to do it, though, if you really want to.  Synchronizing just the profile
files in the {\tt .unison} directory is definitely OK.


\SUBSECTION{Archive Files}{archives}

The name of the archive file on each replica is calculated from 
\begin{itemize}
\item the {\em canonical names} of all the hosts (short names like
  \verb|saul| are converted into full addresses like \verb|saul.cis.upenn.edu|), 
\item the paths to the replicas on all the hosts (again, relative
  pathnames, symbolic links, etc.\ are converted into full, absolute paths), and 
\item an internal version number that is changed whenever a new Unison
  release changes the format of the information stored in the archive.
\end{itemize}
This method should work well for most users.  However, it is occasionally
useful to change the way archive names are generated.  Unison provides
two ways of doing this.

The function that finds the canonical hostname of the local host (which
is used, for example, in calculating the name of the archive file used to
remember which files have been synchronized) normally uses the
\verb|gethostname| operating system call.  However, if the environment
variable \verb|UNISONLOCALHOSTNAME| is set, its value will be used
instead.  This makes it easier to use Unison in situations where a
machine's name changes frequently (e.g., because it is a laptop and gets
moved around a lot).

A more powerful way of changing archive names is provided by the
\verb|rootalias| preference.  The preference file may contain any number of
lines of the form: 
\begin{alltt}
    rootalias = //\NT{hostnameA}//\NT{path-to-replicaA} -> //\NT{hostnameB}/\NT{path-to-replicaB}
\end{alltt}
When calculating the name of the archive files for a given pair of roots,
Unison replaces any root that matches the left-hand side of any rootalias
rule by the corresponding right-hand side.

So, if you need to relocate a root on one of the hosts, you can add a
rule of the form:
\begin{alltt}
    rootalias = //\NT{new-hostname}//\NT{new-path} -> //\NT{old-hostname}/\NT{old-path}
\end{alltt}
Note that root aliases are case-sensitive, even on case-insensitive file
systems.

{\em Warning}: The \verb|rootalias| option is dangerous and should only
be used if you are sure you know what you're doing.  In particular, it
should only be used if you are positive that either (1) both the original
root and the new alias refer to the same set of files, or (2) the files
have been relocated so that the original name is now invalid and will
never be used again.  (If the original root and the alias refer to
different sets of files, Unison's update detector could get confused.)
%
After introducing a new \verb|rootalias|, it is a good idea to run Unison
a few times interactively (with the \verb|batch| flag off, etc.) and
carefully check that things look reasonable---in particular, that update
detection is working as expected.


\SUBSECTION{Preferences}{prefs}

Many details of Unison's behavior are configurable by user-settable
``preferences.''  

Some preferences are boolean-valued; these are often called {\em flags}.
Others take numeric or string arguments, indicated in the preferences
list by {\tt n} or {\tt xxx}.  Most of the string preferences can be
given several times; the arguments are accumulated into a list
internally.

There are two ways to set the values of preferences: temporarily, by
providing command-line arguments to a particular run of Unison, or
permanently, by adding commands to a {\em profile} in the {\tt .unison}
directory on the client host.  The order of preferences (either on the
command line or in preference files) is not significant.  On the command
line, preferences and other arguments (the profile name and roots) can be
intermixed in any order.

To set the value of a preference {\tt p} from the command line, add an
argument {\tt -p} (for a boolean flag) or {\tt -p n} or {\tt -p xxx} (for
a numeric or string preference) anywhere on the command line.  To set a
boolean flag to \verb|false| on the command line, use {\tt -p=false}.

Here are all the preferences supported by Unison.  This list can be
  obtained by typing {\tt unison -help}.
\begin{quote}
\verbatiminput{prefs.tmp} 
\end{quote}
Here, in more detail, is what they do.  Many are discussed in greater detail
in other sections of the manual.
%
\input{prefsdocs.tmp} 


\SUBSECTION{Profiles}{profile}

A {\em profile} is a text file that specifies permanent settings for
roots, paths, ignore patterns, and other preferences, so that they do
not need to be typed at the command line every time Unison is run.
Profiles should reside in the \verb|.unison| directory on the client
machine.  If Unison is started with just one argument \ARG{name} on
the command line, it looks for a profile called \texttt{\ARG{name}.prf} in
the \verb|.unison| directory.  If it is started with no arguments, it
scans the \verb|.unison| directory for files whose names end in
\verb|.prf| and offers a menu (provided that the Unison executable is compiled with the graphical user interface).  If a file named \verb|default.prf| is
found, its settings will be offered as the default choices.

To set the value of a preference {\tt p} permanently, add to the
appropriate profile a line of the form
\begin{verbatim}
        p = true
\end{verbatim}
for a boolean flag or
\begin{verbatim}
        p = <value>
\end{verbatim}
for a preference of any other type.  

Whitespaces around {\tt p} and {\tt xxx} are ignored.
A profile may also include blank lines and lines beginning
with {\tt \#}; both are ignored.

When Unison starts, it first reads the profile and then the command
line, so command-line options will override settings from the
profile.  

Profiles may also include lines of the form \texttt{include
  \ARG{name}}, which will cause the file \ARG{name} (or
\texttt{\ARG{name}.prf}, if \ARG{name} does not exist in the
\verb+.unison+ directory) to be read at the point, and included as if
its contents, instead of the \texttt{include} line, was part of the
profile.  Include lines allows settings common to several profiles to
be stored in one place.

A profile may include a preference `\texttt{label = \ARG{desc}}' to
provide a description of the options selected in this profile.  The
string \ARG{desc} is listed along with the profile name in the profile
selection dialog, and displayed in the top-right corner of the main
Unison window in the graphical user interface.

The graphical user-interface also supports one-key shortcuts for commonly
used profiles.  If a profile contains a preference of the form 
%
`\texttt{key = \ARG{n}}', where \ARG{n} is a single digit, then
pressing this digit key will cause Unison to immediately switch to
this profile and begin synchronization again from scratch.  In this
case, all actions that have been selected for a set of changes
currently being displayed will be discarded.


\SUBSECTION{Sample Profiles}{profileegs}

\SUBSUBSECTION{A Minimal Profile}{minimalprofile}

Here is a very minimal profile file, such as might be found in {\tt
  .unison/default.prf}:
\begin{verbatim}
    # Roots of the synchronization
    root = /home/bcpierce
    root = ssh://saul//home/bcpierce

    # Paths to synchronize 
    path = current
    path = common
    path = .netscape/bookmarks.html
\end{verbatim}

\SUBSUBSECTION{A Basic Profile}{basicprofile}

Here is a more sophisticated profile, illustrating some other useful
features. 
\begin{verbatim}
    # Roots of the synchronization
    root = /home/bcpierce
    root = ssh://saul//home/bcpierce

    # Paths to synchronize 
    path = current
    path = common
    path = .netscape/bookmarks.html

    # Some regexps specifying names and paths to ignore
    ignore = Name temp.*
    ignore = Name *~
    ignore = Name .*~
    ignore = Path */pilot/backup/Archive_*
    ignore = Name *.o
    ignore = Name *.tmp

    # Window height
    height = 37

    # Keep a backup copy of every file in a central location
    backuplocation = central
    backupdir = /home/bcpierce/backups
    backup = Name *
    backupprefix = $VERSION.
    backupsuffix = 

    # Use this command for displaying diffs
    diff = diff -y -W 79 --suppress-common-lines

    # Log actions to the terminal
    log = true
\end{verbatim}

\SUBSUBSECTION{A Power-User Profile}{powerprofile}

When Unison is used with large replicas, it is often convenient to be
able to synchronize just a part of the replicas on a given run (this
saves the time of detecting updates in the other parts).  This can be
accomplished by splitting up the profile into several parts --- a common
part containing most of the preference settings, plus one ``top-level''
file for each set of paths that need to be synchronized.  (The {\tt
  include} mechanism can also be used to allow the same set of preference
settings to be used with different roots.)

The collection
of profiles implementing this scheme might look as follows.
%
The file {\tt default.prf} is empty except for an {\tt include}
directive:
\begin{verbatim}
    # Include the contents of the file common
    include common
\end{verbatim}
Note that the name of the common file is {\tt common}, not {\tt
  common.prf}; this prevents Unison from offering {\tt common} as one of
the list of profiles in the opening dialog (in the graphical UI).

The file {\tt common} contains the real preferences:
\begin{verbatim}
    # Roots of the synchronization
    root = /home/bcpierce
    root = ssh://saul//home/bcpierce

    # (... other preferences ...)

    # If any new preferences are added by Unison (e.g. 'ignore'
    # preferences added via the graphical UI), then store them in the
    # file 'common' rathen than in the top-level preference file
    addprefsto = common

    # Names and paths to ignore:
    ignore = Name temp.*
    ignore = Name *~
    ignore = Name .*~
    ignore = Path */pilot/backup/Archive_*
    ignore = Name *.o
    ignore = Name *.tmp
\end{verbatim}
Note that there are no {\tt path} preferences in {\tt common}.  This
means that, when we invoke Unison with the default profile (e.g., by
typing '{\tt unison default}' or just '{\tt unison}' on the command
line), the whole replicas will be synchronized.  (If we {\em never} want
to synchronize the whole replicas, then {\tt default.prf} would instead
include settings for all the paths that are usually synchronized.)

To synchronize just part of the replicas, Unison is invoked with an
alternate preference file---e.g., doing '{\tt unison workingset}', where the
preference file {\tt workingset.prf} contains
\begin{verbatim}
    path = current/papers
    path = Mail/inbox
    path = Mail/drafts
    include common
\end{verbatim}
causes Unison to synchronize just the listed subdirectories.

The {\tt key} preference can be used in combination with the graphical UI
to quickly switch between different sets of paths.  For example, if the
file {\tt mail.prf} contains
\begin{verbatim}
    path = Mail
    batch = true
    key = 2
    include common
\end{verbatim}
then pressing 2 will cause Unison to look for updates in the {\tt Mail}
subdirectory and (because the {\tt batch} flag is set) immediately
propagate any that it finds.


\SUBSECTION{Keeping Backups}{backups}

When Unison overwrites a file or directory by propagating a new version from
the other replica, it can keep the old version around as a backup.  There
are several preferences that control precisely where these backups are
stored and how they are named.

To enable backups, you must give one or more \verb|backup| preferences.
Each of these has the form
\begin{verbatim}
    backup = <pathspec>
\end{verbatim}
where \verb|<pathspec>| has the same form as for the \verb|ignore|
preference.  For example, 
\begin{verbatim}
    backup = Name *
\end{verbatim}
causes Unison to keep backups of {\em all} files and directories.  The
\verb|backupnot| preference can be used to give a few exceptions: it
specifies which files and directories should {\em not} be backed up, even if
they match the \verb|backup| pathspec. 

It is important to note that the \verb|pathspec| is matched against the path
that is being updated by Unison, not its descendants.  For example, if you
set \verb|backup = Name *.txt| and then delete a whole directory named
\verb|foo| containing some text files, these files will not be backed up
because Unison will just check that \verb|foo| does not match \verb|*.txt|.
Similarly, if the directory itself happened to be called \verb|foo.txt|,
then the whole directory and all the files in it will be backed up,
regardless of their names. 

Backup files can be stored either {\em centrally} or {\em locally}.  This
behavior is controlled by the preference \verb|backuplocation|, whose value
must be either \verb|central| or \verb|local|.  (The default is
\verb|central|.)  

When backups are stored locally, they are kept in the same
directory as the original.

When backups are stored centrally, the directory used to hold them is
controlled by the preference \verb|backupdir| and the
environment variable \verb|UNISONBACKUPDIR|.  (The environment variable is
checked first.)  If neither of these are set, then the directory
\verb|.unison/backup| in the user's home directory is used.

The preference \verb|maxbackups| controls how many previous versions of
each file are kept (including the current version).  

By default, backup files are named \verb|.bak.VERSION.FILENAME|,
where \verb|FILENAME| is the original filename and \verb|VERSION| is the
backup number (1 for the most recent, 2 for the next most recent,
etc.).  This can be changed by setting the preferences \verb|backupprefix|
and/or \verb|backupsuffix|.  If desired, \verb|backupprefix| may include a
directory prefix; this can be used with \verb|backuplocation = local| to put all
backup files for each directory into a single subdirectory.  For example, setting
\begin{verbatim}
    backuplocation = local
    backupprefix = .unison/$VERSION.
    backupsuffix = 
\end{verbatim}
will put all backups in a local subdirectory named \verb|.unison|.  Also,
note that the string \verb|$VERSION| in either \verb|backupprefix| or
\verb|backupsuffix| (it must appear in one or the other) is replaced by
the version number.  This can be used, for example, to ensure that backup
files retain the same extension as the originals.

For backward compatibility, the \verb|backups| preference is also supported.
%
It simply means \verb|backup = Name *| and \verb|backuplocation = local|.


\SUBSECTION{Merging Conflicting Versions}{merge}

Unison can invoke external programs to merge conflicting versions of a file.
The preference \verb|merge| controls this process.  

The \verb|merge| preference may be given once or several times in a
preference file (it can also be given on the command line, of course, but
this tends to be awkward because of the spaces and special characters
involved).  Each instance of the preference looks like this:
\begin{verbatim}
    merge = <PATHSPEC> -> <MERGECMD>
\end{verbatim}
The \verb|<PATHSPEC>| here has exactly the same format as for the
\verb|ignore| preference (see \sectionref{pathspec}{Path specification}).  For example,
using ``\verb|Name *.txt|'' as the \verb|<PATHSPEC>| tells Unison that this
command should be used whenever a file with extension \verb|.txt| needs to
be merged.  

Many external merging programs require as inputs not just the two files that
need to be merged, but also a file containing the {\em last synchronized
  version}.  You can ask Unison to keep a copy of the last synchronized
version for some files using the \verb|backupcurrent| preference. This
preference is used in exactly the same way as \verb|backup| and its meaning
is similar, except that it causes backups to be kept of the {\em current}
contents of each file after it has been synchronized by Unison, rather than
the {\em previous} contents that Unison overwrote.  These backups are kept
on {\em both} replicas in the same place as ordinary backup files---i.e.
according to the \verb|backuplocation| and \verb|backupdir| preferences.
They are named like the original files if \verb|backupslocation| is set to
'central' and otherwise, Unison uses the \verb|backupprefix| and
\verb|backupsuffix| preferences and assumes a version number 000 for these
backups.

The \verb|<MERGECMD>| part of the preference specifies what external command
should be invoked to merge files at paths matching the \verb|<PATHSPEC>|.
Within this string, several special substrings are recognized; these will be
substituted with appropriate values before invoking a sub-shell to execute
the command.  
\begin{itemize}
\item \relax\verb|CURRENT1| is replaced by the name of (a temporary copy of)
  the local variant of the file.
\item \relax\verb|CURRENT2| is replaced by the name of a temporary
  file, into which the contents of the remote variant of the file have
  been transferred by Unison prior to performing the merge.
\item \relax\verb|CURRENTARCH| is replaced by the name of the backed up copy
  of the original version of the file (i.e., the file saved by Unison
  if the current filename matches the path specifications for the
  \verb|backupcurrent| preference, as explained above), if one exists.
  If no archive exists and \relax\verb|CURRENTARCH| appears in the
  merge command, then an error is signalled. 
\item \relax\verb|CURRENTARCHOPT| is replaced by the name of the backed up copy
  of the original version of the file (i.e., its state at the end of
  the last successful run of Unison), if one exists, or the empty
  string if no archive exists.
\item \relax\verb|NEW| is replaced by the name of a temporary file
  that Unison expects to be written by the merge program when it
  finishes, giving the desired new contents of the file.
\item \relax\verb|PATH| is replaced by the path (relative to the roots of
  the replicas) of the file being merged.
\item \relax\verb|NEW1| and \relax\verb|NEW2| are replaced by the names of temporary files
  that Unison expects to be written by the merge program when it
  is only able to partially merge the originals; in this case, \verb|NEW1|
  will be written back to the local replica and \verb|NEW2| to the remote
  replica; \verb|NEWARCH|, if present, will be used as the ``last common
  state'' of the replicas.  (These three options are provided for
  later compatibility with the Harmony data synchronizer.)
\end{itemize}
To accomodate the wide variety of programs that users might want to use for
merging, Unison checks for several possible situations when the merge
program exits:
\begin{itemize}
\item If the merge program exits with a non-zero status, then merge is
  considered to have failed and the replicas are not changed.
\item If the file \verb|NEW| has been created, it is written back to both
  replicas (and stored in the backup directory).  Similarly, if just the
  file \verb|NEW1| has been created, it is written back to both 
  replicas.
\item If neither \verb|NEW| nor \verb|NEW1| have been created, then Unison
  examines the temporary files \verb|CURRENT1|  and \verb|CURRENT2| that
  were given as inputs to the merge program.  If either has been changed (or
  both have been changed in identical ways), then its new contents are written
  back to both replicas.  If either \verb|CURRENT1| or \verb|CURRENT2| has
  been {\em deleted}, then the contents of the other are written back to
  both replicas.
\item If the files \verb|NEW1|, \verb|NEW2|, and \verb|NEWARCH| have all
  been created, they are written back to the local replica, remote replica,
  and backup directory, respectively. If the files \verb|NEW1|, \verb|NEW2| have 
  been created, but \verb|NEWARCH| has not, then these files are written back to the
  local replica and remote replica, respectively.  Also, if \verb|NEW1| and
  \verb|NEW2| have identical contents, then the same contents are stored as
  a backup (if the \verb|backupcurrent| preference is set for this path) to
  reflect the fact that the path is currently in sync. 
  \item If \verb|NEW1| and \verb|NEW2| (resp. \verb|CURRENT1| and
  \verb|CURRENT2|) are created (resp. overwritten) with different contents
  but the merge command did not fail (i.e., it exited with status code 0),
  then we copy \verb|NEW1| (resp. \verb|CURRENT1|) to the other replica and
  to the archive.  
  
  This behavior is a design choice made to handle the case where a merge
  command only synchronizes some specific contents between two files,
  skipping some irrelevant information (order between entries, for
  instance).  We assume that, if the merge command exits normally, then the
  two resulting files are ``as good as equal.'' (The reason we copy one on
  top of the other is to avoid Unison detecting that the files are unequal
  the next time it is run and trying again to merge them when, in fact, the
  merge program has already made them as similar as it is able to.)
\end{itemize}

If the \verb|confirmmerge| preference is set and Unison is not run in
batch mode, then Unison will always ask for confirmation before
actually committing the results of the merge to the replicas.

A large number of external merging programs are available.  
For example, on Unix systems setting the \verb|merge| preference to
\begin{verbatim}
    merge = Name *.txt -> diff3 -m CURRENT1 CURRENTARCH CURRENT2
                            > NEW || echo "differences detected"
\end{verbatim}
\noindent
will tell Unison to use the external \verb|diff3| program for merging.  
%
Alternatively, users of \verb|emacs| may find the following settings convenient:
\begin{verbatim}
    merge = Name *.txt -> emacs -q --eval '(ediff-merge-files-with-ancestor 
                             "CURRENT1" "CURRENT2" "CURRENTARCH" nil "NEW")' 
\end{verbatim}
\noindent
(These commands are displayed here on two lines to avoid running off the
edge of the page.  In your preference file, each command should be written on a
single line.) 

Users running emacs under windows may find something like this useful:
\begin{verbatim}
   merge = Name * -> C:\Progra~1\Emacs\emacs\bin\emacs.exe -q --eval
                            "(ediff-files """CURRENT1""" """CURRENT2""")"
\end{verbatim}

Users running Mac OS X (you may need the Developer Tools installed to get
the {\tt opendiff} utility) may prefer
\begin{verbatim}
    merge = Name *.txt -> opendiff CURRENT1 CURRENT2 -ancestor CURRENTARCH -merge NEW
\end{verbatim}
Here is a slightly more involved hack.  The {\tt opendiff} program can
operate either with or without an archive file.  A merge command of this
form 
\begin{verbatim}
    merge = Name *.txt -> 
              if [ CURRENTARCHOPTx = x ]; 
              then opendiff CURRENT1 CURRENT2 -merge NEW; 
              else opendiff CURRENT1 CURRENT2 -ancestor CURRENTARCHOPT -merge NEW; 
              fi
\end{verbatim}
(still all on one line in the preference file!) will test whether an archive
file exists and use the appropriate variant of the arguments to {\tt
  opendiff}. 

Ordinarily, external merge programs are only invoked when Unison is {\em
  not} running in batch mode.  To specify an external merge program that
should be used no matter the setting of the {\tt batch} flag, use the {\tt
  mergebatch} preference instead of {\tt merge}.

\begin{quote}
\it
Please post suggestions for other useful values of the
\verb|merge| preference to the {\tt unison-users} mailing list---we'd like
to give several examples here.
\end{quote}

\finishlater{
\SUBSECTION{Communicating with a Remote Server}{server}

If you can mount both filesystems on the same host, then you can
run with no server (note, though, that this won't be fast enough over
a phone line)..........
}

\SUBSECTION{The User Interface}{ui}

Both the textual and the graphical user interfaces are intended to be
mostly self-explanatory.  Here are just a few tricks:
\begin{itemize}
\item By default, when running on Unix the textual user interface will
try to put the terminal into the ``raw mode'' so that it reads the input a
character at a time rather than a line at a time.  (This means you can
type just the single keystroke ``\verb|>|'' to tell Unison to
propagate a file from left to right, rather than ``\verb|>| Enter.'')

There are some situations, though, where this will not work --- for
example, when Unison is running in a shell window inside Emacs.
Setting the \verb|dumbtty| preference will force Unison to leave the
terminal alone and process input a line at a time.
\end{itemize}

\SUBSECTION{Exit code}{exit}

When running in the textual mode, Unison returns an exit status, which
describes whether, and at which level, the synchronization was successful.
The exit status could be useful when Unison is invoked from a script.
Currently, there are four possible values for the exit status:
\begin{itemize}
\item [0]: successful synchronization; everything is up-to-date now.
\item [1]: some files were skipped, but all file transfers were successful.
\item [2]: non-fatal failures occurred during file transfer.
\item [3]: a fatal error occurred, or the execution was interrupted.
\end{itemize}
The graphical interface does not return any useful information through the
exit status.

\SUBSECTION{Path specification}{pathspec}
Several Unison preferences (e.g., \verb|ignore|/\verb|ignorenot|,
\verb|follow|, \verb|sortfirst|/\verb|sortlast|, \verb|backup|,
\verb|merge|, etc.)
specify individual paths or sets of paths.  These preferences share a
common syntax based on regular-expressions.  Each preference
is associated with a list of path patterns; the paths specified are those
that match any one of the path pattern.

\begin{itemize}
\item Pattern preferences can be given on the command line,
  or, more often, stored in profiles, using the same syntax as other preferences.  
  For example, a profile line of the form
\begin{alltt}
             ignore = \ARG{pattern}
\end{alltt}
adds \ARG{pattern} to the list of patterns to be ignored.

\item Each \ARG{pattern} can have one of three forms.  The most
general form is a Posix extended regular expression introduced by the
keyword \verb|Regex|.  (The collating sequences and character classes of
full Posix regexps are not currently supported).
\begin{alltt}
                 Regex \ARG{regexp}
\end{alltt}
For convenience, two other styles of pattern are also recognized:
\begin{alltt}
                 Name \ARG{name}
\end{alltt}
matches any path in which the last component matches \ARG{name}, while
\begin{alltt}
                 Path \ARG{path}
\end{alltt}
matches exactly the path \ARG{path}.
%
The \ARG{name} and \ARG{path} arguments of the latter forms of
patterns are {\em not} regular expressions.  Instead, 
standard ``globbing'' conventions can be used in \ARG{name} and
\ARG{path}:  
\begin{itemize}
\item a \verb|*| matches any sequence of characters not including \verb|/|
(and not beginning with \verb|.|, when used at the beginning of a
\ARG{name})
\item a \verb|?| matches any single character except \verb|/| (and leading
  \verb|.|) 
\item \verb|[xyz]| matches any character from the set $\{{\tt x},
  {\tt y}, {\tt z} \}$
\item \verb|{a,bb,ccc}| matches any one of \verb|a|, \verb|bb|, or
  \verb|ccc|. 
\end{itemize}
\item 
The path separator in path patterns is always the
forward-slash character ``/'' --- even when the client or server is
running under Windows, where the normal separator character is a
backslash.  This makes it possible to use the same set of path
patterns for both Unix and Windows file systems.  
\end{itemize}
Some examples of path patterns appear in \sectionref{ignore}{Ignoring
  Paths}.

\SUBSECTION{Ignoring Paths}{ignore}

Most users of Unison will find that their replicas contain lots of
files that they don't ever want to synchronize --- temporary files,
very large files, old stuff, architecture-specific binaries, etc.
They can instruct Unison to ignore these paths using patterns
introduced in \sectionref{pathspec}{Path Patterns}.

For example, the following pattern will make Unison ignore any
path containing the name \verb|CVS| or a name ending in \verb|.cmo|:
\begin{verbatim}
             ignore = Name {CVS,*.cmo}
\end{verbatim}
The next pattern makes Unison ignore the path \verb|a/b|:
\begin{verbatim}
             ignore = Path a/b
\end{verbatim}
Path patterns do {\em not} skip filesnames beginning with \verb|.| (as Name
patterns do).  For example,
\begin{verbatim}
             ignore = Path */tmp
\end{verbatim}
will include \verb|.foo/tmp| in the set of ignore directories, as it is a
path, not a name, that is ignored.

The following pattern makes Unison ignore any path beginning with \verb|a/b|
and ending with a name ending by \verb|.ml|.
\begin{verbatim}
             ignore = Regex a/b/.*\.ml
\end{verbatim}
Note that regular expression patterns are ``anchored'': they must
match the whole path, not just a substring of the path.

Here are a few extra points regarding the \texttt{ignore} preference.
\begin{itemize}
\item If a directory is ignored, all its descendents will be too.
  
\item The user interface provides some convenient commands for adding
  new patterns to be ignored.  To ignore a particular file, select it
  and press ``{\tt i}''.  To ignore all files with the same extension,
  select it and press ``{\tt E}'' (with the shift key).  To ignore all
  files with the same name, no matter what directory they appear in,
  select it and press ``{\tt N}''.
%
These new patterns become permanent: they
are immediately added to the current profile on disk.

\item If you use the \verb|include| directive to include a common
collection of preferences in several top-level preference files, you will
probably also want to set the \verb|addprefsto| preference to the name of
this file.  This will cause any new ignore patterns that you add from
inside Unison to be appended to this file, instead of whichever top-level
preference file you started Unison with.  

\item Ignore patterns can also be specified on the command line, if
you like (this is probably not very useful), using an option like
\verb|-ignore 'Name temp.txt'|.

\item Be careful about renaming directories containing ignored files.
Because Unison understands the rename as a delete plus a create, any ignored
files in the directory will be lost (since they are invisible to Unison and
therefore they do not get recreated in the new version of the directory).

\item There is also an \verb|ignorenot| preference, which specifies a set of
  patterns for paths that should {\em not} be ignored, even if they match an
  \verb|ignore| pattern.  However, the interaction of these two sets of
  patterns can be a little tricky.  Here is exactly how it works:
  \begin{itemize}
  \item Unison starts detecting updates from the root of the
  replicas---i.e., from the empty path.  If the empty path matches an
  \verb|ignore| pattern and does not match an \verb|ignorenot| pattern, then
  the whole replica will be ignored.  (For this reason, it is not a good
  idea to include \verb|Name *| as an \verb|ignore| pattern.  If you want to
  ignore everything except a certain set of files, use \verb|Name ?*|.)
  \item If the root is a directory, Unison continues looking for updates in
  all the immediate children of the root.  Again, if the name of some child matches an
  \verb|ignore| pattern and does not match an \verb|ignorenot| pattern, then
  this whole path {\em including everything below it} will be ignored.  
  \item If any of the non-ignored children are directories, then the process
  continues recursively.
  \end{itemize}
\end{itemize} 

\SUBSECTION{Symbolic Links}{symlinks}

Ordinarily, Unison treats symbolic links in Unix replicas as
``opaque'': it considers the contents of the link to be just the
string specifying where the link points, and it will propagate changes in
this string to the other replica.

It is sometimes useful to treat a symbolic link ``transparently,''
acting as though whatever it points to were physically {\em in} the
replica at the point where the symbolic link appears.  To tell Unison
to treat a link in this manner, add a line of the form
\begin{alltt}
             follow = \ARG{pathspec}
\end{alltt}
to the profile, where \ARG{pathspec} is a path pattern as described in
\sectionref{pathspec}{Path Patterns}.

Windows file systems do not support symbolic links; Unison will refuse
to propagate an opaque symbolic link from Unix to Windows and flag the
path as erroneous.  When a Unix replica is to be synchronized with a
Windows system, all symbolic links should match either an
\verb|ignore| pattern or a \verb|follow| pattern.


\SUBSECTION{Permissions}{perms}

Synchronizing the permission bits of files is slightly tricky when two
different filesytems are involved (e.g., when synchronizing a Windows
client and a Unix server).  In detail, here's how it works:
\begin{itemize}
\item When the permission bits of an existing file or directory are
changed, the values of those bits that make sense on {\em both}
operating systems will be propagated to the other replica.  The other
bits will not be changed.  
\item When a newly created file is propagated to a remote replica, the
permission bits that make sense in both operating systems are also
propagated.  The values of the other bits are set to default values
(they are taken from the current umask, if the receiving host is a
Unix system).
\item For security reasons, the Unix \verb|setuid| and \verb|setgid|
bits are not propagated.  
\item The Unix owner and group ids are not propagated.  (What would
this mean, in general?)  All files are created with the owner and
group of the server process.
\end{itemize}


\SUBSECTION{Cross-Platform Synchronization}{crossplatform}

If you use Unison to synchronize files between Windows and Unix
systems, there are a few special issues to be aware of.

\textbf{Case conflicts.}  In Unix, filenames are case sensitive:
\texttt{foo} and \texttt{FOO} can refer to different files.  In
Windows, on the other hand, filenames are not case sensitive:
\texttt{foo} and \texttt{FOO} can only refer to the same file.  This
means that a Unix \texttt{foo} and \texttt{FOO} cannot be synchronized
onto a Windows system --- Windows won't allow two different files to
have the ``same'' name.  Unison detects this situation for you, and
reports that it cannot synchronize the files.  

You can deal with a case conflict in a couple of ways.  If you need to
have both files on the Windows system, your only choice is to rename
one of the Unix files to avoid the case conflict, and re-synchronize.
If you don't need the files on the Windows system, you can simply
disregard Unison's warning message, and go ahead with the
synchronization; Unison won't touch those files.  If you don't want to
see the warning on each synchronization, you can tell Unison to ignore
the files (see \sectionref{ignore}{Ignore}).

\textbf{Illegal filenames.}  Unix allows some filenames that are
illegal in Windows.  For example, colons (`:') are not allowed in
Windows filenames, but they are legal in Unix filenames.  This means
that a Unix file \texttt{foo:bar} can't be synchronized to a Windows
system.  As with case conflicts, Unison detects this situation for
you, and you have the same options: you can either rename the Unix
file and re-synchronize, or you can ignore it.


\SUBSECTION{Slow Links}{speed}

Unison is built to run well even over relatively slow links such as
modems and DSL connections.  

Unison uses the ``rsync protocol'' designed by Andrew Tridgell and Paul
Mackerras to greatly speed up transfers of large files in which only
small changes have been made.  More information about the rsync protocol
can be found at the rsync web site (\ONEURL{http://samba.anu.edu.au/rsync/}).

If you are using Unison with {\tt ssh}, you may get some speed
improvement by enabling {\tt ssh}'s compression feature.  Do this by
adding the option ``{\tt -rshargs -C}'' to the command line or ``{\tt
  rshargs = -C}'' to your profile.  


\SUBSECTION{Making Unison Faster on Large Files}{speeding}

Unison's built-in implementation of the rsync algorithm makes transferring
updates to existing files pretty fast.  However, for whole-file copies of
newly created files, the built-in transfer method is not highly optimized.
Also, if Unison is interrupted in the middle of transferring a large file,
it will attempt to retransfer the whole thing on the next run.

These shortcomings can be addressed with a little extra work by telling
Unison to use an external file copying utility for whole-file transfers.
The recommended one is the standalone {\tt rsync} tool, which is available
by default on most Unix systems and can easily be installed on Windows
systems using Cygwin.

If you have {\tt rsync} installed on both hosts, you can make Unison use it
simply by setting the {\tt copythreshold} flag to something non-negative.
If you set it to 0, Unison will use the external copy utility for {\em all}
whole-file transfers.  (This is probably slower than letting Unison copy
small files by itself, but can be useful for testing.)  If you set it to a
larger value, Unison will use the external utility for all files larger than
this size (which is given in kilobytes, so setting it to 1000 will cause the
external tool to be used for all transfers larger than a megabyte).

If you want to use a different external copy utility, set both the {\tt
  copyprog} and {\tt copyprogpartial} preferences---the former is used for
the first transfer of a file, while the latter is used when Unison sees a
partially transferred temp file on the receiving host.  Be careful here:
Your external tool needs to be instructed to copy files in place (otherwise
if the transfer is interrupted Unison will not notice that some of the data
has already been transferred, the next time it tries).  The default values
are: 
\begin{verbatim}
   copyprog      =   rsync --inplace --compress
   copyprogrest  =   rsync --partial --inplace --compress
\end{verbatim}
You may also need to set the {\tt copyquoterem} preference.  When it is set
to {\tt true}, this causes Unison to add an extra layer of quotes to
the remote path passed to the external copy program. This is is needed by
rsync, for example, which internally uses an ssh connection, requiring an
extra level of quoting for paths containing spaces. When this flag is set to
{\tt default}, extra quotes are added if the value of {\tt copyprog}
contains the string {\tt rsync}.  The default value is {\tt default},
naturally.

If a {\em directory} transfer is interrupted, the next run of Unison will
automatically skip any files that were completely transferred before the
interruption.  (This behavior is always on: it does not depend on the
setting of the {\tt copythreshold} preference.)  Note, though, that the new
directory will not appear in the destination filesystem until everything has
been transferred---partially transferred directories are kept in a temporary
location (with names like {\tt .unison.DIRNAME....}) until the transfer is
complete.


\SUBSECTION{Fast Update Detection}{fastcheck}

If your replicas are large and at least one of them is on a Windows
system, you may find that Unison's default method for detecting changes
(which involves scanning the full contents of every file on every
sync---the only completely safe way to do it under Windows) is too slow.
Unison provides a preference {\tt fastcheck} that, when set to
\verb|true|, causes it to use file creation times as 'pseudo inode
numbers' when scanning replicas for updates, instead of reading the full
contents of every file.  

When \verb|fastcheck| is set to \verb|no|,
Unison will perform slow checking---re-scanning the contents of each file
on each synchronization---on all replicas.  When \verb|fastcheck| is set
to \verb|default| (which, naturally, is the default), Unison will use
fast checks on Unix replicas and slow checks on Windows replicas.

This strategy may cause Unison to miss propagating an update if the
 modification time and length of the file are both unchanged
by the update.
However, Unison will never {\em overwrite} such an update with a change
from the other replica, since it always does a safe check for updates
just before propagating a change.  Thus, it is reasonable to use this
switch most of the time and occasionally run Unison once with {\tt
  fastcheck} set to \verb|no|, if you are worried that Unison may have
overlooked an update.

Fastcheck is (always) automatically disabled for files with extension
\verb|.xls| or \verb|.mpp|, to prevent Unison from being confused by the 
habits of certain programs (Excel, in particular) of updating files without
changing their modification times.

\SUBSECTION{Mount Points and Removable Media}{mountpoints}

Using Unison removable media such as USB drives can be dangerous unless you
are careful.  If you synchronize a directory that is stored on removable
media when the media is not present, it will look to Unison as though the
whole directory has been deleted, and it will proceed to delete the
directory from the other replica---probably not what you want!

To prevent accidents, Unison provides a preference called
\verb|mountpoint|.  Including a line like
\begin{verbatim}
             mountpoint = foo
\end{verbatim}
in your preference file will cause Unison to check, after it finishes
detecting updates, that something actually exists at the path
\verb|foo| on both replicas; if it does not, the Unison run will
abort. 

\SUBSECTION{Click-starting Unison}{click}

On Windows NT/2k/XP systems, the graphical version of Unison can be
invoked directly by clicking on its icon.  On Windows 95/98 systems,
click-starting also works, {\em as long as you are not using ssh}.
Due to an incompatibility with ocaml and Windows 95/98 that is not
under our control, you must start Unison from a DOS window in Windows
95/98 if you want to use ssh.

When you click on the Unison icon, two windows will be created:
Unison's regular window, plus a console window, which is used only for
giving your password to ssh (if you do not use ssh to connect, you can
ignore this window).  When your password is requested, you'll need to
activate the console window (e.g., by clicking in it) before typing.
If you start Unison from a DOS window, Unison's regular window will
appear and you will type your password in the DOS window you were
using.

To use Unison in this mode, you must first create a profile (see
\sectionref{profile}{Profile}).  Use your favorite editor for this.  


\appendix
\SECTION{Installing Ssh}{ssh}{ssh}

{\em Warning: These instructions may be out of date.  More current
  information can be found the
  \SHOWURL{http://alliance.seas.upenn.edu/~bcpierce/wiki/index.php?n=Main.UnisonFAQOSSpecific}{Unison
    Wiki}.}

Your local host will need just an ssh client; the remote host needs an
ssh server (or daemon), which is available on Unix systems.  Unison is
known to work with ssh version 1.2.27 (Unix) and version 1.2.14
(Windows); other versions may or may not work.

\SUBSECTION{Unix}{ssh-unix}

Most modern Unix installations come with \verb|ssh| pre-installed.

\SUBSECTION{Windows}{ssh-win}
Many Windows implementations of ssh only provide graphical interfaces,
but Unison requires an ssh client that it can invoke with a
command-line interface.  A suitable version of ssh can be installed as
follows.  

\begin{enumerate}
\item Download an \verb|ssh| executable.  
  
Warning: there are many implementations and ports of ssh for
Windows, and not all of them will work with Unison.  We have gotten
Unison to work with Cygwin's port of openssh, and we suggest you try
that one first.  Here's how to install it:
\begin{enumerate}
\item First, create a new folder on your desktop to hold temporary
  installation files.  It can have any name you like, but in these
  instructions we'll assume that you call it \verb|Foo|.
\item Direct your web browser to www.cygwin.com, and click on the
  ``Install now!'' link.  This will download a file, \verb|setup.exe|;
  save it in the directory \verb|Foo|.  The file \verb|setup.exe| is a
  small program that will download the actual install files from
  the Internet when you run it.
\item Start \verb|setup.exe| (by double-clicking).  This brings up a
  series of dialogs that you will have to go through.  Select
  ``Install from Internet.''  For ``Local Package Directory'' select
  the directory \verb|Foo|.  For ``Select install root directory'' we
  recommend that you use the default, \verb|C:\cygwin|.  The next
  dialog asks you to select the way that you want to connect to the
  network to download the installation files; we have used ``Use IE5
  Settings'' successfully, but you may need to make a different
  selection depending on your networking setup.  The next dialog gives
  a list of mirrors; select one close to you.
  
  Next you are asked to select which packages to install.  The default
  settings in this dialog download a lot of packages that are not
  strictly necessary to run Unison with ssh.  If you don't want to
  install a package, click on it until ``skip'' is shown.  For a
  minimum installation, select only the packages ``cygwin'' and
  ``openssh,'' which come to about 1900KB; the full installation is
  much larger.  

  \begin{quote} \em Note that you are plan to build unison using the free
    CygWin GNU C compiler, you need to install essential development
    packages such as ``gcc'', ``make'', ``fileutil'', etc; we refer to
    the file ``INSTALL.win32-cygwin-gnuc'' in the source distribution
    for further details.
  \end{quote}

  After the packages are downloaded and installed, the next dialog
  allows you to choose whether to ``Create Desktop Icon'' and ``Add to
  Start Menu.''  You make the call.
\item You can now delete the directory \verb|Foo| and its contents.
\end{enumerate}
Some people have reported problems using Cygwin's ssh with Unison.  If
you have trouble, you might try this one instead:
\begin{verbatim}
  http://opensores.thebunker.net/pub/mirrors/ssh/contrib/ssh-1.2.14-win32bin.zip
\end{verbatim}

\item You must set the environment variables HOME and PATH\@.
  Ssh will create a directory \verb|.ssh| in the directory given
  by HOME, so that it has a place to keep data like your public and
  private keys.  PATH must be set to include the Cygwin \verb|bin|
  directory, so that Unison can find the ssh executable.
  \begin{itemize}
  \item 
    On Windows 95/98, add the lines
\begin{verbatim}
    set PATH=%PATH%;<SSHDIR>
    set HOME=<HOMEDIR>
\end{verbatim}
    to the file \verb|C:\AUTOEXEC.BAT|, where \verb|<HOMEDIR>| is the
    directory where you want ssh to create its \verb|.ssh| directory,
    and \verb|<SSHDIR>| is the directory where the executable
    \verb|ssh.exe| is stored; if you've installed Cygwin in the
    default location, this is \verb|C:\cygwin\bin|.  You will have to
    reboot your computer to take the changes into account.
  \item On Windows NT/2k/XP, open the environment variables dialog box:
    \begin{itemize}
    \item Windows NT: My Computer/Properties/Environment
    \item Windows 2k: My Computer/Properties/Advanced/Environment
      variables
    \end{itemize}
    then select Path and edit its value by appending \verb|;<SSHDIR>|
    to it, where \verb|<SSHDIR>| is the full name of the directory 
    that includes the ssh executable; if you've installed Cygwin in
    the default location, this is \verb|C:\cygwin\bin|.
  \end{itemize}
  \item Test ssh from a DOS shell by typing
\begin{verbatim}
      ssh <remote host> -l <login name>
\end{verbatim}
    You should get a prompt for your password on \verb|<remote host>|,
    followed by a working connection.
  \item Note that \verb|ssh-keygen| may not work (fails with
  ``gethostname: no such file or directory'') on some systems.  This is
  OK: you can use ssh with your regular password for the remote
  system. 
\item You should now be able to use Unison with an ssh connection. If
  you are logged in with a different user name on the local and remote
  hosts, provide your remote user name when providing the remote root
  (i.e., \verb|//username@host/path...|).
\end{enumerate}

\SECTION{Changes in Version \unisonversion}{news}{news}

\begin{changesfromversion}{2.13.0}
\item The features for performing backups and for invoking external merge
programs have been completely rewritten by Stephane Lescuyer (thanks,
Stephane!).  The user-visible functionality should not change, but the
internals have been rationalized and there are a number of new features.
See the manual (in particular, the description of the \verb|backupXXX|
preferences) for details.

\item Incorporated patches for ipv6 support, contributed by Samuel Thibault.
(Note that, due to a bug in the released OCaml 3.08.3 compiler, this code
will not actually work with ipv6 unless compiled with the CVS version of the
OCaml compiler, where the bug has been fixed; however, ipv4 should continue
to work normally.)

\item OSX interface:
\begin{itemize}
\item Incorporated Ben Willmore's cool new icon for the Mac UI.
\end{itemize}

\item Small fixes:
\begin{itemize}
\item Fixed off by one error in month numbers (in printed dates) reported 
  by Bob Burger
\end{itemize}

\end{changesfromversion}

\begin{changesfromversion}{2.12.0}
\item New convention for release numbering: Releases will continue to be
given numbers of the form \verb|X.Y.Z|, but, 
from now on, just the major version number (\verb|X.Y|) will be considered
significant when checking compatibility between client and server versions.
The third component of the version number will be used only to identify
``patch levels'' of releases.

This change goes hand in hand with a change to the procedure for making new
releases.  Candidate releases will initially be given ``beta release''
status when they are announced for public consumption.  Any bugs that are
discovered will be fixed in a separate branch of the source repository
(without changing the major version number) and new tarballs re-released as
needed.  When this process converges, the patched beta version will be
dubbed stable.

\item Warning (failure in batch mode) when one path is completely emptied.
  This prevents Unison from deleting everything on one replica when
  the other disappear.

\item Fix diff bug (where no difference is shown the first time the diff
  command is given).

\item User interface changes:
\begin{itemize}
\item Improved workaround for button focus problem (GTK2 UI)
\item Put leading zeroes in date fields
\item More robust handling of character encodings in GTK2 UI
\item Changed format of modification time displays, from \verb|modified at hh:mm:ss on dd MMM, yyyy|
to \verb|modified on yyyy-mm-dd hh:mm:ss|
\item Changed time display to include seconds (so that people on FAT
  filesystems will not be confused when Unison tries to update a file
  time to an odd number of seconds and the filesystem truncates it to
  an even number!)
\item Use the diff "-u" option by default when showing differences between files
  (the output is more readable)
\item In text mode, pipe the diff output to a pager if the environment
  variable PAGER is set
\item Bug fixes and cleanups in ssh password prompting.  Now works with
  the GTK2 UI under Linux.  (Hopefully the Mac OS X one is not broken!)
\item Include profile name in the GTK2 window name
\item Added bindings ',' (same as '<') and '.' (same as '>') in the GTK2 UI
\end{itemize}

\item Mac GUI:
\begin{itemize}
\item actions like < and > scroll to the next item as necessary.
\item Restart has a menu item and keyboard shortcut (command-R).
\item 
    Added a command-line tool for Mac OS X.  It can be installed from
    the Unison menu.
\item New icon.
\item   Handle the "help" command-line argument properly.
\item   Handle profiles given on the command line properly.
\item  When a profile has been selected, the profile dialog is replaced by a
    "connecting" message while the connection is being made.  This
    gives better feedback.
\item   Size of left and right columns is now large enough so that
    "PropsChanged" is not cut off.
\end{itemize}


\item Minor changes:
\begin{itemize}
\item Disable multi-threading when both roots are local
\item Improved error handling code.  In particular, make sure all files
  are closed in case of a transient failure
\item Under Windows, use \verb|$UNISON| for home directory as a last resort
  (it was wrongly moved before \verb|$HOME| and \verb|$USERPROFILE| in
  Unison 2.12.0)
\item Reopen the logfile if its name changes (profile change)
\item Double-check that permissions and modification times have been
  properly set: there are some combination of OS and filesystem on
  which setting them can fail in a silent way.
\item Check for bad Windows filenames for pure Windows synchronization
  also (not just cross architecture synchronization).
  This way, filenames containing backslashes, which are not correctly
  handled by unison, are rejected right away.
\item Attempt to resolve issues with synchronizing modification times
  of read-only files under Windows
\item Ignore chmod failures when deleting files
\item Ignore trailing dots in filenames in case insensitive mode
\item Proper quoting of paths, files and extensions ignored using the UI
\item The strings CURRENT1 and CURRENT2 are now correctly substitued when
  they occur in the diff preference
\item Improvements to syncing resource forks between Macs via a non-Mac system.
\end{itemize}

\end{changesfromversion}

\begin{changesfromversion}{2.10.2}
\item \incompatible{} Archive format has changed.  

\item Source code availability: The Unison sources are now managed using
  Subversion.  One nice side-effect is that anonymous checkout is now
  possible, like this:
\begin{verbatim}
        svn co https://cvs.cis.upenn.edu:3690/svnroot/unison/
\end{verbatim}
We will also continue to export a ``developer tarball'' of the current
(modulo one day) sources in the web export directory.  To receive commit logs
for changes to the sources, subscribe to the \verb|unison-hackers| list
(\ONEURL{http://www.cis.upenn.edu/~bcpierce/unison/lists.html}). 

\item Text user interface:
\begin{itemize}
\item Substantial reworking of the internal logic of the text UI to make it
a bit easier to modify.
\item The {\tt dumbtty} flag in the text UI is automatically set to true if
the client is running on a Unix system and the {\tt EMACS} environment
variable is set to anything other than the empty string.
\end{itemize}

\item Native OS X gui:
\begin{itemize}
\item Added a synchronize menu item with keyboard shortcut
\item Added a merge menu item, still needs to be debugged
\item Fixes to compile for Panther
\item Miscellaneous improvements and bugfixes
\end{itemize}

\item Small changes:
\begin{itemize}
\item Changed the filename checking code to apply to Windows only, instead
  of OS X as well.
\item Finder flags now synchronized
\item Fallback in copy.ml for filesystem that do not support \verb|O_EXCL|
\item  Changed buffer size for local file copy (was highly inefficient with
  synchronous writes)
\item Ignore chmod failure when deleting a directory
\item  Fixed assertion failure when resolving a conflict content change /
  permission changes in favor of the content change.
\item Workaround for transferring large files using rsync.
\item Use buffered I/O for files (this is the only way to open files in binary
  mode under Cygwin).
\item On non-Cygwin Windows systems, the UNISON environment variable is now checked first to determine 
  where to look for Unison's archive and preference files, followed by \verb|HOME| and 
  \verb|USERPROFILE| in that order.  On Unix and Cygwin systems, \verb|HOME| is used.
\item Generalized \verb|diff| preference so that it can be given either as just 
  the command name to be used for calculating diffs or else a whole command
  line, containing the strings \verb|CURRENT1| and \verb|CURRENT2|, which will be replaced
  by the names of the files to be diff'ed before the command is called.
\item Recognize password prompts in some newer versions of ssh.
\end{itemize}
\end{changesfromversion}

\begin{changesfromversion}{2.9.20}
\item \incompatible{} Archive format has changed.  
\item Major functionality changes:
\begin{itemize}
\item Major tidying and enhancement of 'merge' functionality.  The main
  user-visible change is that the external merge program may either write
  the merged output to a single new file, as before, or it may modify one or
  both of its input files, or it may write {\em two} new files.  In the
  latter cases, its modifications will be copied back into place on both the
  local and the remote host, and (if the two files are now equal) the
  archive will be updated appropriately.  More information can be found in
  the user manual.  Thanks to Malo Denielou and Alan Schmitt for these
  improvements.

  Warning: the new merging functionality is not completely compatible with
  old versions!  Check the manual for details.
  
\item Files larger than 2Gb are now supported.

\item Added preliminary (and still somewhat experimental) support for the
  Apple OS X operating system.   
\begin{itemize}
\item Resource forks should be transferred correctly.  (See the manual for
details of how this works when synchronizing HFS with non-HFS volumes.)
Synchronization of file type and creator information is also supported.
\item On OSX systems, the name of the directory for storing Unison's
archives, preference files, etc., is now determined as follows:
\begin{itemize}
    \item if \verb+~/.unison+ exists, use it
     \item otherwise, use \verb|~/Library/Application Support/Unison|, 
         creating it if necessary.
\end{itemize}
\item A preliminary native-Cocoa user interface is under construction.  This
still needs some work, and some users experience unpredictable crashes, so
it is only for hackers for now.  Run make with {\tt UISTYLE=mac} to build
this interface.
\end{itemize}
\end{itemize}

\item Minor functionality changes:
\begin{itemize}

\item Added an {\tt ignorelocks} preference, which forces Unison to override left-over
  archive locks.  (Setting this preference is dangerous!  Use it only if you
  are positive you know what you are doing.) 
\item Running with the {\tt -timers} flag set to true will now show the total time taken
  to check for updates on each directory.  (This can be helpful for tidying directories to improve
  update detection times.)
\item Added a new preference {\tt assumeContentsAreImmutable}.  If a directory
  matches one of the patterns set in this preference, then update detection
  is skipped for files in this directory.  (The 
  purpose is to speed update detection for cases like Mail folders, which
  contain lots and lots of immutable files.)  Also a preference
  {\tt assumeContentsAreImmutableNot}, which overrides the first, similarly
  to {\tt ignorenot}.  (Later amendment: these preferences are now called
  {\tt immutable} and {\tt immutablenot}.)

\item The {\tt ignorecase} flag has been changed from a boolean to a three-valued
  preference.  The default setting, called {\tt default}, checks the operating systems
  running on the client and server and ignores filename case if either of them is
  OSX or Windows.  Setting ignorecase to {\tt true} or {\tt false} overrides
  this behavior.  If you have been setting {\tt ignorecase} on the command
  line using {\tt -ignorecase=true} or {\tt -ignorecase=false}, you will
  need to change to {\tt -ignorecase true} or {\tt -ignorecase false}.

\item a new preference, 'repeat', for the text user interface (only).  If 'repeat' is set to
  a number, then, after it finishes synchronizing, Unison will wait for that many seconds and
  then start over, continuing this way until it is killed from outside.  Setting repeat to true
  will automatically set the batch preference to true.  
  
\item Excel files are now handled specially, so that the {\tt fastcheck}
  optimization is skipped even if the {\tt fastcheck} flag is set.  (Excel
  does some naughty things with modtimes, making this optimization
  unreliable and leading to failures during change propagation.)

\item The ignorecase flag has been changed from a boolean to a three-valued
  preference.  The default setting, called 'default', checks the operating systems
  running on the client and server and ignores filename case if either of them is
  OSX or Windows.  Setting ignorecase to 'true' or 'false' overrides this behavior.
  
\item Added a new preference, 'repeat', for the text user interface (only,
  at the moment).  If 'repeat' is set to a number, then, after it finishes
  synchronizing, Unison will wait for that many seconds and then start over,
  continuing this way until it is killed from outside.  Setting repeat to
  true will automatically set the batch preference to true.

\item The 'rshargs' preference has been split into 'rshargs' and 'sshargs' 
  (mainly to make the documentation clearer).  In fact, 'rshargs' is no longer
  mentioned in the documentation at all, since pretty much everybody uses
  ssh now anyway.
\end{itemize}

\item Documentation
\begin{itemize}
\item The web pages have been completely redesigned and reorganized.
  (Thanks to Alan Schmitt for help with this.)
\end{itemize}

\item User interface improvements
\begin{itemize}
\item Added a GTK2 user interface, capable (among other things) of displaying filenames
  in any locale encoding.  Kudos to Stephen Tse for contributing this code!  
\item The text UI now prints a list of failed and skipped transfers at the end of
  synchronization. 
\item Restarting update detection from the graphical UI will reload the current
  profile (which in particular will reset the -path preference, in case
  it has been narrowed by using the ``Recheck unsynchronized items''
  command).
\item Several small improvements to the text user interface, including a
  progress display.
\end{itemize}

\item Bug fixes (too numerous to count, actually, but here are some):
\begin{itemize}
\item The {\tt maxthreads} preference works now.
\item Fixed bug where warning message about uname returning an unrecognized
  result was preventing connection to server.  (The warning is no longer
  printed, and all systems where 'uname' returns anything other than 'Darwin' 
  are assumed not to be running OS X.)
\item Fixed a problem on OS X that caused some valid file names (e.g.,
  those including colons) to be considered invalid.
\item Patched Path.followLink to follow links under cygwin in addition to Unix
  (suggested by Matt Swift).
\item Small change to the storeRootsName function, suggested by bliviero at 
  ichips.intel.com, to fix a problem in unison with the `rootalias'
  option, which allows you to tell unison that two roots contain the same 
  files.  Rootalias was being applied after the hosts were 
  sorted, so it wouldn't work properly in all cases.
\item Incorporated a fix by Dmitry Bely for setting utimes of read-only files
  on Win32 systems.   
\end{itemize}

\item Installation / portability:
\begin{itemize}
\item Unison now compiles with OCaml version 3.07 and later out of the box.
\item Makefile.OCaml fixed to compile out of the box under OpenBSD.
\item a few additional ports (e.g. OpenBSD, Zaurus/IPAQ) are now mentioned in 
  the documentation 
\item Unison can now be installed easily on OSX systems using the Fink
  package manager
\end{itemize}
\end{changesfromversion}

\begin{changesfromversion}{2.9.1}
\item Added a preference {\tt maxthreads} that can be used to limit the
number of simultaneous file transfers.
\item Added a {\tt backupdir} preference, which controls where backup
files are stored.
\item Basic support added for OSX.  In particular, Unison now recognizes
when one of the hosts being synchronized is running OSX and switches to
a case-insensitive treatment of filenames (i.e., 'foo' and 'FOO' are
considered to be the same file).
  (OSX is not yet fully working,
  however: in particular, files with resource forks will not be
  synchronized correctly.)
\item The same hash used to form the archive name is now also added to
the names of the temp files created during file transfer.  The reason for
this is that, during update detection, we are going to silently delete
any old temp files that we find along the way, and we want to prevent
ourselves from deleting temp files belonging to other instances of Unison
that may be running in parallel, e.g. synchronizing with a different
host.  Thanks to Ruslan Ermilov for this suggestion.
\item Several small user interface improvements
\item Documentation
\begin{itemize}
\item FAQ and bug reporting instructions have been split out as separate
      HTML pages, accessible directly from the unison web page.
\item Additions to FAQ, in particular suggestions about performance
tuning. 
\end{itemize}
\item Makefile
\begin{itemize}
\item Makefile.OCaml now sets UISTYLE=text or UISTYLE=gtk automatically,
  depending on whether it finds lablgtk installed
\item Unison should now compile ``out of the box'' under OSX
\end{itemize}
\end{changesfromversion}

\begin{changesfromversion}{2.8.1}
\item Changing profile works again under Windows
\item File movement optimization: Unison now tries to use local copy instead of
  transfer for moved or copied files.  It is controled by a boolean option
  ``xferbycopying''.
\item Network statistics window (transfer rate, amount of data transferred).
      [NB: not available in Windows-Cygwin version.]
\item symlinks work under the cygwin version (which is dynamically linked).
\item Fixed potential deadlock when synchronizing between Windows and
Unix 
\item Small improvements:
  \begin{itemize} 
  \item If neither the {\\tt USERPROFILE} nor the {\\tt HOME} environment
    variables are set, then Unison will put its temporary commit log
    (called {\\tt DANGER.README}) into the directory named by the 
    {\\tt UNISON} environment variable, if any; otherwise it will use
    {\\tt C:}.
  \item alternative set of values for fastcheck: yes = true; no = false;
  default = auto.
  \item -silent implies -contactquietly
  \end{itemize}
\item Source code:
  \begin{itemize}
  \item Code reorganization and tidying.  (Started breaking up some of the
    basic utility modules so that the non-unison-specific stuff can be
    made available for other projects.)
  \item several Makefile and docs changes (for release);
  \item further comments in ``update.ml'';
  \item connection information is not stored in global variables anymore.
  \end{itemize}
\end{changesfromversion}

\begin{changesfromversion}{2.7.78}
\item Small bugfix to textual user interface under Unix (to avoid leaving
  the terminal in a bad state where it would not echo inputs after Unison
  exited).
\end{changesfromversion}

\begin{changesfromversion}{2.7.39}
\item Improvements to the main web page (stable and beta version docs are
  now both accessible).
\item User manual revised.
\item Added some new preferences:
\begin{itemize}
\item ``sshcmd'' and ``rshcmd'' for specifying paths to ssh and rsh programs.
\item ``contactquietly'' for suppressing the ``contacting server'' message
during Unison startup (under the graphical UI).
\end{itemize}
\item Bug fixes:
\begin{itemize}
\item Fixed small bug in UI that neglected to change the displayed column 
  headers if loading a new profile caused the roots to change.
\item Fixed a bug that would put the text UI into an infinite loop if it
  encountered a conflict when run in batch mode.
\item Added some code to try to fix the display of non-Ascii characters in 
  filenames on Windows systems in the GTK UI.  (This code is currently 
  untested---if you're one of the people that had reported problems with
  display of non-ascii filenames, we'd appreciate knowing if this actually 
  fixes things.)
\item `\verb|-prefer/-force newer|' works properly now.  
        (The bug was reported by Sebastian Urbaniak and Sean Fulton.)
\end{itemize}
\item User interface and Unison behavior:
\begin{itemize}
\item Renamed `Proceed' to `Go' in the graphical UI.
\item Added exit status for the textual user interface.
\item Paths that are not synchronized because of conflicts or errors during 
  update detection are now noted in the log file.
\item \verb|[END]| messages in log now use a briefer format
\item Changed the text UI startup sequence so that
  {\\tt ./unison -ui text} will use the default profile instead of failing.
\item Made some improvements to the error messages.
\item Added some debugging messages to remote.ml.
\end{itemize}
\end{changesfromversion}

\begin{changesfromversion}{2.7.7}
\item Incorporated, once again, a multi-threaded transport sub-system.
  It transfers several files at the same time, thereby making much
  more effective use of available network bandwidth.  Unlike the
  earlier attempt, this time we do not rely on the native thread
  library of OCaml.  Instead, we implement a light-weight,
  non-preemptive multi-thread library in OCaml directly.  This version
  appears stable.  

  Some adjustments to unison are made to accommodate the multi-threaded
  version.  These include, in particular, changes to the
  user interface and logging, for example:
  \begin{itemize}
  \item Two log entries for each transferring task, one for the
    beginning, one for the end.
  \item Suppressed warning messages against removing temp files left
    by a previous unison run, because warning does not work nicely
    under multi-threading.  The temp file names are made less likely
    to coincide with the name of a file created by the user.  They
    take the form \\ \verb|.#<filename>.<serial>.unison.tmp|.
  \end{itemize}
\item Added a new command to the GTK user interface: pressing 'f' causes
  Unison to start a new update detection phase, using as paths {\em just}
  those paths that have been detected as changed and not yet marked as
  successfully completed.  Use this command to quickly restart Unison on
  just the set of paths still needing attention after a previous run.
\item Made the {\tt ignorecase} preference user-visible, and changed the
  initialization code so that it can be manually set to true, even if
  neither host is running Windows.  (This may be useful, e.g., when using 
  Unison running on a Unix system with a FAT volume mounted.)
\item Small improvements and bug fixes:
  \begin{itemize}
  \item Errors in preference files now generate fatal errors rather than
    warnings at startup time.  (I.e., you can't go on from them.)  Also,
    we fixed a bug that was preventing these warnings from appearing in the
    text UI, so some users who have been running (unsuspectingly) with 
    garbage in their prefs files may now get error reports.
  \item Error reporting for preference files now provides file name and
    line number.
  \item More intelligible message in the case of identical change to the same 
    files: ``Nothing to do: replicas have been changed only in identical 
    ways since last sync.''
  \item Files with prefix '.\#' excluded when scanning for preference
    files.
  \item Rsync instructions are send directly instead of first
    marshaled.
  \item Won't try forever to get the fingerprint of a continuously changing file:
    unison will give up after certain number of retries.
  \item Other bug fixes, including the one reported by Peter Selinger
    (\verb|force=older preference| not working).
  \end{itemize}
\item Compilation:
  \begin{itemize}
  \item Upgraded to the new OCaml 3.04 compiler, with the LablGtk
    1.2.3 library (patched version used for compiling under Windows).
  \item Added the option to compile unison on the Windows platform with
    Cygwin GNU C compiler.  This option only supports building
    dynamically linked unison executables.
  \end{itemize}
\end{changesfromversion}

\begin{changesfromversion}{2.7.4}
\item Fixed a silly (but debilitating) bug in the client startup sequence.
\end{changesfromversion}

\begin{changesfromversion}{2.7.1}
\item Added \verb|addprefsto| preference, which (when set) controls which
preference file new preferences (e.g. new ignore patterns) are added to.
\item Bug fix: read the initial connection header one byte at a time, so
that we don't block if the header is shorter than expected.  (This bug
did not affect normal operation --- it just made it hard to tell when you
were trying to use Unison incorrectly with an old version of the server,
since it would hang instead of giving an error message.)
\end{changesfromversion}

\begin{changesfromversion}{2.6.59}
\item Changed \verb|fastcheck| from a boolean to a string preference.  Its 
  legal values are \verb|yes| (for a fast check), \verb|no| (for a safe 
  check), or \verb|default| (for a fast check---which also happens to be 
  safe---when running on Unix and a safe check when on Windows).  The default 
  is \verb|default|.
  \item Several preferences have been renamed for consistency.  All
  preference names are now spelled out in lowercase.  For backward
  compatibility, the old names still work, but they are not mentioned in
  the manual any more.
\item The temp files created by the 'diff' and 'merge' commands are now
   named by {\em pre}pending a new prefix to the file name, rather than
   appending a suffix.  This should avoid confusing diff/merge programs
   that depend on the suffix  to guess the type of the file contents.
\item We now set the keepalive option on the server socket, to make sure
  that the server times out if the communication link is unexpectedly broken. 
\item Bug fixes:
\begin{itemize}
\item When updating small files, Unison now closes the destination file.
\item File permissions are properly updated when the file is behind a
  followed link.
\item Several other small fixes.
\end{itemize}
\end{changesfromversion}


\begin{changesfromversion}{2.6.38}
\item Major Windows performance improvement!  

We've added a preference \verb|fastcheck| that makes Unison look only at
a file's creation time and last-modified time to check whether it has
changed.  This should result in a huge speedup when checking for updates
in large replicas.

  When this switch is set, Unison will use file creation times as 
  'pseudo inode numbers' when scanning Windows replicas for updates, 
  instead of reading the full contents of every file.  This may cause 
  Unison to miss propagating an update if the create time, 
  modification time, and length of the file are all unchanged by 
  the update (this is not easy to achieve, but it can be done).  
  However, Unison will never {\em overwrite} such an update with
  a change from the other replica, since it 
  always does a safe check for updates just before propagating a 
  change.  Thus, it is reasonable to use this switch most of the time 
  and occasionally run Unison once with {\tt fastcheck} set to false, 
  if you are worried that Unison may have overlooked an update.

  Warning: This change is has not yet been thoroughly field-tested.  If you 
  set the \verb|fastcheck| preference, pay careful attention to what
  Unison is doing.

\item New functionality: centralized backups and merging 
\begin{itemize}
\item This version incorporates two pieces of major new functionality,
   implemented by Sylvain Roy during a summer internship at Penn: a
   {\em centralized backup} facility that keeps a full backup of
   (selected files 
   in) each replica, and a {\em merging} feature that allows Unison to
   invoke an external file-merging tool to resolve conflicting changes to
   individual files.
 
\item Centralized backups:
\begin{itemize}
  \item Unison now maintains full backups of the last-synchronized versions
      of (some of) the files in each replica; these function both as
      backups in the usual sense
      and as the ``common version'' when invoking external
      merge programs.
  \item The backed up files are stored in a directory ~/.unison/backup on each
      host.  (The name of this directory can be changed by setting
      the environment variable \verb|UNISONBACKUPDIR|.)
  \item The predicate \verb|backup| controls which files are actually
     backed up:
      giving the preference '\verb|backup = Path *|' causes backing up
      of all files.
  \item Files are added to the backup directory whenever unison updates
      its archive.  This means that
      \begin{itemize}
       \item When unison reconstructs its archive from scratch (e.g., 
           because of an upgrade, or because the archive files have
           been manually deleted), all files will be backed up.
       \item Otherwise, each file will be backed up the first time unison
           propagates an update for it.
      \end{itemize}
  \item The preference \verb|backupversions| controls how many previous
      versions of each file are kept.  The default is 2 (i.e., the last 
      synchronized version plus one backup).
  \item For backward compatibility, the \verb|backups| preference is also
      still supported, but \verb|backup| is now preferred.
  \item It is OK to manually delete files from the backup directory (or to throw
      away the directory itself).  Before unison uses any of these files for 
      anything important, it checks that its fingerprint matches the one 
      that it expects. 
\end{itemize}

\item Merging:
\begin{itemize}
  \item Both user interfaces offer a new 'merge' command, invoked by pressing
      'm' (with a changed file selected).  
  \item The actual merging is performed by an external program.  
      The preferences \verb|merge| and \verb|merge2| control how this
      program is invoked.  If a backup exists for this file (see the
      \verb|backup| preference), then the \verb|merge| preference is used for 
      this purpose; otherwise \verb|merge2| is used.  In both cases, the 
      value of the preference should be a string representing the command 
      that should be passed to a shell to invoke the 
      merge program.  Within this string, the special substrings
      \verb|CURRENT1|, \verb|CURRENT2|, \verb|NEW|,  and \verb|OLD| may appear
      at any point.  Unison will substitute these as follows before invoking
      the command:
        \begin{itemize}
        \item \relax\verb|CURRENT1| is replaced by the name of the local 
        copy of the file;
        \item \relax\verb|CURRENT2| is replaced by the name of a temporary
        file, into which the contents of the remote copy of the file have
        been transferred by Unison prior to performing the merge;
        \item \relax\verb|NEW| is replaced by the name of a temporary
        file that Unison expects to be written by the merge program when
        it finishes, giving the desired new contents of the file; and
        \item \relax\verb|OLD| is replaced by the name of the backed up
        copy of the original version of the file (i.e., its state at the 
        end of the last successful run of Unison), if one exists 
        (applies only to \verb|merge|, not \verb|merge2|).
        \end{itemize}
      For example, on Unix systems setting the \verb|merge| preference to
\begin{verbatim}
   merge = diff3 -m CURRENT1 OLD CURRENT2 > NEW
\end{verbatim}
      will tell Unison to use the external \verb|diff3| program for merging.  

      A large number of external merging programs are available.  For 
      example, \verb|emacs| users may find the following convenient:
\begin{verbatim}
    merge2 = emacs -q --eval '(ediff-merge-files "CURRENT1" "CURRENT2" 
               nil "NEW")' 
    merge = emacs -q --eval '(ediff-merge-files-with-ancestor 
               "CURRENT1" "CURRENT2" "OLD" nil "NEW")' 
\end{verbatim}
(These commands are displayed here on two lines to avoid running off the
edge of the page.  In your preference file, each should be written on a
single line.) 

  \item If the external program exits without leaving any file at the
  path \verb|NEW|, 
      Unison considers the merge to have failed.  If the merge program writes
      a file called \verb|NEW| but exits with a non-zero status code,
      then Unison 
      considers the merge to have succeeded but to have generated conflicts.
      In this case, it attempts to invoke an external editor so that the
      user can resolve the conflicts.  The value of the \verb|editor| 
      preference controls what editor is invoked by Unison.  The default
      is \verb|emacs|.

  \item Please send us suggestions for other useful values of the
       \verb|merge2| and \verb|merge| preferences -- we'd like to give several 
       examples in the manual.
\end{itemize}
\end{itemize}

\item Smaller changes:
\begin{itemize}
\item When one preference file includes another, unison no longer adds the
  suffix '\verb|.prf|' to the included file by default.  If a file with 
  precisely the given name exists in the .unison directory, it will be used; 
  otherwise Unison will 
  add \verb|.prf|, as it did before.  (This change means that included 
  preference files can be named \verb|blah.include| instead of 
  \verb|blah.prf|, so that unison will not offer them in its 'choose 
  a preference file' dialog.)
\item For Linux systems, we now offer both a statically linked and a dynamically
  linked executable.  The static one is larger, but will probably run on more
  systems, since it doesn't depend on the same versions of dynamically
  linked library modules being available.
\item Fixed the \verb|force| and \verb|prefer| preferences, which were
  getting the propagation direction exactly backwards.
\item Fixed a bug in the startup code that would cause unison to crash
  when the default profile (\verb|~/.unison/default.prf|) does not exist.
\item Fixed a bug where, on the run when a profile is first created, 
  Unison would confusingly display the roots in reverse order in the user
  interface.
\end{itemize}

\item For developers:
\begin{itemize}
\item We've added a module dependency diagram to the source distribution, in
   \verb|src/DEPENDENCIES.ps|, to help new prospective developers with
   navigating the code. 
\end{itemize}
\end{changesfromversion}

\begin{changesfromversion}{2.6.11}
\item \incompatible{} Archive format has changed.  

\item \incompatible{} The startup sequence has been completely rewritten
and greatly simplified.  The main user-visible change is that the
\verb|defaultpath| preference has been removed.  Its effect can be
approximated by using multiple profiles, with \verb|include| directives
to incorporate common settings.  All uses of \verb|defaultpath| in
existing profiles should be changed to \verb|path|.

Another change in startup behavior that will affect some users is that it
is no longer possible to specify roots {\em both} in the profile {\em
  and} on the command line.

You can achieve a similar effect, though, by breaking your profile into
two:
\begin{verbatim}
  
  default.prf = 
      root = blah
      root = foo
      include common

  common.prf = 
      <everything else>
\end{verbatim}
Now do
\begin{verbatim}
  unison common root1 root2
\end{verbatim}
when you want to specify roots explicitly.

\item The \verb|-prefer| and \verb|-force| options have been extended to
allow users to specify that files with more recent modtimes should be
propagated, writing either \verb|-prefer newer| or \verb|-force newer|.
(For symmetry, Unison will also accept \verb|-prefer older| or
\verb|-force older|.)  The \verb|-force older/newer| options can only be
used when \verb|-times| is also set.

The graphical user interface provides access to these facilities on a
one-off basis via the \verb|Actions| menu.

\item Names of roots can now be ``aliased'' to allow replicas to be
relocated without changing the name of the archive file where Unison
stores information between runs.  (This feature is for experts only.  See
the ``Archive Files'' section of the manual for more information.)

\item Graphical user-interface:
\begin{itemize}
\item A new command is provided in the Synchronization menu for
  switching to a new profile without restarting Unison from scratch.
\item The GUI also supports one-key shortcuts for commonly
used profiles.  If a profile contains a preference of the form 
%
'\verb|key = n|', where \verb|n| is a single digit, then pressing this
key will cause Unison to immediately switch to this profile and begin
synchronization again from scratch.  (Any actions that may have been
selected for a set of changes currently being displayed will be
discarded.) 

\item Each profile may include a preference '\verb|label = <string>|' giving a
  descriptive string that described the options selected in this profile.
  The string is listed along with the profile name in the profile selection
  dialog, and displayed in the top-right corner of the main Unison window.
\end{itemize}

\item Minor:
\begin{itemize}
\item Fixed a bug that would sometimes cause the 'diff' display to order
  the files backwards relative to the main user interface.  (Thanks
  to Pascal Brisset for this fix.)
\item On Unix systems, the graphical version of Unison will check the
  \verb|DISPLAY| variable and, if it is not set, automatically fall back
  to the textual user interface.
\item Synchronization paths (\verb|path| preferences) are now matched
  against the ignore preferences.  So if a path is both specified in a
  \verb|path| preference and ignored, it will be skipped.
\item Numerous other bugfixes and small improvements.
\end{itemize}
\end{changesfromversion}

\begin{changesfromversion}{2.6.1}
\item The synchronization of modification times has been disabled for
  directories.

\item Preference files may now include lines of the form
  \verb+include <name>+, which will cause \verb+name.prf+ to be read
  at that point.

\item The synchronization of permission between Windows and Unix now
  works properly.

\item A binding \verb|CYGWIN=binmode| in now added to the environment
  so that the Cygwin port of OpenSSH works properly in a non-Cygwin
  context.

\item The \verb|servercmd| and \verb|addversionno| preferences can now
  be used together: \verb|-addversionno| appends an appropriate
  \verb+-NNN+ to the server command, which is found by using the value
  of the \verb|-servercmd| preference if there is one, or else just
  \verb|unison|.

\item Both \verb|'-pref=val'| and \verb|'-pref val'| are now allowed for
  boolean values.  (The former can be used to set a preference to false.)

\item Lot of small bugs fixed.
\end{changesfromversion}

\begin{changesfromversion}{2.5.31}
\item The \verb|log| preference is now set to \verb|true| by default,
  since the log file seems useful for most users.  
\item Several miscellaneous bugfixes (most involving symlinks).
\end{changesfromversion}

\begin{changesfromversion}{2.5.25}
\item \incompatible{} Archive format has changed (again).  

\item Several significant bugs introduced in 2.5.25 have been fixed.  
\end{changesfromversion}

\begin{changesfromversion}{2.5.1}
\item \incompatible{} Archive format has changed.  Make sure you
synchronize your replicas before upgrading, to avoid spurious
conflicts.  The first sync after upgrading will be slow.

\item New functionality:
\begin{itemize}
\item Unison now synchronizes file modtimes, user-ids, and group-ids.  

These new features are controlled by a set of new preferences, all of
which are currently \verb|false| by default.  

\begin{itemize}
\item When the \verb|times| preference is set to \verb|true|, file
modification times are propaged.  (Because the representations of time
may not have the same granularity on both replicas, Unison may not always
be able to make the modtimes precisely equal, but it will get them as
close as the operating systems involved allow.)
\item When the \verb|owner| preference is set to \verb|true|, file
ownership information is synchronized.
\item When the \verb|group| preference is set to \verb|true|, group 
information is synchronized.
\item When the \verb|numericIds| preference is set to \verb|true|, owner
and group information is synchronized numerically.  By default, owner and
group numbers are converted to names on each replica and these names are
synchronized.  (The special user id 0 and the special group 0 are never
mapped via user/group names even if this preference is not set.)
\end{itemize}

\item Added an integer-valued preference \verb|perms| that can be used to
control the propagation of permission bits.  The value of this preference
is a mask indicating which permission bits should be synchronized.  It is
set by default to $0o1777$: all bits but the set-uid and set-gid bits are
synchronised (synchronizing theses latter bits can be a security hazard).
If you want to synchronize all bits, you can set the value of this
preference to $-1$.

\item Added a \verb|log| preference (default \verb|false|), which makes
Unison keep a complete record of the changes it makes to the replicas.
By default, this record is written to a file called \verb|unison.log| in
the user's home directory (the value of the \verb|HOME| environment
variable).  If you want it someplace else, set the \verb|logfile|
preference to the full pathname you want Unison to use.

\item Added an \verb|ignorenot| preference that maintains a set of patterns 
  for paths that should definitely {\em not} be ignored, whether or not
  they match an \verb|ignore| pattern.  (That is, a path will now be ignored
  iff it matches an ignore pattern and does not match any ignorenot patterns.)
\end{itemize}
  
\item User-interface improvements:
\begin{itemize}
\item Roots are now displayed in the user interface in the same order
as they were given on the command line or in the preferences file.
\item When the \verb|batch| preference is set, the graphical user interface no 
  longer waits for user confirmation when it displays a warning message: it
  simply pops up an advisory window with a Dismiss button at the bottom and
  keeps on going.
\item Added a new preference for controlling how many status messages are
  printed during update detection: \verb|statusdepth| controls the maximum
  depth for paths on the local machine (longer paths are not displayed, nor
  are non-directory paths).  The value should be an integer; default is 1.  
\item Removed the \verb|trace| and \verb|silent| preferences.  They did
not seem very useful, and there were too many preferences for controlling
output in various ways.
\item The text UI now displays just the default command (the one that
will be used if the user just types \verb|<return>|) instead of all
available commands.  Typing \verb|?| will print the full list of
possibilities.
\item The function that finds the canonical hostname of the local host
(which is used, for example, in calculating the name of the archive file
used to remember which files have been synchronized) normally uses the
\verb|gethostname| operating system call.  However, if the environment
variable \verb|UNISONLOCALHOSTNAME| is set, its value will now be used
instead.  This makes it easier to use Unison in situations where a
machine's name changes frequently (e.g., because it is a laptop and gets
moved around a lot).
\item File owner and group are now displayed in the ``detail window'' at
the bottom of the screen, when unison is configured to synchronize them.
\end{itemize}

\item For hackers:
\begin{itemize}
\item Updated to Jacques Garrigue's new version of \verb|lablgtk|, which
  means we can throw away our local patched version.  

  If you're compiling the GTK version of unison from sources, you'll need
  to update your copy of lablgtk to the developers release.
  (Warning: installing lablgtk under Windows is currently a bit
  challenging.) 

\item The TODO.txt file (in the source distribution) has been cleaned up
and reorganized.  The list of pending tasks should be much easier to
make sense of, for people that may want to contribute their programming
energies.  There is also a separate file BUGS.txt for open bugs.
\item The Tk user interface has been removed (it was not being maintained
and no longer compiles).
\item The \verb|debug| preference now prints quite a bit of additional
information that should be useful for identifying sources of problems.
\item The version number of the remote server is now checked right away 
  during the connection setup handshake, rather than later.  (Somebody
  sent a bug report of a server crash that turned out to come from using
  inconsistent versions: better to check this earlier and in a way that
  can't crash either client or server.)
\item Unison now runs correctly on 64-bit architectures (e.g. Alpha
linux).  We will not be distributing binaries for these architectures
ourselves (at least for a while) but if someone would like to make them
available, we'll be glad to provide a link to them.
\end{itemize}

\item Bug fixes:
\begin{itemize}
\item Pattern matching (e.g. for \verb|ignore|) is now case-insensitive
  when Unison is in case-insensitive mode (i.e., when one of the replicas
  is on a windows machine).
\item Some people had trouble with mysterious failures during
  propagation of updates, where files would be falsely reported as having
  changed during synchronization.  This should be fixed.
\item Numerous smaller fixes.
\end{itemize}
\end{changesfromversion}

\begin{changesfromversion}{2.4.1}
\item Added a number of 'sorting modes' for the user interface.  By
default, conflicting changes are displayed at the top, and the rest of
the entries are sorted in alphabetical order.  This behavior can be
changed in the following ways:
\begin{itemize}
\item Setting  the \verb|sortnewfirst| preference to \verb|true| causes
newly created files to be displayed before changed files.
\item Setting \verb|sortbysize| causes files to be displayed in
increasing order of size.
\item Giving the preference \verb|sortfirst=<pattern>| (where
\verb|<pattern>| is a path descriptor in the same format as 'ignore' and 'follow'
patterns, causes paths matching this pattern to be displayed first.
\item Similarly, giving the preference \verb|sortlast=<pattern>| 
causes paths matching this pattern to be displayed last.
\end{itemize}
The sorting preferences are described in more detail in the user manual.
The \verb|sortnewfirst| and \verb|sortbysize| flags can also be accessed
from the 'Sort' menu in the grpahical user interface.

\item Added two new preferences that can be used to change unison's
fundamental behavior to make it more like a mirroring tool instead of
a synchronizer.
\begin{itemize}
\item Giving the preference \verb|prefer| with argument \verb|<root>|
(by adding \verb|-prefer <root>| to the command line or \verb|prefer=<root>|)
to your profile) means that, if there is a conflict, the contents of
\verb|<root>| 
should be propagated to the other replica (with no questions asked).
Non-conflicting changes are treated as usual.
\item Giving the preference \verb|force| with argument \verb|<root>|
will make unison resolve {\em all} differences in favor of the given
root, even if it was the other replica that was changed.
\end{itemize}
These options should be used with care!  (More information is available in
the manual.)

\item Small changes:
\begin{itemize}
\item 
Changed default answer to 'Yes' in all two-button dialogs in the 
  graphical interface (this seems more intuitive).

\item The \verb|rsync| preference has been removed (it was used to
activate rsync compression for file transfers, but rsync compression is
now enabled by default). 
\item  In the text user interface, the arrows indicating which direction
changes are being 
  propagated are printed differently when the user has overridded Unison's
  default recommendation (\verb|====>| instead of \verb|---->|).  This
  matches the behavior of the graphical interface, which displays such
  arrows in a different color.
\item Carriage returns (Control-M's) are ignored at the ends of lines in
  profiles, for Windows compatibility.
\item All preferences are now fully documented in the user manual. 
\end{itemize}
\end{changesfromversion}

\begin{changesfromversion}{2.3.12}
\item \incompatible{} Archive format has changed.  Make sure you
synchronize your replicas before upgrading, to avoid spurious
conflicts.  The first sync after upgrading will be slow.

\item New/improved functionality:
\begin{itemize}
\item  A new preference -sortbysize controls the order in which changes
  are displayed to the user: when it is set to true, the smallest
  changed files are displayed first.  (The default setting is false.) 
\item A new preference -sortnewfirst causes newly created files to be 
  listed before other updates in the user interface.
\item We now allow the ssh protocol to specify a port.  
\item Incompatible change: The unison: protocol is deprecated, and we added
  file: and socket:.  You may have to modify your profiles in the
  .unison directory.
  If a replica is specified without an explicit protocol, we now
  assume it refers to a file.  (Previously "//saul/foo" meant to use
  SSH to connect to saul, then access the foo directory.  Now it means
  to access saul via a remote file mechanism such as samba; the old
  effect is now achieved by writing {\tt ssh://saul/foo}.)
\item Changed the startup sequence for the case where roots are given but
  no profile is given on the command line.  The new behavior is to
  use the default profile (creating it if it does not exist), and
  temporarily override its roots.  The manual claimed that this case
  would work by reading no profile at all, but AFAIK this was never
  true.
\item In all user interfaces, files with conflicts are always listed first
\item A new preference 'sshversion' can be used to control which version
  of ssh should be used to connect to the server.  Legal values are 1 and 2.
  (Default is empty, which will make unison use whatever version of ssh
  is installed as the default 'ssh' command.)
\item The situation when the permissions of a file was updated the same on
  both side is now handled correctly (we used to report a spurious conflict)

\end{itemize}

\item Improvements for the Windows version:
\begin{itemize}
\item The fact that filenames are treated case-insensitively under
Windows should now be handled correctly.  The exact behavior is described
in the cross-platform section of the manual.
\item It should be possible to synchronize with Windows shares, e.g.,
  //host/drive/path.
\item Workarounds to the bug in syncing root directories in Windows.
The most difficult thing to fix is an ocaml bug: Unix.opendir fails on
c: in some versions of Windows.
\end{itemize}

\item Improvements to the GTK user interface (the Tk interface is no
longer being maintained): 
\begin{itemize}
\item The UI now displays actions differently (in blue) when they have been
  explicitly changed by the user from Unison's default recommendation.
\item More colorful appearance.
\item The initial profile selection window works better.
\item If any transfers failed, a message to this effect is displayed along with
  'Synchronization complete' at the end of the transfer phase (in case they
  may have scrolled off the top).
\item Added a global progress meter, displaying the percentage of {\em total}
  bytes that have been transferred so far.
\end{itemize}

\item Improvements to the text user interface:
\begin{itemize}
\item The file details will be displayed automatically when a
  conflict is been detected.
\item when a warning is generated (e.g. for a temporary
  file left over from a previous run of unison) Unison will no longer
  wait for a response if it is running in -batch mode.
\item The UI now displays a short list of possible inputs each time it waits
  for user interaction.  
\item The UI now quits immediately (rather than looping back and starting
  the interaction again) if the user presses 'q' when asked whether to 
  propagate changes.
\item Pressing 'g' in the text user interface will proceed immediately
  with propagating updates, without asking any more questions.
\end{itemize}

\item Documentation and installation changes:
\begin{itemize}
\item The manual now includes a FAQ, plus sections on common problems and
on tricks contributed by users.
\item Both the download page and the download directory explicitly say
what are the current stable and beta-test version numbers.
\item The OCaml sources for the up-to-the-minute developers' version (not
guaranteed to be stable, or even to compile, at any given time!) are now
available from the download page.
\item Added a subsection to the manual describing cross-platform
  issues (case conflicts, illegal filenames)
\end{itemize}

\item Many small bug fixes and random improvements.

\end{changesfromversion}

\begin{changesfromversion}{2.3.1}
\item Several bug fixes.  The most important is a bug in the rsync
module that would occasionally cause change propagation to fail with a
'rename' error.
\end{changesfromversion}

\begin{changesfromversion}{2.2}
\item The multi-threaded transport system is now disabled by default.
(It is not stable enough yet.)
\item Various bug fixes.
\item A new experimental feature: 

  The final component of a -path argument may now be the wildcard 
  specifier \verb|*|.  When Unison sees such a path, it expands this path on 
  the client into into the corresponding list of paths by listing the
  contents of that directory.  

  Note that if you use wildcard paths from the command line, you will
  probably need to use quotes or a backslash to prevent the * from
  being interpreted by your shell.

  If both roots are local, the contents of the first one will be used
  for expanding wildcard paths.  (Nb: this is the first one {\em after} the
  canonization step -- i.e., the one that is listed first in the user 
  interface -- not the one listed first on the command line or in the
  preferences file.)
\end{changesfromversion}

\begin{changesfromversion}{2.1}
\item The transport subsystem now includes an implementation by
Sylvain Gommier and Norman Ramsey of Tridgell and Mackerras's
\verb|rsync| protocol.  This protocol achieves much faster 
transfers when only a small part of a large file has been changed by
sending just diffs.  This feature is mainly helpful for transfers over
slow links---on fast local area networks it can actually degrade
performance---so we have left it off by default.  Start unison with
the \verb|-rsync| option (or put \verb|rsync=true| in your preferences
file) to turn it on.

\item ``Progress bars'' are now diplayed during remote file transfers,
showing what percentage of each file has been transferred so far.

\item The version numbering scheme has changed.  New releases will now
      be have numbers like 2.2.30, where the second component is
      incremented on every significant public release and the third
      component is the ``patch level.''

\item Miscellaneous improvements to the GTK-based user interface.
\item The manual  is now available in PDF format.

\item We are experimenting with using a multi-threaded transport
subsystem to transfer several files at the same time, making
much more effective use of available network bandwidth.  This feature
is not completely stable yet, so by default it is disabled in the
release version of Unison.

If you want to play with the multi-threaded version, you'll need to
recompile Unison from sources (as described in the documentation),
setting the THREADS flag in Makefile.OCaml to true.  Make sure that
your OCaml compiler has been installed with the \verb|-with-pthreads|
configuration option.  (You can verify this by checking whether the
file \verb|threads/threads.cma| in the OCaml standard library
directory contains the string \verb|-lpthread| near the end.)
\end{changesfromversion}

\begin{changesfromversion}{1.292}
\item Reduced memory footprint (this is especially important during
the first run of unison, where it has to gather information about all
the files in both repositories). 
\item Fixed a bug that would cause the socket server under NT to fail
  after the client exits. 
\item Added a SHIFT modifier to the Ignore menu shortcut keys in GTK
  interface (to avoid hitting them accidentally).  
\end{changesfromversion}

\begin{changesfromversion}{1.231}
\item Tunneling over ssh is now supported in the Windows version.  See
the installation section of the manual for detailed instructions.

\item The transport subsystem now includes an implementation of the
\verb|rsync| protocol, built by Sylvain Gommier and Norman Ramsey.
This protocol achieves much faster transfers when only a small part of
a large file has been changed by sending just diffs.  The rsync
feature is off by default in the current version.  Use the
\verb|-rsync| switch to turn it on.  (Nb.  We still have a lot of
tuning to do: you may not notice much speedup yet.)

\item We're experimenting with a multi-threaded transport subsystem,
written by Jerome Vouillon.  The downloadable binaries are still
single-threaded: if you want to try the multi-threaded version, you'll
need to recompile from sources.  (Say \verb|make THREADS=true|.)
Native thread support from the compiler is required.  Use the option
\verb|-threads N| to select the maximal number of concurrent 
threads (default is 5).  Multi-threaded
and single-threaded clients/servers can interoperate.  

\item A new GTK-based user interface is now available, thanks to
Jacques Garrigue.  The Tk user interface still works, but we'll be
shifting development effort to the GTK interface from now on.
\item OCaml 3.00 is now required for compiling Unison from sources.
The modules \verb|uitk| and \verb|myfileselect| have been changed to
use labltk instead of camltk.  To compile the Tk interface in Windows,
you must have ocaml-3.00 and tk8.3.  When installing tk8.3, put it in
\verb|c:\Tcl| rather than the suggested \verb|c:\Program Files\Tcl|, 
and be sure to install the headers and libraries (which are not 
installed by default).

\item Added a new \verb|-addversionno| switch, which causes unison to
use \verb|unison-<currentversionnumber>| instead of just \verb|unison|
as the remote server command.  This allows multiple versions of unison
to coexist conveniently on the same server: whichever version is run
on the client, the same version will be selected on the server.
\end{changesfromversion}

\begin{changesfromversion}{1.219}
\item \incompatible{} Archive format has changed.  Make sure you
synchronize your replicas before upgrading, to avoid spurious
conflicts.  The first sync after upgrading will be slow.

\item This version fixes several annoying bugs, including:
\begin{itemize}
\item Some cases where propagation of file permissions was not
working.
\item umask is now ignored when creating directories
\item directories are create writable, so that a read-only directory and
    its contents can be propagated.
\item Handling of warnings generated by the server.
\item Synchronizing a path whose parent is not a directory on both sides is
now flagged as erroneous.  
\item Fixed some bugs related to symnbolic links and nonexistant roots.
\begin{itemize}
\item 
   When a change (deletion or new contents) is propagated onto a 
     'follow'ed symlink, the file pointed to by the link is now changed.
     (We used to change the link itself, which doesn't fit our assertion
     that 'follow' means the link is completely invisible)
   \item When one root did not exist, propagating the other root on top of it
     used to fail, becuase unison could not calculate the working directory
     into which to write changes.  This should be fixed.
\end{itemize}
\end{itemize}

\item A human-readable timestamp has been added to Unison's archive files.

\item The semantics of Path and Name regular expressions now
correspond better. 

\item Some minor improvements to the text UI (e.g. a command for going
back to previous items)

\item The organization of the export directory has changed --- should
be easier to find / download things now.
\end{changesfromversion}

\begin{changesfromversion}{1.200}
\item \incompatible{} Archive format has changed.  Make sure you
synchronize your replicas before upgrading, to avoid spurious
conflicts.  The first sync after upgrading will be slow.

\item This version has not been tested extensively on Windows.

\item Major internal changes designed to make unison safer to run
at the same time as the replicas are being changed by the user.

\item Internal performance improvements.  
\end{changesfromversion}

\begin{changesfromversion}{1.190}
\item \incompatible{} Archive format has changed.  Make sure you
synchronize your replicas before upgrading, to avoid spurious
conflicts.  The first sync after upgrading will be slow.

\item A number of internal functions have been changed to reduce the
amount of memory allocation, especially during the first
synchronization.  This should help power users with very big replicas.

\item Reimplementation of low-level remote procedure call stuff, in
preparation for adding rsync-like smart file transfer in a later
release.   

\item Miscellaneous bug fixes.
\end{changesfromversion}

\begin{changesfromversion}{1.180}
\item \incompatible{} Archive format has changed.  Make sure you
synchronize your replicas before upgrading, to avoid spurious
conflicts.  The first sync after upgrading will be slow.

\item Fixed some small bugs in the interpretation of ignore patterns. 

\item Fixed some problems that were preventing the Windows version
from working correctly when click-started.

\item Fixes to treatment of file permissions under Windows, which were
causing spurious reports of different permissions when synchronizing
between windows and unix systems.

\item Fixed one more non-tail-recursive list processing function,
which was causing stack overflows when synchronizing very large
replicas. 
\end{changesfromversion}

\begin{changesfromversion}{1.169}
\item The text user interface now provides commands for ignoring
  files. 
\item We found and fixed some {\em more} non-tail-recursive list
  processing functions.  Some power users have reported success with
  very large replicas.
\item \incompatible 
Files ending in \verb|.tmp| are no longer ignored automatically.  If you want
to ignore such files, put an appropriate ignore pattern in your profile.

\item \incompatible{} The syntax of {\tt ignore} and {\tt follow}
patterns has changed. Instead of putting a line of the form
\begin{verbatim}
                 ignore = <regexp>
\end{verbatim}
  in your profile ({\tt .unison/default.prf}), you should put:
\begin{verbatim}
                 ignore = Regexp <regexp>
\end{verbatim}
Moreover, two other styles of pattern are also recognized:
\begin{verbatim}
                 ignore = Name <name>
\end{verbatim}
matches any path in which one component matches \verb|<name>|, while
\begin{verbatim}
                 ignore = Path <path>
\end{verbatim}
matches exactly the path \verb|<path>|.

Standard ``globbing'' conventions can be used in \verb|<name>| and
\verb|<path>|:  
\begin{itemize}
\item a \verb|?| matches any single character except \verb|/|
\item a \verb|*| matches any sequence of characters not including \verb|/|
\item \verb|[xyz]| matches any character from the set $\{{\tt x},
  {\tt y}, {\tt z} \}$
\item \verb|{a,bb,ccc}| matches any one of \verb|a|, \verb|bb|, or
  \verb|ccc|. 
\end{itemize}

See the user manual for some examples.
\end{changesfromversion}

\begin{changesfromversion}{1.146}
\item Some users were reporting stack overflows when synchronizing
  huge directories.  We found and fixed some non-tail-recursive list
  processing functions, which we hope will solve the problem.  Please 
  give it a try and let us know.
\item Major additions to the documentation.  
\end{changesfromversion}

\begin{changesfromversion}{1.142}
\item Major internal tidying and many small bugfixes.
\item Major additions to the user manual.
\item Unison can now be started with no arguments -- it will prompt
automatically for the name of a profile file containing the roots to
be synchronized.  This makes it possible to start the graphical UI
from a desktop icon.
\item Fixed a small bug where the text UI on NT was raising a 'no such
  signal' exception.
\end{changesfromversion}

\begin{changesfromversion}{1.139}
\item The precompiled windows binary in the last release was compiled
with an old OCaml compiler, causing propagation of permissions not to
work (and perhaps leading to some other strange behaviors we've heard
reports about).  This has been corrected.  If you're using precompiled
binaries on Windows, please upgrade.
\item Added a \verb|-debug| command line flag, which controls debugging
of various modules.  Say \verb|-debug XXX| to enable debug tracing for
module \verb|XXX|, or \verb|-debug all| to turn on absolutely everything.
\item Fixed a small bug where the text UI on NT was raising a 'no such signal'
exception.
\end{changesfromversion}

\begin{changesfromversion}{1.111}
\item \incompatible{} The names and formats of the preference files in
the .unison directory have changed.  In particular:
\begin{itemize}
\item the file ``prefs'' should be renamed to default.prf
\item the contents of the file ``ignore'' should be merged into
  default.prf.  Each line of the form \verb|REGEXP| in ignore should
  become a line of the form \verb|ignore = REGEXP| in default.prf.
\end{itemize}
\item Unison now handles permission bits and  symbolic links.  See the
manual for details.

\item You can now have different preference files in your .unison
directory.  If you start unison like this
\begin{verbatim}
             unison profilename
\end{verbatim}
(i.e. with just one ``anonymous'' command-line argument), then the
file \verb|~/.unison/profilename.prf| will be loaded instead of
\verb|default.prf|. 

\item Some improvements to terminal handling in the text user interface

\item Added a switch -killServer that terminates the remote server process
when the unison client is shutting down, even when using sockets for 
communication.  (By default, a remote server created using ssh/rsh is 
terminated automatically, while a socket server is left running.)
\item When started in 'socket server' mode, unison prints 'server started' on
  stderr when it is ready to accept connections.  
  (This may be useful for scripts that want to tell when a socket-mode server 
  has finished initalization.)
\item We now make a nightly mirror of our current internal development
  tree, in case anyone wants an up-to-the-minute version to hack
  around with.
\item Added a file CONTRIB with some suggestions for how to help us
make Unison better.
\end{changesfromversion}



\finishlater{
\SECTION{Other Synchronizers}{other}{other}

Unison is just one of several file synchronizers that are currently
available. 

Check out:
  http://www.bell-labs.com/project/stage/
  I notice a bunch of people are also doing "data vaulting", e.g.,
    http://www.pc.ibm.com/us/thinkpad/datavault.html
  midnight commander??

Also:
  D. Duchamp
  A Toolkit Approach to Partially Disconnected Operation
  Proc. USENIX 1997 Ann. Technical Conf.
  USENIX, Anaheim CA, pp. 305-318, January 1997
}

\finishlater{
\SECTION{TODO}{todo}{ }

Things to write about:
\begin{itemize}
\item When started in 'socket server' mode, Unison prints 'server started' on
  stderr when it is ready to accept connections.  
  (This may be useful for scripts that want to tell when a socket-mode server 
  has finished initialization.)
\item {\tt DANGER.README}.
\end{itemize}
}

\finishlater{
Things to write about later:
\begin{itemize}
\item Document different reporting of file status when no archives
  were found.
\item Document buttons in graphical UI
\end{itemize}
}

\iftextversion
\SECTION{Junk}{ }{ } 
\fi

\ifhevea\begin{rawhtml}</div>\end{rawhtml}\fi

\end{document}
